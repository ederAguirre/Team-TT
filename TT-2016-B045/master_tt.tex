\documentclass[12pt,oneside,onecolumn]{book}      
%\usepackage[nottoc]{tocbibind}
\usepackage[spanish]{babel}
\usepackage[T1]{fontenc} %Paquetes de Fuentes que queramos usar
\usepackage[utf8]{inputenc}
\usepackage[spanish]{babel}
\usepackage{amsmath,amssymb,amsfonts}
\usepackage{fancyhdr}
\usepackage{graphicx} % Imagenes 
\usepackage{cite} 
\usepackage{hhline}
\usepackage{multicol}
\usepackage{longtable}
\usepackage{amssymb}
\usepackage{t1enc}
\usepackage[letterpaper, left=3cm, right=2cm, top=2cm, bottom=2cm]{geometry} 
\usepackage{amsmath}
\usepackage{color}
%\usepackage[export]{adjustbox}
\usepackage{pdfpages}
\usepackage{subcaption}
\usepackage{hyperref}
\usepackage{listings}
\usepackage{float}
\usepackage{textcomp}
\usepackage{appendix}
\usepackage{multirow}
\usepackage{longtable}
\usepackage{verbatim} 
%\usepackage{usecases}
\usepackage{casouso_estilo} 


\AtBeginDocument{%
  \renewcommand\tablename{Tabla}
  \renewcommand{\bibname}{Referencias}
  \renewcommand{\contentsname}{Índice}
  \renewcommand{\listfigurename}{Índice de Figuras}
  \renewcommand{\listtablename}{Índice de Tablas}

}

% Redefinition of ToC command to get centered heading
\makeatletter
\renewcommand\tableofcontents{%
  \null\hfill\textbf{\Large\contentsname}\hfill\null\par
  \@mkboth{\MakeUppercase\contentsname}{\MakeUppercase\contentsname}%
  \@starttoc{toc}%
}
\makeatother
%Para encabezados y Pies de Pagina
\pagestyle{fancy}
   \lhead{}
    \chead{}
    \rhead{}
    \lfoot{}
    \cfoot{}
    \rfoot{\thepage}


\fancyhead[R]{}
    \renewcommand{\headrulewidth}{0pt}
%   \renewcommand{\footrulewidth}{0.4pt}


\fancypagestyle{plain}{%
    \renewcommand{\headrulewidth}{0pt}%
    \fancyhf{}%
    \fancyfoot[R]{\thepage}%
}

 \newtheorem{theorem}{Teorema}[chapter]
%\newtheorem{corollary}[theorem]{}
%\newtheorem{claim}{Claim}
\newtheorem{definition}{Definici\'on}[chapter]
\newtheorem{example}{Ejemplo}[chapter]
\newtheorem{proposition}{Proposici\'on}[chapter]
\newtheorem{myclaim}{Claim}[chapter]

\newcommand{\rand}{\stackrel{\$}\leftarrow}



%opening
%\title{}
%\author{}

\begin{document}

\begin{titlepage}
	\parbox{2cm}{
	\begin{picture}(18,4)
	    \put(-21,240){\includegraphics[width=2cm,height=3cm]{./images/IPN.jpg}}
	    \put(0,-280){\includegraphics[width=0.5cm,height=18.3cm]{./images/lineaAzul.jpg}}
	    \put(-37,-335){\includegraphics[width=3cm,height=3cm]{./images/ESCOM.jpg}}
	    \end{picture}}
	\parbox{14cm}{\vspace{1cm} 
	    
	    {\fontsize{19}{30} \textbf{  INSTITUTO POLIT\'ECNICO NACIONAL}}
	    \begin{center}
	    {\fontsize{16}{20} \textbf{Escuela Superior de C\'omputo}}\vspace{1cm}\\
	    {\fontsize{18}{20} \textbf{ESCOM}}\vspace{2cm}\\
	    
	    {\fontsize{14}{20} \textit{Trabajo Terminal}}\vspace{1cm}\\
	    {\fontsize{16}{20} \textbf{``Protocolo criptográfico para el almacenamiento sin duplicados en la nube, resistente a ataques por fuerza bruta.''}}\vspace{1cm}\\
	    {\fontsize{14}{20} \textit{2016-B045}}\vspace{1cm}\\
	    {\fontsize{14}{20} \textit{Presentan}}\\
	    {\fontsize{14}{20} \textbf{Eder Jonathan Aguirre Cruz \\ Diana Leslie González Olivier \\ \vspace{.2cm} Jhonatan Saulés Cortés }}\vspace{2.5cm}\\
	    {\fontsize{14}{20} \textit{Directora}}\\
	    {\fontsize{14}{20} \textbf{Dra. Sandra D\'\i az Santiago}}   \vspace{3cm}
	    \end{center}

	       \hfill   \fontsize{14}{20}  \textbf{Mayo 2017} 
	    
	}
\end{titlepage}

\frontmatter
%----------------------------INDICE------------------------------------------------------------------------------------------
\tableofcontents
\listoffigures

\listoftables
\mainmatter
%------------------------------------------------------------------------------------------------------------------------------------
\chapter{Introducción}

\cfinput{archivosTex/introduccion}

%------------------------------------------------------------------------------------------------------------------------------------
\chapter{Marco Teórico}

\cfinput{archivosTex/preliminares}

%------------------------------------------------------------------------------------------------------------------------------------

\chapter{Estado del Arte.}

\cfinput{archivosTex/estado-arte}

%------------------------------------------------------------------------------------------------------------------------------------
\chapter{Protocolo Dupless}

\cfinput{archivosTex/protocoloDupless}

%------------------------------------------------------------------------------------------------------------------------------------
\chapter{Análisis}

\cfinput{archivosTex/analisis}

%------------------------------------------------------------------------------------------------------------------------------------
\chapter{Diseño}

\cfinput{archivosTex/diseño}

%------------------------------------------------------------------------------------------------------------------------------------
\appendix
\chapter{Lista de acrónimos}
\cfinput{archivosTex/acronimos}
\chapter{Glosario de términos}
\cfinput{archivosTex/GlosarioTerminos}


%\chapter{Requerimientos del sistema }

\section{Requerimientos Funcionales. }

\begin{table}[htb]
\centering
\begin{tabular}{| p{2.5cm} |  p{13.5cm} |}
\hline
\multicolumn{2}{|c|}{Servidor de Llaves} \\ \hline
ID & Descripción \\
\hline \hline
RF – SLL1 & El sistema permitirá gestionar estados para la administración de llaves de usuario a través de una clave secreta (Ks) propia del servidor de llaves. \\ \hline
RF – SLL2 & El sistema debe gestionar 3 estados en el servidor de llaves: Generación, Cambios y Eliminación de llaves de usuarios para manipular un archivo. \\ \hline
RF – SLL3 & El sistema comenzará la creación de una llave (K) cuando el usuario solicite la carga de un archivo a almacenar. \\ \hline
\end{tabular}
\caption{Requerimientos funcionales del servidor de llaves}
\label{Servidor de Llaves }
\end{table}
%\chapter{Adversarios Clasificadores}



Los adversarios clasificadores son programas de c\'omputo que se dedican a observar  mensajes que se intercambian  entre  los  usuarios  de  correo  electrónico,  con  el  fin  de  clasificarlos  e 
identificar  a  todos  los  usuarios  que  cumplan  con  cierto  criterio.  Esta  clasificación  se hace de manera masiva a trav\'es de  una búsqueda de palabras clave dentro de 
los  mensajes de  los usuarios. Por ejemplo, el  clasificador puede estar interesado en  los mensajes  que  contienen  la  palabra  clave  "Bomba",  así  que 
todos  los  mensajes  que contengan esta palabra serán etiquetados en una clasificación en específico, este proceso se lleva a cabo por medio de técnicas de 
“Reconocimiento de patrones” y “Aprendizaje Máquina” para encontrar y clasificar los mensajes 
que intercepta~\cite{clas,Attacks}.
\\
La  clasificación  de  estos  mensajes  tiene  diversos  usos, ya  que  pueden  ser  clasificados con  fines demográficos, con  fines comerciales o con  fines gubernamentales. Todo esto 
con  el  propósito  de  generar  las  estadísticas  de  comportamientos  e  intereses  de  los usuarios de correo electrónico. \\




En este trabajo terminal, se considera que un adversario clasificador solo es capaz de realizar ataques de solo texto cifrado (ciphertext only attack, en ingl\'es). Como se mencion\'o en el cap\'itulo anterior, en este ataque  el adversario solo  cuenta  con  los  textos  cifrados  que  va  recopilando  de  un  canal  o  base  de 
datos. Posteriormente, el adversario utiliza  estos  textos  cifrados  para  hacer  un  análisis  criptográfico  de  cómo  se comporta la técnica de cifrado  y tratar de hallar el texto en claro a partir de los textos cifrados que va recopilando. 
\\
Este  tipo  de  ataques  es  muy  común  en  el  internet  aunque  con  muy  baja  efectividad cuando   se   implementa   en   comunicaciones   altamente   protegidas,   y   cuando   se 
implementa en canales de comunicación desprotegidos  la  información obtenida  llega  a ser  muy  pobre.  En  los  últimos  años  se  han  dado  cuenta  que  si   este  tipo  de 
adversarios   atacan las  comunicaciones  sin  cifrado se  obtienen  características  valiosas  sobre  los usuarios  que  utilizan  este  tipo  de  canales  de  comunicación,  este  tipo  de  ataques  son 
ejecutados por  adversarios clasificadores.\\


\subsection{Esquema Golle - Farahat}
La  única  referencia  que se tiene  sobre  un  esquema  criptográfico  contra  adversarios clasificadores es el que propusieron Golle y Farahat~\cite{Attacks}. En este art\'iculo se habla por primera vez de las caracter\'isticas de este tipo de adversarios  y se considera la posibilidad de utilizar un esquema de cifrado con un nivel de seguridad menor.
%en el artículo 
%“Defending Email Communication Against Profiling” de Philippe Golle  y  Ayman  Farahat, ambos  miembros del 
%“Palo Alto Research Center”.\cite{Attacks}\\
%En su artículo se aborda el ataque de adversarios clasificadores sobre los mensajes de correo electrónico, en el cual se proponen un protocolo para la comunicación por correo electrónico. 
Golle y Farahat proponen un protocolo que hace uso de una función  de cifrado, el cual sustituye cada una de las palabras del mensaje por otra de la misma extensión y frecuencia gramatical, esta función esta pensada para textos en idioma ingl\'es. 
%Esta función tiene como parámetro una clave $K$.  
Para cifrar se utiliza una clave que se genera  usando los datos de cabecera que acompañan al mensaje los cuales pueden ser dirección  del remitente, la dirección del destinatario, la hora a la que se envía el correo electrónico y potencialmente otros campos. Estos datos se introducen en una función hash lenta y el resultado de esta función es la clave $K$. Estas funciones hash  tiene con una complejidad de c\'alculo moderadamente mas alta que las 
funciones hash est\'andar.  \\
Este  protocolo resulta inseguro para la criptografía moderna pero es efectivo contra el ataque de clasificadores. Por otro lado este protocolo resuelve dos problemas, le  permite  a los usuarios  calcular  la  clave de cifrado y descifrado  fácilmente  ya  que  los datos del mensaje con que se calcula son  públicos, resolviendo así el intercambio de claves. Al usar un cifrado de tipo semántico se permite que el texto se vea como un texto en inglés pero indistinguible para los clasificadores y por lo tanto este clasifica incorrectamente el mensaje cifrado.\\

\section{CAPTCHA}
 Es un programa informático diseñado para diferenciar un ser humano de una computadora, CAPTCHA son las siglas de prueba de Turing completamente automática y pública para diferenciar computadoras de humanos ({\it Completely Automated Public Turing test to tell Computers and Humans Apart} por sus siglas en ingl\'es).  Un CAPTCHA es una prueba que es fácil de pasar por un usuario humano pero difícil de pasar por una máquina. Uno de los CAPTCHAs más com\'unes son imágenes distorsionadas de cadenas cortas de caracteres. Para un humano es generalmente muy fácil recuperar la cadena original de la imagen distorsionada, pero es difícil para  los algoritmos de reconocimiento de caracteres recuperar la cadena original de la imagen distorsionada.\\
Un CAPTCHA en un algoritmo aleatorio $G$, que recibe como parámetro una cadena de caracteres STR y produce como resultado un CAPTCHA $G(x)$\\

\begin{figure}[H]
\centering
	\includegraphics[width=5cm, height=3cm]{./images/cptc.jpeg}
	\caption{CAPTCHA}
	\label{fig:3-6}
\end{figure}
 
%La criptografía moderna se basa en dos corrientes metodológicas que son la criptografía simétrica  y  la criptografía asimétrica. Estas dos técnicas tienen como propósito ocultar 
%el  contenido  de  un  mensaje  con  el  fin  de  que  solo  sea  leído  por  aquellos  que  estén autorizados, a esto se le llama cifrado. 




\section{Esquema de Secreto Compartido de Shamir}
El esquema de secreto compartido fue propuesto por Adi Shamir en 1997\cite{shamir}. 
El objetivo de este método es dividir un secreto $K$ en $w$ partes,que son dadas a $w$ participantes. Para recuperar el secreto es necesario tener al menos $u$
elementos de las $w$ partes siendo $u \leq w$. Y no es posible recuperar el secreto si se tienen menos que $u$ partes.

Para construir el esquema del secreto compartido primero es necesario seleccionar un número primo $p \geq w+1$ el cual define al conjunto $\mathbb{Z}_p$.
\\
\\
El procedimiento para dividir un secreto $K$ en $w$ partes es el siguiente:

\begin{enumerate}
 \item Se seleccionan $w$ elementos distintos de cero del conjunto $\mathbb{Z}_p$ denotados como $x_i$ donde $1 \leq i \leq w$.
 \item Se seleccionan $u-1$ elementos aleatorios de $\mathbb{Z}_p$ denotados como $a_1,...,a_{u-1}$.
 \item Se construye el polinomio $y_x$ de la siguiente forma. Sea \begin{equation}
            y(x)=K+\sum_{j=1}^{u-1} a_j x^j \bmod \quad p
           \end{equation}
           Por medio de este polinomio se calculan los elementos $y_i$.
 \item La salida es el conjunto $S=\{(x_1,y_1),...,(x_w,y_w)\}$.
\end{enumerate}

Para recuperar el secreto solo tenemos que resolver un sistema de ecuaciones  que es definido por el polinomio característico $a(x)=a_0+a_1x+...+a_{u-1}x^{u-1}$.
\\
\\
Posteriormente se seleccionan $u$ pares de elementos $(x_w,y_w)$ con los que obtendremos nuestro sistema de ecuaciones a resolver. El elemento que nos interesa obtener del sistema de ecuaciones es $a_0$ ya que este es el valor de nuestro secreto $K$.
\begin{example}
Si se considera el conjunto  $\mathbb{Z}_{11}$ y se desea compartir el secreto $K=8$, entre 5 participantes,
de tal manera que solo cuando se re\'unan cualesquiera 2 de ellos sea posible recuperar $K$, es decir
 $w=5$ y $u=2$.

Se seleccionan los $u-1$ elementos del conjunto $\mathbb{Z}_{11}$, puesto que $u=2$ en este caso solo 
hay que escoger un elemento: $a_1=5$.

A continuaci\'on se seleccionan $w$ elementos de $\mathbb{Z}_{11}$, por ejemplo 
 $x_1=2, x_2=7, x_3=9, x_4=10, x_5=3$.

Posteriormente, se calcula el conjunto de elementos $y_i$ por medio de la ecuaci\'on 
\begin{equation} \nonumber
 y_i=k+\sum_{j=1}^{u-1} a_j x_i^j \bmod \quad p
\end{equation}

En el caso del ejemplo, la ecuaci\'on anterior queda como sigue:

\begin{equation} \nonumber
 y_i=K+a_1x_i \bmod \quad p
\end{equation}
Y se obtienen los siguientes valores:
$$\begin{array}{lcl}
y_1&=&8+5(2) \bmod 11=7, \\
y_2&=&8+5(7) \bmod 11=10,\\
y_3&=&8+5(9) \bmod 11=9, \\ 
y_4&=&8+5(10) \bmod11=3, \\
y_5&=&8+5(3) \bmod 11=1. 
\end{array}$$

Finalmente, se tienen los pares $S=\{ (2,7), (7,10), (9,9), (10,3),(3,1)\}$
%$A_1(2,7)$\hspace{1cm}$A_2(7,10)$\hspace{1cm}$A_3(9,9)$\hspace{1cm}$A_4(10,3)$\hspace{1cm}$A_5(3,1)$\\


Para recuperar la llave $K$ es necesario seleccionar $u=2$ pares del conjunto $S$, por ejemplo
$A_2(7,10)$, $A_4(10,3)$. Con estos pares se puede crear el siguiente  sistema de ecuaciones:
$$\begin{array}{lcl}
 a_0+a_1x_2&=&y_2 \\
 a_0+a_1x_4&=&y_4 
 \end{array}$$
 Es importante notar, que en tal sistema de ecuaciones, las inc\'ognitas son $a_0$ y $a_1$, las cuales
 son desconocidas para los $w$ participantes.  Para el esquema de secreto compartido de
 Shamir, es de particular inter\'es $a_0$, ya que  $a_0=K$.
Al sustituir los pares $A_2$ y $A_4$ en el sistema de ecuaciones anterior, se tiene:
\begin{align}
a_0+7a_1&=10 \label{eq-3}\\
a_0+10a_1&=3 \label{eq-4}
\end{align}
Para resolver este sistema se puede utilizar cualquiera de los métodos com\'unes que se usan en álgebra, solo que respetando el conjunto $\mathbb{Z}_p$, en este caso se resolverá por el método suma y resta.\\
%\begin{equation*}
% a_0+7a_1=10
%\end{equation*}
%\begin{equation}
% a_0+10a_1=3
%\end{equation}
Multiplicamos la ecuación~(\ref{eq-3}) por $-1$ y obtenemos:
\begin{equation}
 -a_0-7a_1=-10 \label{eq-5}
\end{equation}
%\begin{equation}
% a_0+10a_1=3
%\end{equation}
sumamos la ecuación~(\ref{eq-4}) con (\ref{eq-5})  d\'andonos como resultado:
\begin{equation} \nonumber
 3a_1=4
\end{equation}
de donde es posible despejar$a_1$
\begin{equation}
 a_1=\frac{4}{3} \label{eq-8}
\end{equation}
Puesto que 4 es el inverso multiplicativo de 3,  la ecuación~(\ref{eq-8}) queda de la siguiente forma
\begin{equation} \nonumber
 a_1=(4)(4)=16 \bmod 11 =5
\end{equation}
Sustituimos $a_1$ en la ecuación~(\ref{eq-4}) 
\begin{equation} \nonumber
 a_0+10(5)=3
\end{equation}
Simplificamos y despejamos $a_0$
\begin{align*}
 a_0+(50 \bmod 11)=3 \\
  a_0=-3 \bmod11=8
\end{align*}
Como $a_0=8$ podemos ver que se recuper\'o a $K$ exitosamente ya que  $a_0=K$.
\end{example}


\section{Esquema Díaz - Chakraborty}
Otro esquema para combatir a los adversarios clasificadores fue propuesto en 2012 por D\'iaz y 
Chakraborty~\cite{clas}.  El esquema D\'iaz-Chakraborty utiliza CAPTCHAs y un algoritmo de clave secreta,
para  proteger el correo electr\'onico. 
 Para obtener la clave se genera una cadena al azar, a la se le aplica una funci\'on hash, con esta clave se cifra el
 mensaje.  Tanto el mensaje cifrado como el CAPTCHA se env\'ian al receptor. El receptor debe
resolver el CAPTCHA, para obtener la cadena, aplicarle la funci\'on hash y as\'i obtener la clave de cifrado. Puesto que un adversario clasificador es un programa de c\'omputo, no podr\'a resolver un CAPTCHA y por tanto no podr\'a obtener la clave de cifrado.  Este esquema se muestra en la Figura~\ref{fig:protocol}.

\begin{figure}[h]
\centering
%\subfigure{
\begin{tabular}{|l|}
\hline
\begin{minipage}{220pt}
\begin{tabbing}
\ \ \ \ \ \=\ \ \ \ \=\ \ \ \ \=\ \ \ \ \=\ \ \ \ \=\ \ \ \ \=\ \ \
\ \kill \\
\ \ \ \ {\bf Protocol} ${\mathbb P}(x)$\\
\> 1. \> $k\rand {\sf STR}$; \\
\> 2. \> $k^{\prime} \leftarrow G(k)$; \\
\> 3. \> $K \leftarrow H(k)$; \\
\> 4. \> $c\leftarrow E_K(x)$;\\
\> 5. \> {\bf return} $(c, k^{\prime})$
\end{tabbing}
\end{minipage}\\
\hline
\end{tabular}
\caption{\label{fig:protocol} Protocol D\'iaz-Chakraborty.}
\end{figure}

 
%\begin{enumerate}
% \item \textbf{Esquema P}: Se genera una cadena de caracteres aleatoriamente llamada STR la cual es introducida a una funcion Hash para generar la clave $K$, con esta clave se cifra el mensaje original $M$. La cadena STR es convertida en un CAPTCHA y se envía junto con el texto cifrado al destinatario.\\\\
% \textbf{Esquema} P(x)
% \begin{enumerate}
%  \item $k\longleftarrow^\$ STR;$
%  \item $k'\longleftarrow G(k);$
%  \item $K\longleftarrow H(k);$
%  \item $c\longleftarrow E_k(x);$
%  \item $return(c,k')$
% \end{enumerate}
 Contemplando que es muy común que el usuario no consiga resolver el CAPTCHA D\'iaz y Chakraborty propusieron una variante del esquema anterior, el cual se describe a continuaci\'on. 
 Se genera una cadena de caracteres aleatoriamente llamada STR la cual se codifica a un valor entero. El valor entero es dividido en 5 pares $(x,k')$ por medio del algoritmo de Secreto Compartido, cada uno de los elementos $k'$ de los pares generados es decodificado a su correspondiente valor en cadena de caracteres para posteriormente ser convertidos en CAPTCHAS. Para finalizar la cadena STR se introduce en una función Hash para generar la llave $K$. Con esta llave se cifra el mensaje de correo y se envía junto con los pares de $(x,CAPTCHA)$.
Este esquema se puede observar en la Figura~\ref{fig:P-with-sharing}.
\begin{figure}[h]
\begin{center}
\begin{tabular}{|l|}
\hline
\begin{minipage}{220pt}
\begin{tabbing}
\ \ \ \ \ \=\ \ \ \ \=\ \ \ \ \=\ \ \ \ \=\ \ \ \ \=\ \ \ \ \=\ \ \
\ \kill \\
\ \ \ \ {\bf Protocol} ${\mathbb P}^{\prime}(x)$\\
\> 1. \> $k\rand {\sf STR}$; \\
\> 2. \> $k^\prime \leftarrow {\sf ENCD}(k,0);$\\
\> 3. \> $\{ (x_1, k_1^\prime), \ldots, (x_w, k_w^\prime)\} \leftarrow {\sf SHARE}^p_{u,w}(k^\prime)
$; \\
\> 4. \> {\bf for} $i=1$ {\bf to } $w$; \\
\> 5. \>\> $(k_i, \lambda_i) \leftarrow {\sf ENCD}^{-1}(k_i^\prime)$; \\
\> 6. \>\> $c_i \leftarrow G(k_i)$; \\
\> 7. \>{\bf end for} \\
\> 8. \> $K \leftarrow H(k)$; \\
\> 9. \> $C\leftarrow E_K(x)$; \\
\> 10. \> \hspace*{2mm}{\bf return} $[C,\{ (x_1, c_1, \lambda_1), \ldots, (x_w,
c_w, \lambda_w)\}]$
\end{tabbing}
\end{minipage}\\
\hline
\end{tabular}
\end{center}
\caption{\label{fig:P-with-sharing} Variante del  protocolo D\'iaz-Chakraborty}
\end{figure} 


% \begin{enumerate}
%  \item $k\longleftarrow^\$STR;$
%  \item $k'\longleftarrow ENCD(k);$
%  \item $\{(x_1,k'_1),...,(x_w,k'_w)\}\longleftarrow SHARE^p_{u,w}(k');$
%  \item $for\quad i=1\quad to \quad w;$
%  \item $\quad (k_i)\longleftarrow ENCD^{-1}(k'_1);$
%  \item $\quad c_i\longleftarrow G(k_i);$
%  \item $end for$
%  \item $K\longleftarrow H(k);$
%  \item $C\longleftarrow E_k(x);$
%  \item $return [C,\{(x_1,c_1),...,(x_w,c_w)\}]$
% \end{enumerate}


Este nuevo esquema se cre\'o pensando en que el usuario pueda tener más oportunidades de recuperar el mensaje cifrado y esto sucede gracias a el algoritmo de Secreto Compartido, ya que no este podemos tener la misma llave repartida en $n$ CAPTCHAS. A continuaci\'on se describe las funciones {\sf ENCD} y ${\sf ENCD}^{-1}$, cuyo
prop\'osito es convertir una cadena de caracteres a enteros y viceversa. 

\subsection{Codificaci\'on de caracteres a enteros}
Se tiene un conjunto de caracteres $AL$ compuesto por $AL=\{A,B,...,Z\}\cup\{a,b,...,z\}\cup\{0,1,...,9\}\cup\{+,/\}$ con una cardinalidad $|AL|=64$.

Para obtener una representación binaria de 64 elementos son necesarios 6 bits por lo que para todos los elementos $\sigma\epsilon AL$ existe una cadena binaria. Una vez establecido esto el procedimiento para realizar la conversión es el siguiente:
\begin{enumerate}
 \item Tomamos una cadena de caracteres y la separamos caracter por caracter y los intercambiamos por su correspondiente número entero en $AL$ $\alpha _0||\alpha _1||...||\alpha _m$
 \item Posteriormente cada uno de los enteros lo convertimos en un binario de 6 bits y se concatenan uno detrás del otro $\Psi\longleftarrow bin_6(\alpha _0)||bin_6(\alpha _1)||...||bin_6(\alpha _m$
 \item La cadena binaria $\Psi$ la convertimos a entero $v\longleftarrow toInt(\Psi)$
\end{enumerate}
\textbf{Ejemplo}:\\
Tenemos la cadena $STR=`ABC`$ de la cual cambiaremos cada caracter por su correspondiente valor entero en $AL$ quedando de la siguiente manera $\alpha =\{0,1,2\}$
\\ 
\\
Ahora cada  uno de los elementos de $\alpha$ lo convertiremos a su correspondiente representacion binaria, $bin_6(0)=000000, bin_6(1)=000001, bin_6(2)=000010$ y concatenamos cada una quedando $\Psi = 000000000001000010$. 


La cadena binaria $\Psi$ se convertirá en un entero $v=toInt(\Psi )$ que da como resultado $v=66$. El entero $v$ que obtenemos es el valor entero.

\subsection{Decodificación de enteros a caracteres}

También es necesario convertir un entero a una cadena de caracteres y para esto se realiza el proceso inverso:
\begin{enumerate}
 \item El entero $v$ es convertido en un número binario  $z=toBin_6(v)$ 
 \item Separamos $z$ en cadenas de 6 bits y cada una de ellas la interpretamos como un entero $toInt(z_0)||toInt(z_1)||...||toInt(z_w)$
 \item Cada uno de estos valores se convierte a su correspondiente caracter en $AL$ se concatenan para generar la cadena de caracteres final.
\end{enumerate}
\textbf{Ejemplo}:\\
El entero $v=66$ se representa como una cadena de 18 bits $z=000000000001000010$, la cual se divide en sub cadenas 6 bits quedando $z_0=000000, z_1=000001, z_2=000010$, para cada uno de estos números binarios se procede a convertirlo en un entero $toInt(z_0)=0, toInt(z_1)=1, toInt(z_2)=2$, por último estos son intercambiados por sus correspondientes caracteres en $AL$ y concatenados resultando en $s=`ABC`$




        

%\chapter{Tecnologías usadas}

Tomando en cuenta la información ya vertida en este documento, a continuación se explicará detalladamente la propuesta de solución.\\
En la figura \ref{fig:4-1-1} se tiene el diagrama general del sistema, se puede apreciar la comunicación entre las diferentes entidades que se usaran, que datos se mandan y reciben y por que canales transitan estos datos. A continuación se describe de manera general como es el proceso de envío y recepción de correos electrónicos ideado para este esquema.\\
\begin{enumerate}
 \item {Envío}
\begin{itemize}
\item El remitente escribe el correo electrónico y le da enviar.\\
\item El correo electrónico pasa por el complemento del cliente de correo.\\
\item El cliente genera a partir del correo una clave que usaremos para cifrar el mensaje.\\
\item Se cifra y se empaqueta el mensaje con el protocolo SMTP.\\
\item Se coloca una bandera en el mensaje.\\
\item La clave se convierte en CAPTCHA y es enviada al servidor de CAPTCHAS.\\
\item Se envía el mensaje de correo electrónico al destinatario.\\
\end{itemize}

\item{Recepción}
\begin{itemize}
\item El receptor abre un correo electrónico cifrado con el presente esquema.\\
\item El cliente lo descarga del servidor por medio del protocolo POP3 o IMAP.\\
\item Se hace una petición al servidor de CAPTCHAS para recuperar los CAPTCHAS del correo.\\
\item El usuario resuelve el CAPTCHA y se recalcula la clave de descifrado.\\
\item Se descifra el mensaje y se le muestra al usuario.\\
\end{itemize}
\end{enumerate}
\begin{figure}[h]
	\includegraphics[width=1\linewidth, height=10cm]{./images/0001.jpg}
	\caption{Diagrama General del sistema}
	\label{fig:4-1-1}
\end{figure}
\section{Tecnologías}
Como ya se ha visto en el esquema anterior se necesita hacer uso de las herramientas adecuadas para poder desarrollar este esquema de cifrado. Las herramientas que se analizaron se describen en las siguientes secciones.\\
\subsection{Cliente de correo electrónico}
Un cliente de correo electrónico es necesario para el desarrollo de este proyecto ya que en él se instalará un complemento que cifre el mensaje, envíe los CAPTCHAS y descifre los mensajes de correo electrónico. Para ello buscamos un cliente de correo electrónico que cuente con el soporte de los protocolos POP3, SMTP y IMAP; sus licencias son de código libre; soporte la instalación de APIs externas; y tenga soporte en los sistemas operativos \textbf{\textit{Windows}}, \textbf{\textit{IOS}} y \textbf{\textit{Linux}}. Por lo tanto se investigaron los siguientes clientes de correo electrónico que se encuentra en el mercado: \\
\begin{longtable}[H]{| p{2,5cm} | p{2cm} |p{2cm}|p{1,5cm}|p{2cm}|p{3cm}|p{2cm}|}%\footnotesize
 \hline
 \textbf{Cliente de correo electrónico}&\textbf{Sistema Operativo}&\textbf{Protocolos soportados}&\textbf{Código Libre}&\textbf{Agregar funcionalidad}&\textbf{Extra}&\textbf{Gratuita o de paga}\\
 \hline
 \textbf{eM client}&Windows 7, 8 \& 10 ; IOS&POP3, SMTP, IMAP, EWS, AirSyn&NO&NO&100\% compatible con gmail y sus APIs&Ambos\\
 \hline
 \textbf{Postbox}&Windows, IOS&POP3, SMTP, IMAP&NO&SI por medio de APIs&Sincronización con Dropbox, OneDrive, Facebook y Twitter&Ambos\\
 \hline
 \textbf{Zimbra}&Windows, IOS \& Linux&POP3, SMTP, IMAP&SI&SI por medio de APIs&Una plataforma de nivel empresarial y capas se soportar sincronización con múltiples servicios&Ambos\\
 \hline
 \textbf{Opera Mail}&Windows, IOS \& Linux&POP3, SMTP, IMAP&SI&NO&La plataforma para desarrollar en Opera se actualiza cada semana&Gratuito\\
 \hline
 \textbf{Thunderbird}&Windows, IOS \& Linux&POP3, SMTP, IMAP&SI&SI por medio de APIs&Cliente de correo versátil y fácilmente escalable y una comunicad de desarrollo bastante amplia&Gratuito\\
 \hline
 \textbf{Nylas N1}&Windows, IOS \& Linux&POP3, SMTP, IMAP&SI&Si directamente compilando& &Gratuito
 
    \label{tabla:Descripcion de clientes}
    \\
  \hline

\end{longtable}

\begin{itemize}
 \item El cliente de correo electrónico \textbf{\textit{eM client}} tiene una sincronización a 100\% con las cuentas de \textbf{\textit{Gmail}} y sus APIs, cuenta con una versión gratuita y una versión de paga; puede hacer migración de mensajes de correo electrónico y contactos de diversos clientes de correo electrónico y tiene una compatibilidad con muchos servidores de correo electrónico.\cite{em}\\Su desventaja es que su código es cerrado y permite agregar funcionalidades.
 \item El cliente de correo electrónico \textbf{\textit{Postbox}} esta soportada en los sistemas operativos \textbf{\textit{Windows 7}} o posteriores y \textbf{\textit{IOS}}, esta aplicación es generada por el servidor de correo electrónico \textbf{\textit{Postbox}} por lo tanto cuenta con una sincronización al 100\% con este servidor, también soporta otros servidores de correo como \textbf{\textit{Gmail}} y \textbf{\textit{Outlook}}; este cliente puede sincronizarse con \textbf{\textit{Dropbox}}, \textbf{\textit{Onedrive}} y redes sociales como \textbf{\textit{Facebook}}, \textbf{\textit{Twitter}}, entre otras. Es posible agregar más funcionalidades con la instalación de APIs.\\Una desventaja de esta aplicación es que su código es cerrado, pero gracias a que esta basado en código de \textbf{\textit{Mozilla}} puedes desarrollar APIs para agregarle tus propias funciones. \cite{box}
 \item El cliente de correo electrónico \textbf{\textit{zimbra}} es la aplicación más completa que se analizó, tiene compatibilidad con el servidor \textbf{\textit{zimbra}} pero es capaz de soportar otros servidores de correo electrónico, se encuentra en los 3 sistemas para PC, \textbf{\textit{Windows}}, \textbf{\textit{IOS}} \& \textbf{\textit{Linux}}, da la facilidad de agregarle funcionalidades por medio de la instalación de APIs y gracias a que su código es abierto se pueden programar funciones propias. Este cliente cuenta con la versión gratuita y la versión de paga. Una gran ventaja que tiene es que oferta certificaciones en el desarrollo APIs para este cliente de correo electrónico.\cite{zim}\\La única desventaja que se encontró en este cliente de correo es que la plataforma es demasiado grande y el tiempo que se necesita invertir al estudio del código es demasiado y el tiempo de desarrollo de este proyecto es muy corto.
 \item \textbf{\textit{Opera mail}} es un cliente de correo electrónico que salió al mercado recientemente y se puede instalar en los sistemas operativos \textbf{\textit{Windows}}, \textbf{\textit{IOS}} \& \textbf{\textit{Linux}}, es capaz de comunicarse con diversos servidores de correo electrónico y su código es de libre acceso.\\Su principal desventaja es que las funcionalidades que se quieran agregar no pueden ser adquiridas por medio de la instalación de complementos o APIs.\cite{opera}
 \item Por último tenemos a \textbf{\textit{Thunderbird}} que es un cliente de correo electrónico desarrollado por \textbf{\textit{Mozilla}}, este cliente es de código abierto y la instalación de APIs para agregar funcionalidad es demasiado sencilla; es un cliente de correo que puede ser instalado en los sistemas operativos \textbf{\textit{Windows}}, \textbf{\textit{IOS}} y \textbf{\textit{Linux}}.\cite{thun}
\end{itemize}
Por lo tanto el cliente de correo electrónico que se usará es \textbf{\textit{Thunderbird}}, al ser un cliente de correo electrónico casi tan completo como \textbf{\textit{zimbra}} pero con la facilidad de desarrollar APIs más rápido.\\
\subsection{Lenguajes de programación.}
Uno de los objetivos que se tienen en este proyecto es generar un complemento para un cliente de correo electrónico por lo tanto al escoger a \textbf{\textit{Thunderbird}} como cliente tenemos que programar con el lenguaje que fue desarrollado para tener la mayor compatibilidad.\\Para el desarrollo de este proyecto se utilizará\cite{thun}:
\begin{itemize}
 \item Lenguaje Python
 \item HTML 5
 \item CSS3
\end{itemize}
A pesar de ser una aplicación de escritorio este cliente de correo electrónico está construido con lenguajes web.
\subsection{Tipos de CAPTCHAS}
En el esquema de cifrado es necesario esconder la palabra que genera la clave en un CAPTCHA para ser enviado a otro usuario y descifre el mensaje, pero existen varios tipos de CAPTCHAS que se pueden utilizar\cite{cit}.\\Los CAPTCHAS se pueden clasificar de la siguiente forma:
\begin{itemize}
 \item CAPTCHAS de texto: Este tipo de CAPTCHAS genera una pregunta sencilla donde la respuesta a la pregunta es la respuesta al CAPTCHA.
 \item CAPTCHAS de imágenes: Este tipo de CAPTCHAS nos muestran en una imagen una cadena de caracteres distorsionados, esta cadena de caracteres es la repuesta del CAPCHA transformada en una imagen.
 \item CAPTCHAS de audio: Este tipo de CAPTCHAS se caracterizan por  ser un audio con ruidos de fondo, donde nos dirán la respuesta oculta.
 \item CAPTCHAS de video: Este tipo de CAPTCHAS nos muestran un video de unos cuantos segundos donde una palabra aparece alrededor de la pantalla, esta palabra es la respuesta al CAPTCHA.
 \item CAPTCHAS de acertijos: Este tipo de CPATCHAS es versatil, ya que se trata de pequeños acertijos que resolver donde la respuesta no es una palabra si no una acción o serie de acciones. Los CAPTCHAS de acertijos pueden ser armar un rompecabezas pequeño, seleccionar la imagen que no corresponde, unir figurar geométricas, etc. 
\end{itemize}
Para poder decidir qué tipo de CAPTCHAS se utilizará se consideró los siguientes puntos:
\begin{itemize}
 \item Facilidad de creación.
 \item Peso en bytes del CAPTCHA.
 \item Forma del despliegue.
 \item Tipo de respuesta.
\end{itemize}
Por lo tanto se necesita un tipo de CAPTCHA con poco peso, cuya respuesta sea una cadena de caracteres y su forma de despliegue sea fácil de implementar. \\
Considerando las necesidades anteriores las opciones son CAPTCHAS de imágenes y CAPTCHAS de texto, pero en este proyecto se utilizarán los CAPTCHAS de imágenes porque en ellos seran almacenadas las palabras claves de cifrado de los diferentes mensajes de correo electrónico y estas son cadenas de caracteres que no se les puede generar una pregunta coherente. \\
\subsection{Bases de datos para almacenar los CAPTCHAS.}
Para escoger un gestor de base de datos que controle la información de los usuarios registrados en la aplicación propuesta, la información de los  mensajes que envían entre usuarios y los CAPTCHAS asociados a los mensajes para ser descifrados se consideraron 3 características principales en un gestor base de datos relacional:\\
\begin{itemize}
 \item Rapidez en transferencias de información.
 \item Número de usuarios conectados que soporta.
 \item Facilidad de comunicación entre los lenguajes Python, HTML con los gestores de base de datos.
\end{itemize}
Los 3 gestores que se analizaron fueron SQLite, MySQL y PostGreSQL.\\
SQLite es un gestor de base de datos sumamente ligero que soporta hasta 2 terabytes de información, esta base de datos es compatible con python y lenguajes de programación web. Este gestor por su misma ligereza es posible implementarla en dispositivos móviles, pero no es recomendable cuando múltiples usuarios están haciendo peticiones al mismo tiempo ya que su rendimiento baja\cite{DB}.\\
MySQL es un gestor de base de datos que se caracteriza por su transferencia al hacer consultas de datos almacenados; es uno de los gestores libres más utilizados en aplicaciones web y cuenta con diferentes APIs para hacer la comunicación con los lenguajes Python, PHP, C++, etc. Y soporta peticiones de múltiples usuarios gracias a la implementación de hilos.\\
PostGreSQL es un gestor de base de datos que puede hacer operaciones complejas como subconsultas, transacciones y rollbacks’s. Soporta las peticiones de múltiples usuarios pero en cuanto a la velocidad de transferencia de información, comparado con MySQL, es muy lenta pero lo compensa manteniendo una velocidad casi invariable a pesar de que la base se mantenga con un número grande de registros.\cite{sql}\\
Se eligió el gestor de base de datos MySQL porque el proyecto necesita mayor rapidez en la transferencia de información más que generar filtros muy complejos para la búsqueda de información.\\
\clearpage
%


\section{Descripción General del Desarrollo del Protocolo}
El protocolo criptográfico para evitar duplicados almacenados en la nube, es un proyecto que involucra la tecnología de software para ofrecer un funcionamiento eficiente y útil para las necesidades de los usuarios que se encuentran inmersos en el cómputo nube. 
Nuestro protocolo, se compone a su vez de diferentes módulos. Cada uno de ellos busca satisfacer: \\ 

\begin{itemize}

\item La seguridad de los archivos de los usuarios en la nube
\item Almacenamiento seguro en la sube
\item Conexión de diversos usuarios 
\item Ahorro en el consumo de espacio ofrecido en la nube 
\item Fácil acceso al almacenamiento de los archivos de los usuarios 

\end{itemize} 

Para lograrlo, el protocolo criptográfico se compone de diversos módulos que a su vez integran diferentes aplicaciones criptográficas. Dichos módulos serán detallados en las próximas secciones. 

\section{Servidor de Llaves}
El módulo Servidor de llaves tiene una función muy importante dentro del funcionamiento del protocolo criptográfico, ya que sin dicho servidor los usuarios no podrían conectarse para poder utilizar el servicio de almacenamiento seguro. 

\subsubsection{Objetivo}
\begin{itemize}
	\item Generar llaves certificadas para cifrar los archivos.
	\item Brindar seguridad ante ataques por fuerza bruta.
\end{itemize}

\subsubsection{Entradas}
\begin{itemize}
	\item El factor de ocultamiento (Proporcionado por el cliente)
	\item La clave privada del servidor \textit{d} , generada mediante el algoritmo RSA. 
\end{itemize}

El desarrollo de éste módulo, fué realizado mediante el lenguaje de programación Python en su versión 2.7. \\
Las librerías utilizadas para llevar acabo la implementación de éste módulo son: 

\begin{lstlisting}[language=Python,frame=single, keywordstyle=\color{blue}]
			import SocketServer
			import threading
			import time
			from rsagen import *
\end{lstlisting}

Para crear el socket utilizado en el servidor, ocupamos el modulo \textbf{socket} de Python, el cuál importamos desdse la librería \textbf{SocketServer}, dicho módulo simplifica la tarea de escribir servidores de red montados en el lenguaje de programación python. \\ La librería \textbf{threading} construye interfaces de subprocesamiento de nivel superior en la parte superior del thread.\\
La librería \textbf{time} disponible en Python proporciona funciones para trabajar con los tiempos y para convertir representaciones. \\
Finalmente,\textbf{ rsagen} es una implementación de Python RSA pura. Es compatible con el cifrado,  el descifrado, la firma, verificación de firmas y la generación de claves de acuerdo con PKCS \# 1 versión 1.5. Se puede utilizar como una biblioteca de Python, así como en la línea de comandos.
\\ \\ 

Una vez que importamos las bibliotecas pertinentes, se generan las llaves del servidor, tanto la pública \textit{(e)}, como la privada \textit{(d)}, para esto usamos el método \textit{gen\_rsa(usuario)}, es decir utilizamos RSA para la generación de llaves y para ello hacemos uso de \textit{rsagen}, donde ya viene la implementación de dicho algoritmo. \\ 

Posteriormente leemos el archivo \textit{n} y la llave privada \textit{d} del servidor.

	\begin{lstlisting}[language=Python,frame=single, keywordstyle=\color{blue}]
	usuario = 'server'
	gen_rsa(usuario)
	n = int(open("key_n_server.PEM", "r").read())
	d = int((open("key_d_server.PEM", "r").read())
	\end{lstlisting}

Para continuar, creamos un TCP Handler el cual utiliza el protocolo TCP de Internet, que proporciona transmisiones continuas de datos entre el cliente y el servidor.

 
\begin{lstlisting}[language=Python,frame=single, keywordstyle=\color{blue}]
#creo mi TCP Handler
class MiTcpHandler(SocketServer.BaseRequestHandler):
\end{lstlisting}


Ya que tenemos la conexión entre las dos partes ahora estamos listos para recibir los datos enviados por el cliente, lo que vamos a recibir se le llama factor de ocultamiento  \textit{x} esto nos ayuda a que el servidor quien va a realizar una firma a ciegas no sepa que es lo que este firmando, y no se entere de la información manejada, ya que tenemos este factor ahora si, el servidor se dispone a realizar la firma a ciegas \textit{y} para que pueda continuar con la creación de la llave para poder cifrar los archivos

\begin{lstlisting}[language=Python,frame=single, keywordstyle=\color{blue}]
       def handle(self):
        data= ""
        while data != "salir":
            #intento recibir informacion
            try:
                data= self.request.recv(65536)
                print data
                x = int(data)
                print "x:", x
                y = pow(x,d,n)
				#Devolvemos el mensaje al cliente
                print "y:", y
                self.request.send(str(y))
                time.sleep(0.1)
			 #espero 0.1 segundos antes de leer neuvamente
            #si hubo un error lo digo y termino el handle
            except:
                print "El cliente D/C o hubo un error"
                data="salir"
\end{lstlisting}

Obteniendo la firma a ciegas \textit{y} se la enviamos a nuestro cliente para que pueda continuar en la creación de la llave \textit{z}, esto se ve explicado mas adelante cuando se explique la interfaz.



%En la función handle mientras \textit{data} sea distinto de \textit{“Salir”} intentará recibir la información, imprimirá en pantalla \textit{“Nuevo Cliente Conectado”} 

			%\begin{figure}[H]
			%\centering
			%\includegraphics[width=14cm, height=10cm]{./images/servidor/04.jpg}
			%\caption{Conexión de Clientes}
			%\label{fig:6-1-4} 
			%\end{figure}



\section{Aplicación Criptográfica (Cifrado/Descifrado)}

%Éste módulo del protocolo es de suma importancia para ofrecer la seguridad de los archivos como se menciona anteriormente, ya que en éste módulo se lleva a cabo para blindar un archivo, es decir cifrarlo. 
%El desarrollo de éste módulo fue realizado en el lenguaje de programación Python versión 2.7.3. Además de que para la implementación del algoritmo de cifrado AES se utilizó principalmente la biblioteca criptográfica Pycripto 2.3. 

\subsection{Cifrado}
Éste algoritmo que forma parte de la aplicación criptográfica, se lleva a cabo del lado del cliente, dicho algoritmo de cifrado se encargará de brindar la seguridad a los archivos que los usuarios deseen almacenar en la nube. \\
El cifrado de archivos se lleva a cabo bajo la utilización del algoritmo de cifrado \textbf{AES} que proveé la librería criptográfica \textbf{PyCripto 2.3} propia del lenguaje \textbf{Python}. Ésta librería poseé la seguridad necesaria para satisfacer a los requerimientos del protocolo criptográfico.  \\ 

\subsubsection{Objetivo}
Proteger la información que se almacenará en la nube.

\subsubsection{Entradas}
	\begin{itemize}
		\item El archivo que se almacenará en la nube.
		\item La clave para poder cifrarlo, que en este caso es la llave \textit{(z)} generada por el cliente en el módulo anterior. 
	\end{itemize}

La implementación del algoritmo, se llevó a cabo de la siguiente manera: 

\begin{lstlisting}[language=Python,frame=single, keywordstyle=\color{blue}]
import base64
import hashlib, Crypto.Cipher.AES, Crypto.Util.Counter
import hmac
import tkFileDialog
\end{lstlisting}

Siendo \textbf{haslib, Crypto.Cipher.AES, Crypto.Util.Counter, hmac} librerías criptográficas, es decir, utilizadas para llevar a cabo operaciones relacionadas con la implementación del algoritmo de cifrado \textit{AES} en sus 3 tipos de tamaños de claves \textit{(128, 192, 256 bits)}.\\ \\ 

La librería \textbf{tkFileDialog} utilizada para cuando el usuario desee elegir desde su PC un archivo para almacenarlo en la nube, se abra un panel de archivos, permitiendo acceder a sus carpetas personales, y de manera gráfica, dicho usuario pueda elegir el archivo con solo darle un clic. \\ \\ 

\textbf{Base 64} Una librería utilizada para convertir las salidas del algoritmo \textit{AES} a caracteres dentro del \textit{código ASCII}, ya que \textit{AES} involucra funciones que obtienen a la salida caracteres en el sistema hexadecimal y son difíciles de procesar en su forma original para su uso posterior. 

\begin{itemize}

	\item Para poder comenzar el proceso de cifrado, es necesario obtener la clave que se utilizará para llevar a cabo el proceso. \\ Dicha clave se obtiene de la siguiente manera: 

	\begin{lstlisting}[language=Python,frame=single, keywordstyle=\color{blue}]
	contentK = open("key_z", "rb"), read()
	\end{lstlisting}

Abrimos el archivo \textbf{key\_z} y almacenamos su contenido en la variable \textbf{contentK}. Éste archivo contiene la clave que se necesita para poder cifrar el archivo que el usuario desea, dicha \textit{clave (z)} fue generada y escrita en este archivo en el módulo anterior. 

	\item Se crea un objeto de tipo \textit{AES} que almacenamos en la variable \textbf{cipher}, el cual contiene como parámetros la clave que obtuvo del archivo  \textbf{key\_z}, el modo de operación que se utilizará \textbf{(CTR)}, etc.
			
\begin{lstlisting}[language=Python,frame=single, keywordstyle=\color{blue},breaklines=true]
cipher = Crypto.Cipher.AES.new(contentK, Crypto.Cipher.AES.MODE_CTR, counter=ctr)
\end{lstlisting}

	\item Para cifrar el archivo, mandamos llamar al método \textbf{\textit{encrypt(m)}} \textit{(m almacena el contenido del archivo que se desea cifrar)} y almacenamos el resultado de dicho método en la variable \textbf{ctext.} y esta variable contiene el archivo cifrado por completo
			
	\begin{lstlisting}[language=Python,frame=single, keywordstyle=\color{blue}]
		ctext = cipher.encrypt(m)
	\end{lstlisting}



\end{itemize}


\subsection{Descifrado}
Éste algoritmo que forma parte de la aplicación criptográfica, al igual que el cifrado, se lleva a cabo del lado del cliente, éste algoritmo será el encargado de que los usuarios puedan recuperar sus archivos originales, es decir, tomar de la nube aquel archivo que se encuentre cifrado y posteriormente descifrarlo para poder acceder a este archivo en su forma original. \\
El descifrado de archivos se lleva a cabo bajo la utilización del algoritmo de descifrado \textbf{AES} que, al igual que el cifrado lo proveé la librería criptográfica \textbf{PyCripto 2.3} propia del lenguaje \textbf{Python}.   \\ 

\subsubsection{Objetivo}
Descifrar los archivos almacenados de forma íntegra.

\subsubsection{Entradas}
	\begin{itemize}
		\item El archivo que se almacenó en la nube.
		\item La clave para poder descifrarlo, que en este caso es la llave \textit{(z)} generada por el cliente.
	\end{itemize}

La implementación del algoritmo, se llevó a cabo de la siguiente manera: 

\begin{itemize}
	\item Las librerías utilizadas para llevar a cabo el descifrado de archivos son las siguientes: 
			
\begin{lstlisting}[language=Python,frame=single, keywordstyle=\color{blue}]
		import Crypto.Cipher.AES
		import Crypto.Util.Counter
\end{lstlisting}

Siendo \textbf{Crypto.Cipher.AES, Crypto.Util.Counter}, librerías criptográficas, es decir, utilizadas para llevar a cabo operaciones relacionadas con la implementación del algoritmo de descifrado \textit{AES} en sus 3 tipos de tamaños de claves \textit{(128, 192, 256 bits)}. 

	\item Al igual que en el proceso de cifrado, para comenzar dicho proceso es necesario obtener la clave que se utilizará para el descifrado. Dicha clave se obtiene de la siguiente manera:  

\begin{lstlisting}[language=Python,frame=single, keywordstyle=\color{blue}]
    file_d = open("llaves_clientes/key_d_"+nom_user+".PEM", "rb").read()

\end{lstlisting}

Abrimos el archivo \textbf{key\_z} y almacenamos su contenido en la variable \textbf{contentK}. 

		\item El siguiente paso, es abrir el archivo temporal que se generó al momento de cifrar el archivo del usuario. 
			
\begin{lstlisting}[language=Python,frame=single, keywordstyle=\color{blue}]
	textcifrado = open("./tempc", "rb"), read()
\end{lstlisting}

Una vez dentro del archivo, almacenamos el cifrado en la variable \textbf{textcifrado} para utilizarlo posteriormente. 

		\item Creamos un objeto \textit{AES} que almacenamos en la variable \textbf{ciphe}r, el cual contiene como parámetros la llave que obtuvo del archivo \textbf{key\_z }, el modo de operación que se utilizará para el descifrado\textbf{(CTR)}, etc.
			
\begin{lstlisting}[language=Python,frame=single, keywordstyle=\color{blue},breaklines=true]
cipher = Crypto.Cipher.AES.new(contentK, Crypto.Cipher.AES.MODE_CTR, counter = ctr)
\end{lstlisting}


		\item Desciframos el archivo, mandamos llamar al método \textbf{\textit{decrypt(textcifrado)}}  y almacenamos el resultado de dicho método en la variable \textbf{plaintext}
		
\begin{lstlisting}[language=Python,frame=single, keywordstyle=\color{blue}]
	plaintext = cipher.decrypt(textcifrado)
\end{lstlisting}

		\item Para finalizar, mandamos escribir a un archivo \textbf{fname} \textit{(Es el nombre del archivo original del usuario)} el archivo tal y como estaba antes de cifrarlo. 
			
\begin{lstlisting}[language=Python,frame=single, keywordstyle=\color{blue}]
		outf = open(fname, "wb")
		outf.write(plaintext)
		outf.close()
\end{lstlisting}

\end{itemize}

\section{Interfaz Web}

En ésta sección, se encuentra el detalle de la implementación de los módulos anteriores dentro de una interfaz web.\\
Dicha interfaz se desarrolló bajo el sistema operativo Linux. Esto debido a que presenta varias ventajas, es un sistema operativo de software libre, por ende no es necesario comprar una licencia para instalarlo y utilizarlo en un equipo informático. Es un sistema multitarea, multiusuario, compatible con \textit{UNIX}, y proporciona una interfaz de comandos y una interfaz gráfica.\\
También elegimos como manejador de base de datos de la interfaz \textit{SQLite}.\\
\textit{SQLite} es una herramienta de software libre, que permite almacenar información en dispositivos empotrados de una forma sencilla, eficaz, potente, rápida y en equipos con pocas capacidades de hardware. Además agrega extensiones que facilitan su uso en cualquier ambiente de desarrollo. Esto permite que \textit{SQLite} soporte desde las consultas más básicas hasta las más complejas del lenguaje SQL, y lo más importante es que se puede usar tanto en dispositivos móviles como en sistemas de escritorio, sin necesidad de realizar procesos complejos de importación y exportación de datos, ya que existe compatibilidad entre las diversas plataformas disponibles, haciendo que la portabilidad entre dispositivos y plataformas sea transparente.\\
Sin embargo la mayor parte de la codificación fue desarrollada e implementada en \textit{Django}, éste es un framework para aplicaciones web gratuito y open source, escrito en \textit{Python}.\\
\textit{Python} es un lenguaje de programación poderoso y fácil de aprender. Cuenta con estructuras de datos eficientes y de alto nivel y un enfoque simple pero efectivo a la programación orientada a objetos. Un lenguaje ideal para desarrollo rápido de aplicaciones en diversas áreas y sobre la mayoría de las plataformas.\\
Lo cuál hizo posible la compatibilidad entre los módulos desarrollados y una interfaz web que fuera intuitiva,segura y fácil de manejar para los usuarios.\\ 

%Todas estas herramientas en conjunto nos han permitido tener una aplicación web funcional, segura y eficaz. 

%En ésta sección, se encuentra el detalle de la implementación de los módulos anteriores dentro de una interfaz web. Dicha interfaz se desarrolló bajo el lenguaje de programación Python en su versión 2.7. Para que la interfaz web pudiera llevarse a cabo de manera conjunta con éste lenguaje de programación,  fué necesario la implementción de un Framework basado en Python, dicho Framework se llama \textit{Django} en su versión 1.8 con compatibilidad con el Sistema Operativo Linux. \textit{Django} proveé de muchas herramientas web basadas en Python, lo cuál hizo posible la compatibilidad entre los módulos anteriormente desarrollados y una interfaz web que fuera intuitiva y fácil de manejar para los usuarios.  \\ 

\section{Mapa de Navegación}
En la figura  ~\ref{fig:6-1-1} se muestra el mapa de navegación de la aplicación web.

			\begin{figure}[H]
			\centering
			\includegraphics[width=14cm, height=7.5cm]{./images/MapaDeNavegacion.jpg}
			\caption{Mapa de navegación}
			\label{fig:6-1-1} 
			\end{figure}

Dentro de la interfaz web, existen diferentes módulos que la conforman. \\ Dichos módulos son: 

\subsection{Registrar un Nuevo Usuario} 
Éste módulo que compone a la interfaz, se encargará de poder llevar a cabo el registro en la plataforma de nuevos usuarios de nuestro protocolo criptográfico. En dicho módulo participará el usuario activamente con la interfaz, la cuál le solicitará datos personales como \textbf{Nombre, Apellidos, Correo electrónico, Contraseña}. Con ésta información, la plataforma generará un par de claves de usuario (Con el algoritmo de generación de claves RSA) para que estas sean utilizadas en el proceso de carga y descarga de archivos. \\ 

A continuación, se muestra a detalle como es que se lleva a cabo el registro de un usuario dentro de la interfaz web: \\ 

En este fragmento de código podemos observar la clase \textbf{\textit{RegisterView}} que se encarga de crear un usuario, lo que hace es obtener los datos del formulario que ha llenado el usuario en la página y crea un objeto \textbf{\textit{user}} para poder identificarlo como usuario, una vez que su registro haya sido exitoso se crean sus llaves certificadas.

\begin{lstlisting}[language=Python,frame=single, keywordstyle=\color{blue},breaklines=true]
	class RegisterView(FormView):
		template_name = 'nube/register.html'
		form_class = RegistrationForm

		def form_valid(self, form):

			user= User.objects.create_user(
			username = form.cleaned_data["username"],
			password = form.cleaned_data['password1'],
			email = form.cleaned_data['email']
			)

			nom_user = user.username
			gen_rsa(nom_user)
			return super(RegisterView, self).form_valid(form)
		def get_success_url(self):
			return reverse('register-success')

class RegisterSuccessView(TemplateView):
	template_name = 'nube/success.html'

class IndexView(TemplateView):
	template_name = 'nube/index.html'
\end{lstlisting}

Para que un usuario pueda iniciar sesión es necesario que la información que ingresó en la plataforma sea validada por el sistema ya que debe estar registrada en él.
Para esto tenemos la clase llamada \textbf{\textit{LoginView}} donde se obtienen los datos del usuario y se valida con los datos que ya existen en la base de datos. Si los datos del usuario se encuentran almacenados, la plataforma redireccionará a la página del perfil del usuario. En caso contrario, la plataforna regresará al index.

\begin{lstlisting}[language=Python,frame=single, keywordstyle=\color{blue},breaklines=true]
from nube.models import Document
from os import walk
from glob import glob
from nube.Cliente import *
from criptoleslie import config
from nube.forms import RegistrationForm, LoginForm, UploadForm

nom_user = ""

class LoginView(FormView):
	template_name = 'nube/login.html'
	form_class = LoginForm

	@method_decorator(csrf_protect)
	def dispatch(self, *args, **kwargs):
		return super(LoginView, self).dispatch(*args, **kwargs)

	def form_valid(self, form):
		login(self.request, form.get_user())
		return super(LoginView, self).form_valid(form)

	def get_success_url(self):
		try:
			return config.LOGIN.REDIRECT_URL
		except:
			return "/profile/"

class LogoutView(View):
\end{lstlisting}

Para poder listar todos los archivos que tiene almacenados el usuario que ha iniciado sesión, realizamos un ciclo for para recorrer el arreglo que contiene los nombres de los archivos que se encuentran en el directorio.

\begin{lstlisting}[language=Python,frame=single, keywordstyle=\color{blue},breaklines=true]
		
			{{ lstFils.i }}
		
\end{lstlisting}

Dicho arreglo que se recorrió, fué creado en las vistas, ahí tenemos una función de donde obtenemos la ruta donde se encuentran almacenados los archivos y con ayuda de la función \textbf{\textit{os.walk}} podemos obtener la lista de los directorios y subcarpetas que existen en éste.

\begin{lstlisting}[language=Python,frame=single, keywordstyle=\color{blue},breaklines=true, showstringspaces=false]
def lista(request):
	path = '/home/jhonatan/pycharmProjects/CriptoLeslie/criptoleslie/Cifrados/'
	lstFiles = []

	lstDir = os.walk(path) 

	for root, dirs, files in lstDir:
		for fichero in files:
			(nombreFichero, extension) = os.path.splitext(fichero)
			lstFiles.append(nombreFichero + extension)

	print(lstFiles)
	print('LISTADO FINALIZADO')
	return render(request, 'nube/profiile.html', {'lstFiles': lstFiles})
\end{lstlisting}

Para poder seleccionar el archivo que se va a subir, en el html mandamos a llamar el formulario creado para poder \textit{cargar un archivo}.

\begin{lstlisting}[language=HTML,frame=single, keywordstyle=\color{blue},breaklines=true]
<div class="row" id="content-container">
	<div class="page-header">
		<h2></h2>
	</div>
	<div class="jumbotron content">
		<form acion="" method="post" enctype="multipart/form-data">
			{{ form.as_p }}
			<input type="submit" value="Subir">
		</form>
	</div>
</div>
\end{lstlisting}


Este método fue implementado para crear el documento y guardarlo en la base de datos. Se crea un objeto de tipo documento donde obtenemos el nombre del archivo que fue previamente llenado desde el formulario por el cliente y se almacena en la base de datos con la línea \textbf{\textit{newdoc.save(form)}}.
En la siguiente línea \textbf{\textit{(subir\_arch…)}}, se manda a llamar la función \textbf{\textit{subir archivo}} que es la que se encarga de cifrar el archivo.

\begin{lstlisting}[language=Python,frame=single, keywordstyle=\color{blue},breaklines=true]
def upload_file(request):
	if request.method == 'POST':
	   form = UploadForm(request.POST, request.FILES)
	   if form.is_valid():
		newdoc = Document(filename=request.POST['filename'], docfile=request.FILES['docfile'])
		newdoc.save(form)
		subir_arch(neewdoc.filename, nom_user)
		return redirect("profile")
	else:
		form= UploadForm()
	return render(request, 'nube/upload.html', {'form': form})
\end{lstlisting}



A continuación, se explica con detalle la parte del cifrado de archivos. \\
Para empezar, importamos algunas bibliotecas criptográficas, como son:

\begin{lstlisting}[language=Python,frame=single, keywordstyle=\color{blue},breaklines=true]
	import Crypto.Cipher.AES, Crypto.Util.Counter
	import hmac
	from rsagen import *
\end{lstlisting}


\begin{itemize}
	\item  La biblioteca \textbf{Crypto.Cipher.AES} tiene implementado el algoritmo AES en todas sus versiones y lo utilizaremos para cifrar.
	\item  La biblioteca \textbf{hmac} contiene la implementación de las funciones Hash que parael desarrollo de este protocolo criptográfico se utilizó \textbf{SHA256}.
	\item La biblioteca \textbf{rsagen}, en ésta implementamos el algoritmo de RSA para la generación de llaves.

\end{itemize}



En ésta parte de la codificación, se tiene la implementación de la función \textbf{\textit{subir archivos}}, donde lo primero que hacemos es abrir las llaves del servidor y crear una conexión con ayuda de sockets entre nuestro cliente y el servidor.
		
\begin{lstlisting}[language=Python,frame=single, keywordstyle=\color{blue},breaklines=true]
def subir_arch(filename,nom_user):
    n = int(open("llaves_servidor/key_n_server.PEM", "r").read())
    e = int(open("llaves_servidor/key_e_server.PEM", "r").read())
    host, port = "localhost" , 9999
    sock= socket.socket()
    sock.connect((host,port))
\end{lstlisting}


Una vez creada la conexión  entre cliente - servidor, se obtiene el archivo que se va a subir, esto gracias a que cuando se manda a llamar la función, ésta pasa por parámetros el nombre del archivo y del usuario. Con esto se logra abrir el archivo y se le aplica una función hash de \textbf{\textit{SHA256 (h(m))}} y se opera con la llave pública del servidor \textit{(e)}, a esto se le llama \textit{factor de ocultamiento (x)}, con este factor se puede llevar a cabo una firma a ciegas por parte de nuestro servidor, éste factor se le envía al servidor por medio de  sockets.

\begin{lstlisting}[language=Python,frame=single, keywordstyle=\color{blue},breaklines=true, showstringspaces=false]
 while mensaje != "salir":
	 r = randint(0,n)
	m = str(open(filename, "rb").read())
	 h = int(hashlib.sha256(m).hexdigest(), 16)
           print "h(m): ", h
           x1 = pow(r,e,n)
           x = (h * x1) % n
           print "x: ", x
           f_x = open("x", "w")
          f_x.write(str(x))
          f_x.close()
          mensaje = open("x", "r").read() 

	  try:
           	 sock.send(mensaje)

	except:
           	 print "no se mando el mensaje"
            	mensaje="salir"
\end{lstlisting}

Una vez realizada la firma a ciegas \textit{(y)} la envía al cliente por medio de sockets, el cuál procede a calcular la llave \textit{z} (que sirve para cifrar los archivos). Y se crea un archivo con dicha llave, ya que esta llave es demasiado grande para que funcione en \textit{AES}, es por ello que se debe de sacar un hash con \textit{SHA256} para que quede del tamaño permitido por \textit{AES (128 bits)} y de igual manera creamos un archivo con la llave.

\begin{lstlisting}[language=Python,frame=single, keywordstyle=\color{blue},breaklines=true, showstringspaces=false]
print "z: ", z_long
        f_z = open("llaves_clientes/key_z_"+filename2+"_"+nom_user+".PEM", "w")
        f_z.write(str(z_long))
        f_z.close()
        check = pow(z_long,e,n)
        print "check: ", check
        print "h: ", h % n

        kz = str(open("llaves_clientes/key_z_"+filename2+"_"+nom_user+".PEM", "rb").read())
        z = hashlib.sha256(kz).hexdigest()[:16]
        wz =  open("llaves_clientes/key_z_"+filename2+"_"+nom_user+".PEM", "w")
        wz.write(str(z))
        wz.close()
\end{lstlisting}

Para cifrar el archivo se guarda el contenido de la llave \textit{z} y es utilizada para generar un vector de inicialización usando una función \textit{SHA256}, dicho vector se almacena en un archivo para que pueda ser utilizado posteriormente a la hora de descifrar el archivo, posteriormente se cifra el archivo con ayuda de las funciones definidas en las bibliotecas criptográficas que se mencionan anteriormente, y se le indica el tamaño de la llave de \textit{128 bits}, con un modo de operación \textit{CTR} y con la función \textbf{\textit{encrypt()}} se cifra el archivo, éste archivo se almacena en una carpeta y se le pone el nombre original pero agregandole una  extensión \textit{.aes}

\begin{lstlisting}[language=Python,frame=single, keywordstyle=\color{blue},breaklines=true, showstringspaces=false]
        contentK = open("llaves_clientes/key_z_"+filename2+"_"+nom_user+".PEM", "rb").read()
        iv = hmac.new(contentK, m, hashlib.sha256).hexdigest()[:32]  
        escribeIV = open("llaves_clientes/vector_"+filename2+"_"+nom_user+".txt","wb")
        escribeIV.write(iv)
        escribeIV.close()
        ctr = Crypto.Util.Counter.new(128, initial_value=long(iv, 16))
        cipher = Crypto.Cipher.AES.new(contentK, Crypto.Cipher.AES.MODE_CTR, counter=ctr)
        ctext = cipher.encrypt(m)
        fc1 = str(ctext)
        print ""
        print "Mensaje Cifrado... C1 "
\end{lstlisting}



%\chapter{Conclusiones y Trabajo a Futuro}
\section{Conclusiones}



%\appendix
\chapter{Código fuente del prototipo 2}
\label{Anexos A}
A continuación se muestra el código fuente desarrollado en el prototipo 2.
\begin{itemize}
\item Archivo de cifrado (cifrado.py).
\end{itemize}

\begin{lstlisting}[frame=single]
 #! /usr/bin/env python
from Crypto.Hash import SHA256
from Crypto.Cipher import AES
from captcha.image import ImageCaptcha
import os
import random
hash = SHA256.new()
semilla=""
r=0
image = ImageCaptcha(fonts=['./fon/A.ttf', './fon/B.ttf'])
for i in range(5):
	r=random.randrange(100)
	semilla=semilla+chr(r)
	print str(i)+" "+str(r)+" "+chr(r)+" "+semilla
	
print semilla+"\n"

data = image.generate(semilla)
image.write(semilla, '/tmp/out.png')
image.write(semilla, 'out.png')

os.remove("/tmp/out.png")

hash.update(semilla)
otra=hash.digest()
llave = ""
print otra

for i  in range(16):
	llave=llave+otra[i]
	print str(i)+" "+otra[i]+" "+llave
\end{lstlisting}
\begin{lstlisting}[frame=single]
print "\n"
print llave
archy=open('llave.txt','w')
archy.write(semilla)
archy.close()
arc=open('cifrado.txt','w')
archi=open('1443750804.V805Idc01e2M920300.jonnytest:2,S','r')
obj = AES.new(llave, AES.MODE_ECB)
lineas=' '
c=0
while lineas!="": 
	c=c+1
	lineas=archi.read(16)
	
	if (((len(lineas))<16)and((len(lineas))>0)):
		c=16-(len(lineas))
		aux=lineas
		for i in range(c):
			aux=aux+" "
	else:
		aux=lineas
	
	ciphertext = obj.encrypt(aux)
	arc.write(ciphertext)
	print str(c) +"   " + lineas + "   "+str(len(lineas))+"   "
	+str(len(aux))+"   "+ciphertext
	

archi.close()
arc.close()



\end{lstlisting}
\begin{itemize}
\item Ventana de despliegue de CAPTCHAS (ventana.py)
\end{itemize}

\begin{lstlisting}[frame=single]
 #!/usr/bin/python
import Tkinter
import Image, ImageTk
imagenAnchuraMaxima=300
imagenAlturaMaxima=200
from Crypto.Hash import SHA256
from Crypto.Cipher import AES

import random
hash = SHA256.new()
# -*- coding: utf-8 -*-
\end{lstlisting}
\begin{lstlisting}[frame=single]
def funcion():

	a=e.get()
	print(a)
	hash.update(a)
	otra=hash.digest()
	llave = ""
	print otra
	for i  in range(16):	
		llave=llave+otra[i]
		print str(i)+" "+otra[i]+" "+llave
	
	print "\n"
	print llave
	archi=open('cifrado.txt','r')
	arc=open('descifrado.txt','w')
	obj = AES.new(llave, AES.MODE_ECB)
	lineas=' '
	c=0
	while lineas!="": 
		c=c+1
		lineas=archi.read(16)	
		if (((len(lineas))<16)and((len(lineas))>0)):
			c=16-(len(lineas))
			aux=lineas
			for i in range(c):
				aux=aux+" "
		else:
			aux=lineas
	
		ciphertext = obj.decrypt(aux)
		arc.write(ciphertext)
		print str(c) +"   " + lineas + "   "
		+str(len(lineas))+"   "+str(len(aux))+"   "
		+ciphertext
	archi.close()
	arc.close()
	root.quit()

# abrimos una imagen
img = Image.open('out.png')

img.thumbnail((imagenAnchuraMaxima,imagenAlturaMaxima)
, Image.ANTIALIAS)

root = Tkinter.Tk()
\end{lstlisting}
\begin{lstlisting}[frame=single]
root.title("Mostrar imagen")
# Convertimos la imagen a un objeto PhotoImage de Tkinter
tkimage = ImageTk.PhotoImage(img)

# Ponemos la imagen en un Lable dentro de la ventana
label=Tkinter.Label(root, image=tkimage, width=imagenAnchuraMaxima
, height=imagenAlturaMaxima).pack()

valor = ""
e = Tkinter.Entry(root)
e.pack()

buttonStart2=Tkinter.Button(root, text="Cerrar",
                            command=funcion).pack()
# Mostramos la ventana

root.mainloop()


\end{lstlisting}

\chapter{Código fuente del prototipo 8}
\label{Anexos B}
A continuación se muestra el código fuente desarrollado en el prototipo 8.

\begin{itemize}
\item Biblioteca de cifrado (Ek.py).
\end{itemize}

\begin{lstlisting}[frame=single]
from Crypto.Hash import SHA256
import os
import random
import base64
import json
from random import randrange
from map import mapeoBtoI
from map import mapeoItoB
from image import ImageCaptcha
from Crypto.Cipher import AES

def isprime(n): 

 n = abs(int(n)) 
 # 0 y 1 no son primos 
 if n < 2: 
  return False 
 # 2 es el unico primo par 
 if n == 2:  
  return True  
 # El resto de pares no son primos 
 if not n & 1:  
  return False 
 # El rango comienza en 3 y solo necesita subir 
 # hasta la raiz cuadrada de n  
 # para todos los impares 
 for x in range(3, int(n**0.5)+1, 2): 
  if n % x == 0: 
   return False 
 return True 
\end{lstlisting}
\begin{lstlisting}[frame=single]
def primoSig(num):
 buscar=True
 while buscar:
  if isprime(num):
   buscar=False
  else:
   num+=1
 return num
def crearSemilla(tam):
 r=0
 semilla=""
 for i in range(tam):
  r=random.randrange(64)
  semilla=semilla+str(mapeoItoB(r))
 return semilla

def crearLlave(semilla1):
 aux=""
 llave=""
 hash = SHA256.new()
 hash.update(semilla1)
 aux=hash.digest()
 llave = ""
 for i  in range(16):
  llave=llave+aux[i]
 return llave

def crearCAPTCHA(op,semilla2,asunto):
 imagen=""
 aux=""
 ax=[]
 xa=""
 s=[]
 c=0
 asunto=asunto.replace(" ","_")
 os.mkdir('./'+asunto+"",0755)
 image=ImageCaptcha(fonts=['./SSE/fon/A1.ttf','./SSE/fon/A1.ttf'])
 if (op==0):
  aux='./'+asunto+'/CAPTCHA00.png'
  image.write(semilla2, aux)
  return './'+asunto
 else:
  for x in semilla2:
   #print(x)
   aux='./'+asunto+'/CAPTCHA'+str(c)+'.png'
   xa='CAPTCHA'+str(c)+'.png'
\end{lstlisting}
\begin{lstlisting}[frame=single]
   ax.append(xa)
   s.append(ax)
   ax=[]
   image.write(x, aux)
   c=c+1
  return (s,'./'+asunto)

def encodeSS(strin):
 bina=""
 aux=""
 for i in range(len(strin)):
  aux=bin(mapeoBtoI(strin[i])).replace("0b","")
  if (len(aux)==6):
   bina=bina+aux
  else:
   while ((len(aux))<6):
    aux="0"+aux
   bina=bina+aux
 return int(str(bina),2)

def decodeSS(strr,w0):
 c=0
 s=""
 capt=""
 letras=[]
 z=bin(strr).replace("0b","")
 while (len(z)<(6*w0)):
  z="0"+z
 for i in z:
  if (c==5):
   c=0
   s=s+i
   letras.append(s)
   s=""
  else:
   c=c+1
   s=s+i
 for j in letras:
  capt=capt+mapeoItoB(int(str(j),2))
 return capt

def eucExt(a,b):
 r = [a,b]
 s = [1,0] 
 i = 1
 q = [[]]
\end{lstlisting}
\begin{lstlisting}[frame=single]
 while (r[i] != 0): 
  q = q + [r[i-1] // r[i]]
  r = r + [r[i-1] % r[i]]
  s = s + [s[i-1] - q[i]*s[i]]
  i = i+1
 return s[i-1]%b

def GenerarPares(p=7,w=5,t=2,k=0):
 pares =[]
 a = [k]
 for aux in range(0,w):
  print(aux)
  pares.append([randrange(p),0])
 print("X->")
 print(pares)
 for aux in range(1,t):
  print(aux)
  a.append(randrange(p))
 print("A->")
 print(a)
# for aux in range(0,w):
#  suma = k+(a[1]*pares[aux][0])
#  pares[aux][1] = suma%p
 for aux in pares:
  print("suma")
  suma = k
  print(suma)
  for aux2 in range(1,t):
   print("sin ecuacion")
   print(suma)
   suma = (suma+(a[aux2]*(aux[0]**aux2)))%p
   print("con ecuacion")
   print(suma)
  aux[1] =suma
 return pares

def secreto(pares,p):
 suma = 0
# print("pares")
# print(pares)
 for aux in pares:
#  print("par")
#  print(aux)
  ind = pares.index(aux)
#  print("index")
#  print(ind)
\end{lstlisting}
\begin{lstlisting}[frame=single]
  lis = pares[:ind] + pares[(ind+1):]
#  print("otros pares")
#  print(lis)
  num=1
  den=1
  for aux2 in lis:
#   print("numerodor")
   num = (num*(aux2[0])*-1)%p
#   print(num)
#   print("denominador")
   den = (den*((aux[0]-aux2[0])%p))%p
#   print(den)
#  print("Euclides")
  den = eucExt(den,p)
#  print(den)
  suma += (den*aux[1]*num)%p
#  print("suma")
#  print(suma)
 return suma%p

def cifrar(body,asunto1,op1=1,ta=5,w1=5,t1=2):
 ruta=""
 salida=""
 if t1>w1:
  salida=""
  ruta=None
  print("w1 < t1")
  return (salida,ruta)
 semilla3=crearSemilla(ta)
 num=0
 cap=[]
 zp=primoSig(2**(6*ta))
 disc={}
 #print(semilla3)
 if (op1==0):
  ruta=crearCAPTCHA(0,semilla3,asunto1)
 else:
  ruta=[]
  num=encodeSS(semilla3)
  pares=GenerarPares(zp,w1,t1,num)
  #print(pares)
  for x in pares:
   cap.append(decodeSS(x[1],w1))
  ruta=crearCAPTCHA(op1,cap,asunto1)
  num=0
  print(cap)
\end{lstlisting}
\begin{lstlisting}[frame=single]
  for x in pares:
   ruta[0][num].insert(0,x[0])
   num=num+1
  for i in ruta[0]:
   disc[i[1]]=i[0]
  print(ruta)
  lista=open(ruta[1]+"/lista.json","w") 
  lista.write(json.dumps(disc))
  lista.close
 k=crearLlave(semilla3)
 obj = AES.new(k, AES.MODE_ECB)
 salida=""
 ax=0
 c=0
 strr=""
 #print len(body)
 while (ax < len(body)):
  while (c<16):
   if (ax>=len(body)):
    strr=strr+" "
   else:
    strr=strr+body[ax]
   c=c+1
   ax=ax+1
   #print str(c) +" " + str(ax) 
  c=0
  #print strr
  ciphertext = obj.encrypt(strr)
  salida=salida+ciphertext
  strr=""
 salida = base64.b64encode(salida)
 return (salida,ruta)

def descifrar(body1,capt1,op2):
 aux=[]
 ax=0
 pares=[]
 zp=0
 
 if (op2==0):
  k=crearLlave(capt1)
 else:
  w=len(capt1[0][1])
  zp=primoSig(2**(6*(len(capt1[0][1]))))
  for x in capt1:
   aux=x
\end{lstlisting}
\begin{lstlisting}[frame=single]
   aux[1]=encodeSS(x[1])
   pares.append(aux) 
  #print(pares)
  ax=secreto(pares,zp)
  #print(ax)
  semilla4=decodeSS(ax,w)
  #print(semilla4)
  k=crearLlave(semilla4)
  #print(k)
 obj = AES.new(k, AES.MODE_ECB)
 salida=""
 ax=0
 c=0
 strr=""
 #print len(body1)
 body1 = base64.b64decode(body1)
 while (ax < len(body1)):
  while (c<16):
   if (ax>=len(body1)):
    strr=strr+" "
   else:
    strr=strr+body1[ax]
   c=c+1
   ax=ax+1
   #print str(c) +" " + str(ax) 
  c=0
  #print strr
  ciphertext = obj.decrypt(strr)
  salida=salida+ciphertext
  strr=""
 return salida
\end{lstlisting}

\begin{itemize}
\item Generador de imágenes CAPTCHAS (imagen.py).
\end{itemize}

\begin{lstlisting}[frame=single]
# coding: utf-8

import os
import random
from PIL import Image
from PIL import ImageFilter
from PIL.ImageDraw import Draw
from PIL.ImageFont import truetype
try:
 from cStringIO import StringIO as BytesIO
except ImportError:
 from io import BytesIO
\end{lstlisting}
\begin{lstlisting}[frame=single]
try:
 from wheezy.captcha import image as wheezy_captcha
except ImportError:
 wheezy_captcha = None
DATA_DIR = os.path.join(os.path.abspath(os.path.dirname(__file__))
                        , 'data')
DEFAULT_FONTS = [os.path.join(DATA_DIR, 'DroidSansMono.ttf')]

if wheezy_captcha:
 __all__ = ['ImageCaptcha', 'WheezyCaptcha']
else:
 __all__ = ['ImageCaptcha']

class _Captcha(object):
 def generate(self, chars, format='png'):
  im = self.generate_image(chars)
  out = BytesIO()
  im.save(out, format=format)
  out.seek(0)
  return out

 def write(self, chars, output, format='png'):
  im = self.generate_image(chars)
  return im.save(output, format=format)

class WheezyCaptcha(_Captcha):
 def __init__(self, width=200, height=75, fonts=None):
  self._width = width
  self._height = height
  self._fonts = fonts or DEFAULT_FONTS

 def generate_image(self, chars):
  text_drawings = [wheezy_captcha.warp(),wheezy_captcha.rotate(),
                   wheezy_captcha.offset(),]
  fn = wheezy_captcha.captcha(
   drawings=[
    wheezy_captcha.background(),
    wheezy_captcha.text(fonts=self._fonts, drawings=text_drawings),
    wheezy_captcha.curve(),
    wheezy_captcha.noise(),
    wheezy_captcha.smooth(),],
   width=self._width,
   height=self._height,
  )
  
  return fn(chars)
\end{lstlisting}
\begin{lstlisting}[frame=single]
class ImageCaptcha(_Captcha):

 def __init__(self,width=160,height=60,fonts=None,font_sizes=None):
 
  self._width = width
  self._height = height
  self._fonts = fonts or DEFAULT_FONTS
  self._font_sizes = font_sizes or (46, 58, 68)
  self._truefonts = []

 @property
 def truefonts(self):
 
  if self._truefonts:
   return self._truefonts
  self._truefonts = tuple([
   truetype(n, s)
   for n in self._fonts
   for s in self._font_sizes
  ])
  return self._truefonts

 @staticmethod
 def create_noise_curve(image, color):
 
  w, h = image.size
  x1 = random.randint(0, int(w / 5))
  x2 = random.randint(w - int(w / 5), w)
  y1 = random.randint(h / 5, h - int(h / 5))
  y2 = random.randint(y1, h - int(h / 5))
  points = [(x1, y1), (x2, y2)]
  end = random.randint(160, 200)
  start = random.randint(0, 20)
  Draw(image).arc(points, start, end, fill=color)
  return image

 @staticmethod
 def create_noise_dots(image, color, width=3, number=30):
  draw = Draw(image)
  w, h = image.size
  while number:
   x1 = random.randint(0, w)
   y1 = random.randint(0, h)
   draw.line(((x1, y1), (x1 - 1, y1 - 1)), fill=color, width=width)
   number -= 1
  return image
\end{lstlisting}
\begin{lstlisting}[frame=single]
 def create_captcha_image(self, chars, color, background):

  image = Image.new('RGB', (self._width, self._height), background)
  draw = Draw(image)

  def _draw_character(c):
   font = random.choice(self.truefonts)
   w, h = draw.textsize(c, font=font)

   #dx = random.randint(4, 6)
   #dy = random.randint(4, 8)
   im = Image.new('RGBA', (w+30 , h+30 ))
   Draw(im).text((0, 0), c, font=font, fill=color)

   # rotate
   #im = im.crop(im.getbbox())
   #im = im.rotate(random.uniform(-30, 30),Image.BILINEAR,expand=1)
   # warp
   #dx = w * random.uniform(0.1, 0.3)
   #dy = h * random.uniform(0.2, 0.3)
   #x1 = int(random.uniform(-dx, dx))
   #y1 = int(random.uniform(-dy, dy))
   #x2 = int(random.uniform(-dx, dx))
   #y2 = int(random.uniform(-dy, dy))
   #w2 = w + abs(x1) + abs(x2)
   #h2 = h + abs(y1) + abs(y2)
   #data = (
   # x1, y1,
   # -x1, h2 - y2,
   # w2 + x2, h2 + y2,
   # w2 - x2, -y1,
   #)
   #im = im.resize((w2, h2))
   #im = im.transform((w, h), Image.QUAD, data)
   return im
   
  images = []
  for c in chars:
   images.append(_draw_character(c))

  text_width = sum([im.size[0] for im in images])
  width = max(text_width, self._width)
  image = image.resize((width, self._height))
  average = int(text_width / len(chars))
  rand = int(0.25 * average)
  offset = int(average * 0.1)
\end{lstlisting}
\begin{lstlisting}[frame=single]
  for im in images:
   w, h = im.size
   mask = im.convert('L').point(lambda i: i * 1.97)
   image.paste(im, (offset, int((self._height - h) / 2)), mask)
   offset = offset + w + random.randint(-rand, 0)

  return image

 def generate_image(self, chars):
  """Generate the image of the given characters.

  :param chars: text to be generated.
  """
  background = random_color(238, 255)
  color = random_color(0, 200, random.randint(220, 255))
  im = self.create_captcha_image(chars, color, background)
  self.create_noise_dots(im, color)
  self.create_noise_curve(im, color)
  im = im.filter(ImageFilter.SMOOTH)
  return im


def random_color(start, end, opacity=None):
 red = random.randint(start, end)
 green = random.randint(start, end)
 blue = random.randint(start, end)
 if opacity is None:
  return (red, green, blue)
 return (red, green, blue, opacity)
\end{lstlisting}
\begin{itemize}
\item Empaquetado de imágenes CAPTCHA (empaquetar.py).
\end{itemize}

\begin{lstlisting}[frame=single]
from Ek_din import cifrar
from subprocess import call
import types
import os
from os import path

def listFiles(folder):
 return [d for d in os.listdir(folder) 
         if path.isfile(path.join(folder, d))]

def empaquetar(body,asunto,op):
 s=cifrar(body,asunto,op)
 asunto=asunto.replace(" ","_")
 disc={}
\end{lstlisting}
\begin{lstlisting}[frame=single]
 #print(s[1])
 ass=asunto+".zip"
 if type(s[1])==types.StringType:
  zi=call("zip -r "+ass+" "+s[1], shell=True)
  mv=call("mv ./"+ass+" ./CAPTCHAS", shell=True)
  rmm=call("rm -rf "+s[1], shell=True)
  return (s[0],"./CAPTCHAS/"+ass)
 else:
  zi=call("zip -r "+ass+" "+s[1][1], shell=True)
  mv=call("mv ./"+ass+" ./CAPTCHAS", shell=True)
  rmm=call("rm -rf "+s[1][1], shell=True)
  return (s[0],"./CAPTCHAS/"+ass)
\end{lstlisting}

\chapter{Código fuente del prototipo 9}
\label{Anexos C}

A continuación se muestra el código fuente desarrollado en el prototipo 9.
\begin{itemize}
\item Estructura de la base de datos (script.sql).
\end{itemize}

 \begin{lstlisting}[frame=single]
 CREATE TABLE IF NOT EXISTS `Mensaje` (
  `firma_digital` varchar(255) CHARACTER SET utf8 NOT NULL,
  `correo_destino` varchar(50) CHARACTER SET utf8 NOT NULL,
  `ruta_archivo` varchar(255) CHARACTER SET utf8 NOT NULL,
  `correo_electronico` varchar(50) CHARACTER SET utf8 NOT NULL,
  PRIMARY KEY (`correo_destino`,`firma_digital`)
 ) ENGINE=MyISAM DEFAULT CHARSET=utf8 COLLATE=utf8_unicode_ci;

 CREATE TABLE IF NOT EXISTS `Usuario` (
  `correo_electronico` varchar(50) CHARACTER SET utf8 NOT NULL,
  `nombre` varchar(150) CHARACTER SET utf8 NOT NULL,
  `contrasena` varchar(20) COLLATE utf8_unicode_ci NOT NULL
 ) ENGINE=MyISAM DEFAULT CHARSET=utf8 COLLATE=utf8_unicode_ci;
 \end{lstlisting}
\begin{itemize}
\item Alta de usuario en el servidor de CAPTCHAS (AltaUsuario.php).
\end{itemize}

 \begin{lstlisting}[frame=single]
 <html>
 <head>
  <title>Alta de usuario</title> 
 </head>

 <body>
<?php
$usuario = trim($_POST["nombre"]);
$contra = trim($_POST["contrasena"]);
$correo = trim($_POST["correo_electronico"]);
if (empty($usuario)){
 echo '<p name="respuesta">0</p>';
}elseif (empty($contra)) {
 echo '<p name="respuesta">1</p>';
\end{lstlisting}
\begin{lstlisting}[frame=single]
}elseif (empty($correo)) {
 echo '<p name="respuesta">2</p>';
}else{
 $enlace = new mysqli('mysql.hostinger.mx', 'u715698692_corre',
                      'correocifrado','u715698692_corre');
 if($enlace->connect_errno){
  echo '<p name="respuesta">3</p>';
  die("Error en conexion");
 }

 if (!file_exists("./Usuarios/".$correo)) {
  mkdir("./Usuarios/".$correo);
 }

 $query = "SELECT nombre FROM Usuario 
            WHERE correo_electronico like '$correo'";
 $result = $enlace->query($query);
 $aux = $result->num_rows;
 if($aux == 1){
  echo '<p name="respuesta">5</p>';
  $enlace->close();
 }else{
  $result->free();
  if($enlace->query("INSERT INTO `Usuario`
  (`correo_electronico`, `nombre`, `contrasena`) 
  VALUES 
  ('".$correo."','".$usuario."','".$contra."')") === TRUE){
   echo '<p name="respuesta">5</p>';
   $enlace->close();
  }else{
   echo '<p name="respuesta">6</p>';
   $enlace->close();
  }
 }
}
?>
 </body>
</html>
 \end{lstlisting}

\begin{itemize}
\item Carga de imágenes CAPTCHAS en el servidor (AltaMensage.php).
\end{itemize}

 \begin{lstlisting}[frame=single]
 <html>
 <head>
  <title>Alta de Mensaje</title> 
 </head>
\end{lstlisting}
\begin{lstlisting}[frame=single]
 <body>
<?php
$usuario = trim($_POST["nombre"]);
$contra = trim($_POST["contrasena"]);
$correo = trim($_POST["correo_electronico"]);
$firma = trim($_POST["firma"]);
$correo_des = trim($_POST["correo_destino"]);

if (empty($usuario)){
 echo '<p name="respuesta">0</p>';
 die();
}elseif (empty($contra)) {
 echo '<p name="respuesta">1</p>';
 die();
}elseif (empty($correo)) {
 echo '<p name="respuesta">2</p>';
 die();
}elseif (empty($firma)) {
 echo '<p name="respuesta">3</p>';
 die();
}elseif (empty($correo_des)) {
 echo '<p name="respuesta">4</p>';
 die();
}elseif (!is_uploaded_file($_FILES["archivo"]["tmp_name"])){
 echo '<p name="respuesta">6</p>';
 die();
}else{
 $enlace = new mysqli('mysql.hostinger.mx', 'u715698692_corre', 
           'correocifrado','u715698692_corre');
 if($enlace->connect_errno){
  echo '<p name="respuesta">7</p>';
  die("Error en conexion");
 }
 $query = "SELECT nombre, contrasena 
           FROM Usuario 
           WHERE Correo_Electronico like '$correo' ";
 $result = $enlace->query($query);
 $aux = $result->num_rows;
 
 if($aux >0){
  $row = $result->fetch_array(MYSQLI_ASSOC);
 }else{
  echo '<p name="respuesta">11</p>';
  $enlace->close();
  die("Error de autenticacion");
 }
\end{lstlisting}
\begin{lstlisting}[frame=single]
 if(!(strcmp($row["nombre"],$usuario) == 0)){
  echo '<p name="respuesta">8</p>';
  $enlace->close();
  die("Error de autenticacion 1");
 }
 if (!(strcmp($row["contrasena"],$contra) == 0)) {
  echo '<p name="respuesta">8</p>';
  $enlace->close();
  die("Error de autenticacion 2");
 }
 $result->free();
 if(strcmp($_FILES['archivo']['type'], "application/zip")==0){
  $file= sha1($correo_des.$firma.$correo).".zip";
  $ruta=join(DIRECTORY_SEPARATOR,array("./Usuarios",
        $correo,$file));
  $query = "SELECT ruta_archivo 
            FROM Mensaje 
            WHERE Correo_Electronico like '$correo' 
            and firma_digital like '$firma' 
            and correo_destino like '$correo_des'";
  $result = $enlace->query($query);
  $aux = $result->num_rows;

  if($aux >0){
   echo '<p name="respuesta">12</p>';
   $enlace->close();
   die("Error de autenticacion");
  }else{
   $result->free();
   if (!file_exists($ruta)) {
    move_uploaded_file($_FILES['archivo']['tmp_name'], $ruta);
    echo "<pre>";
    print_r($ruta);
    if($enlace->query("INSERT INTO `Mensaje`
    (`firma_digital`, `correo_destino`, `ruta_archivo`, 
    `correo_electronico`) 
    VALUES ('".$firma."','".$correo_des."',
    '".$ruta."','".$correo."')") === TRUE){
     echo '<p name="respuesta">5</p>';
     $enlace->close();
    }else{
     echo '<p name="respuesta">9</p>';
     $enlace->close();
    }
   }else{
    echo '<p name="respuesta">13</p>';
\end{lstlisting}
\begin{lstlisting}[frame=single]
    $enlace->close();
    die("Error de autenticacion");
   }
  }
 }else{
  echo '<p name="respuesta">10</p>';
  $enlace->close();
 }

}
?>
 </body>
</html>
 \end{lstlisting}
\begin{itemize}
\item Descarga de imágenes CPATCHAS del servidor (BusquedaArchivo.php).
\end{itemize}

 \begin{lstlisting}[frame=single]
 <html>
 <head>
  <title>Busqueda de Archivos</title> 
 </head>

 <body>
<?php

$correo = trim($_POST["correo_electronico"]);
$firma = trim($_POST["firma"]);
$correo_des = trim($_POST["correo_destino"]);

if (empty($correo)) {
 echo '<p name="respuesta">0</p>';
 die();
}elseif (empty($firma)) {
 echo '<p name="respuesta">1</p>';
 die();
}elseif (empty($correo_des)) {
 echo '<p name="respuesta">2</p>';
 die();
}else{

 $enlace = new mysqli('mysql.hostinger.mx', 'u715698692_corre',
           'correocifrado','u715698692_corre');
 if($enlace->connect_errno){
  echo '<p name="respuesta">7</p>';
  die("Error en conexion");

 }
\end{lstlisting}
\begin{lstlisting}[frame=single]
 $query = "SELECT ruta_archivo 
           FROM Mensaje 
           WHERE Correo_Electronico like '$correo' 
           and firma_digital like '$firma' 
           and correo_destino like '$correo_des'";
 $result = $enlace->query($query);
 $aux = $result->num_rows;
 if($aux == 1){
  $row = $result->fetch_array(MYSQLI_ASSOC);
  $ruta = "http://correocifrado.esy.es".$row["ruta_archivo"];
  echo '<p name="respuesta">'.$ruta.'</p>';
  $enlace->close();
 }else{
  echo '<p name="respuesta">4</p>';
  $enlace->close();
  die("Error de autenticacion");
 }

}
?>
 </body>
</html>
 \end{lstlisting}

\chapter{Intalación de biblioteca GTK+ 3 y entorno gráfico GNOME 3}
\label{Anexos D}

A continuación se muestra los pasos a seguir para la intalación de las bibliotecas GTK3+ y el entorno grafico GNOME 3.

\section{Instalación del entorno gráfico GNOME 3.}


La instalación del entorno gráfico GNOME 3 se realizo en un sistema operativo XUBUNTU 15.1 y XUBUNTU 14.1. A continuación se explica los pasos a seguir para la instalación de este entorno gráfico.
\begin{itemize}
\item Se abre una terminal del sistema operativo.
\item Se ingresan los siguientes comandos a la terminal para instalar los repositorios de descarga.
\begin{lstlisting}
sudo add-apt-repository ppa:gnome3-team/gnome3
sudo add-apt-repository ppa:gnome3-team/gnome3-staging
\end{lstlisting}

\item Posteriormente se ingresa el siguiente comando a la terminal para actualizar los repositorios de descarga.
\begin{lstlisting}
sudo apt-get update
\end{lstlisting}

\item Una vez terminada la actualización se  ingresa este último comando para terminar con la instalación del entorno gráfico GNOME 3.
\begin{lstlisting}
sudo apt-get dist-upgrade
\end{lstlisting}
Una vez que la instalación termine se reinicia el equipo para activar el entorno gráfico GNOME 3.

\end{itemize}
\pagebreak
\section{Instalación de la biblioteca gráfica GTK+ 3.}

La instalación del biblioteca gráfica GTK+ 3 se realizo en un sistema operativo XUBUNTU 15.1 y XUBUNTU 14.1 siguiendo  los tutoriales proporcionados por la pagina de Python GTK+ 3 Tutorial y GNOME developer.
Uno de los requisitos previos para la instalación de GTK+ 3 es la instalación de JHBuild la cual se instalo siguiendo el tutorial de GNOME developer encontrado en la siguiente pagina web:\\
\url{https://developer.gnome.org/jhbuild/unstable/getting-started.html.es}\\

Después de la instalación de JHBuild se prosiguió con la instalación de la biblioteca gráfica GTK+ 3. A continuación se presentan los pasos a seguir para la instalación de de la biblioteca.
\begin{itemize}
\item Se abre una terminal del sistema operativo.
\item Se ingresan los siguientes comandos.
\begin{lstlisting}
$ jhbuild build pygobject
$ jhbuild build gtk+
$ jhbuild shell
\end{lstlisting}
\end{itemize}

\chapter{Código fuente del prototipo 10}
\label{Anexos E}

A continuación se muestra el código fuente desarrollado en el prototipo 10.
\begin{itemize}
\item Archivo de configuración JSON config.json).
\end{itemize}

\begin{lstlisting}[frame=single]
{
certfile: "./Seguridad/server2048.pem",
passwdSSE: "12345678",
passwd: "360_live",
portSmtp: "587",
portPop: "995",
ssl: true,
hostSmtp: "smtp-mail.outlook.com",
user: "jonny.test.arc.99@hotmail.com",
SSE: false,
nombre: "jonathan arcos",
hostPop: "pop3.live.com",
keyfile: "./Seguridad/server2048.key",
delete: 0
}
\end{lstlisting}
\begin{itemize}
\item Interfaz gráfica del cliente de correo electrónico (setup.py).
\end{itemize}

\begin{lstlisting}[frame=single]
import gi
import os
import email
import json
import SSE
import re
import captchas
import http
import logging

gi.require_version('Gtk', '3.0')
from gi.repository import Gtk, Gio
\end{lstlisting}
\begin{lstlisting}[frame=single]
from smtp2 import datosPrincipales
from smtp2 import validarSmtp
from listarCorreos import listaCorreosView
from listarCorreos import contarCorreo
from listarCorreos import body
from pop3 import conexionPop3
from pop3 import validarPop
from envios import envios
from salidaSmtp import salida

path = './Usuarios'

def listdirs(folder):
 return [d for d in os.listdir(folder) 
        if os.path.isdir(os.path.join(folder, d))]

def listFiles(folder):
 return [d for d in os.listdir(folder) 
        if os.path.isfile(os.path.join(folder, d))]

class MyWindow(Gtk.Window):

 user=listdirs(path)
 selectCarpeta=None
 selectUsuario=None
 config={}

 def __init__(self,config):
  self.config = config
  Gtk.Window.__init__(self, title="Cliente de Correos")
  self.set_border_width(4)
  self.set_default_size(800, 600)

  self.notebook = Gtk.Notebook()
  self.add(self.notebook)

  self.page = self.newPage()
  self.page.set_border_width(10)
  self.notebook.append_page(self.page, Gtk.Label('Index'))

 def visorCorreo(self):
  vistaCorre = Gtk.Box(orientation=Gtk.Orientation.VERTICAL, 
                       spacing=10)
  self.emisor = Gtk.Label("De: ")
  self.emisor.set_justify(Gtk.Justification.LEFT)
  self.destinatorio = Gtk.Label("Para: ")
\end{lstlisting}
\begin{lstlisting}[frame=single]
  self.destinatorio.set_justify(Gtk.Justification.LEFT)
  self.asunto = Gtk.Label("Asunto: ")
  self.asunto.set_justify(Gtk.Justification.LEFT)
  descifrado= Gtk.Button(label="Descifrar")
  descifrado.connect("clicked", self.descifrarBody)
  box1 = Gtk.VBox(False,10)
  box1.pack_start(self.emisor,True,True,0)
  box1.pack_end(self.destinatorio,False,True,0)
  box2 = Gtk.HBox(False,0)
  box2.pack_start(self.asunto,False, False, 0)
  box2.pack_end(descifrado,False,False,0)
  self.cuerpo = Gtk.TextView()
  self.cuerpo.set_wrap_mode(Gtk.WrapMode.WORD)
  self.cuerpo.set_editable(False)
  scrol = Gtk.ScrolledWindow()
  scrol.set_policy(Gtk.PolicyType.AUTOMATIC, 
                   Gtk.PolicyType.AUTOMATIC)
  scrol.set_vexpand(True)
  scrol.add(self.cuerpo)
  vistaCorre.add(box1)
  vistaCorre.add(box2)
  vistaCorre.add(scrol)
  return vistaCorre

 def visorCorreoNuevo(self):
  vistaCorre = Gtk.Box(orientation=Gtk.Orientation.VERTICAL, 
                       spacing=10)
                       
  newDest = Gtk.Entry(name="Destino")
  newDest.set_editable(True)
  newAsunto = Gtk.Entry(name="Asunto")
  newAsunto.set_editable(True)
  destinatorio = Gtk.Label("De: ")
  destinatorio.set_justify(Gtk.Justification.LEFT)
  asunto = Gtk.Label("Asunto: ")
  asunto.set_justify(Gtk.Justification.LEFT)
  Cerrar = Gtk.Button(label="Cerrar")
  Cerrar.connect("clicked", self.cerrarPagina)
  Enviar = Gtk.Button(label="Enviar")
  Enviar.connect("clicked", self.enviarMensage)
  box1 = Gtk.HBox(False,0)
  box1.pack_start(destinatorio,False,False,0)
  box1.pack_start(newDest,True,True,0)  
  box2 = Gtk.HBox(False,0)
  box2.pack_start(asunto,False, False, 0)
  box2.pack_start(newAsunto,True, True, 0)
\end{lstlisting}
\begin{lstlisting}[frame=single]
  box3 = Gtk.HBox(False,0)
  box3.pack_end(Cerrar,False, False, 0)
  box3.pack_end(Enviar,False, False, 0)

  cuerpo = Gtk.TextView(name="cuerpo")
  cuerpo.set_wrap_mode(Gtk.WrapMode.WORD)
  #WRAP_WORD
  scrol = Gtk.ScrolledWindow()
  scrol.set_policy(Gtk.PolicyType.AUTOMATIC, 
                   Gtk.PolicyType.AUTOMATIC)
  scrol.set_vexpand(True)
  scrol.add(cuerpo)
  vistaCorre.add(box1)
  vistaCorre.add(box2)
  vistaCorre.add(scrol)
  vistaCorre.add(box3)
  return vistaCorre

 def cerrarPagina(self,button):
  print("cerrarPagina")
  page=self.notebook.get_current_page()
  self.notebook.remove_page(page)
  self.notebook.show_all()

 def enviarMensage(self,button):
  print("enviarMensage")
  page=self.notebook.get_current_page()
  contenedor = self.notebook.get_nth_page(page)
  asunto = ""
  destino = ""
  cuerpo = ""
  for c in contenedor.get_children():
    for x in c.get_children():
     if isinstance(x,Gtk.Entry):
      if x.get_name() == "Destino":
       destino = x.get_text()
      elif x.get_name() == "Asunto":
       asunto = x.get_text()
     if isinstance(x,Gtk.TextView):
      buf = x.get_buffer()
      end_iter = buf.get_end_iter()
      start_iter = buf.get_start_iter()
      cuerpo = x.get_buffer().get_text(start_iter, end_iter, True)
  if not destino.split():
   dialog = Gtk.MessageDialog(self, 0, Gtk.MessageType.ERROR,
    Gtk.ButtonsType.CANCEL, "Error al enviar Mensaje")
\end{lstlisting}
\begin{lstlisting}[frame=single]
   dialog.format_secondary_text(
    "El correo Destinatario no ha sido ingresado.")
   dialog.run()
   dialog.destroy()
  elif not asunto and not cuerpo:
   dialog = Gtk.MessageDialog(self, 0, Gtk.MessageType.WARNING,
    Gtk.ButtonsType.OK_CANCEL, "Mensaje vacio")
   dialog.format_secondary_text(
    "El mensaje de correo esta vacio, decea que se envie?")
   response = dialog.run()
   if response == Gtk.ResponseType.OK:
    print("Mensaje Incompleto")
    t = envios(destino,self.user[0],asunto,cuerpo,[],self.config)
    t.start()
   dialog.destroy()
  else:
   print("Mensaje Completo")
   t = envios(destino,self.user[0],asunto,cuerpo,[],self.config)
   t.start()
  print("Asunto: "+asunto)
  print("Destino: "+destino)
  print("Cuerpo: "+cuerpo)
  print("cerrarPagina")
  page=self.notebook.get_current_page()
  self.notebook.remove_page(page)
  self.notebook.show_all()
  #print(contenedor.query_child_packing())

 def descifrarBody(self, button):
  print "Descifrar body"
  bodyBuffer=self.cuerpo.get_buffer()
  start_iter = bodyBuffer.get_start_iter()
  end_iter = bodyBuffer.get_end_iter()
  text = bodyBuffer.get_text(start_iter, end_iter, True) 
  firma=text.find("------SSE Cipher------")
  if(firma>=0):
   text2 = text[firma:]
   m = re.search('\-\n(.+)\n\-',text2)
   if(m!=None):
    textFirma = m.group(1)
    print textFirma
    correoOrigen=self.emisor.get_text()
    correoDestino=self.destinatorio.get_text()
    m = re.search(
    "([(a-z0-9\_\-\.)]+@[(a-z0-9\_\-\.)]+\.[(a-z)]{2,15})"
    ,correoOrigen)
\end{lstlisting}
\begin{lstlisting}[frame=single]
    correoOriegen = m.group(1)
    m = re.search(
    "([(a-z0-9\_\-\.)]+@[(a-z0-9\_\-\.)]+\.[(a-z)]{2,15})"
    ,correoDestino)
    
    correoDestino = m.group(1)
    despliegue=captchas.buscarCAPTCHAS(textFirma,
                                       correoDestino,
                                       correoOriegen)
    print(correoOriegen)
    print(correoDestino)
    if len(despliegue)>0:
     ventanaCaptcha(self,despliegue)
    else:
     dialog = Gtk.MessageDialog(self, 0, Gtk.MessageType.ERROR,
     Gtk.ButtonsType.CANCEL, "Error al descargar CAPTCHA")
     dialog.format_secondary_text(
      "Ocurrio un error con el servidor, intentarlo mas tarde")
     dialog.run()
     dialog.destroy()

 def listaMail(self, usuario, carpeta):
  
  software_liststore = Gtk.ListStore(str, str, str, str)
  #archivos = listdirs(path+usuario+"/"+carpeta)
  if self.selectCarpeta == None:
   archivos = [("", "",  "", "")]
  else:
   archivos = [("prueba", "prueba",  "prueba", "mail-attachment")]
  for archivo in archivos:
   software_liststore.append(list(archivo))
  lista = software_liststore.filter_new()
  self.listaCar = Gtk.TreeView.new_with_model(lista)
  self.listaCar.connect("row-activated",self.celdasCorreo)
  for i, column_title in enumerate(["Asunto", 
                                    "Correo", 
                                    "Fecha", 
                                    "Adjunto"]):                   

   if i == 3:
    renderer = Gtk.CellRendererPixbuf()
    column = Gtk.TreeViewColumn(column_title,renderer,icon_name=i)
   else:
    renderer = Gtk.CellRendererText()
    column = Gtk.TreeViewColumn(column_title, renderer, text=i)
   self.listaCar.append_column(column)
\end{lstlisting}
\begin{lstlisting}[frame=single]
  scrollable_treelist = Gtk.ScrolledWindow()
  scrollable_treelist.set_vexpand(True)
  scrollable_treelist.set_hexpand(True)
  scrollable_treelist.add(self.listaCar)
  
  return scrollable_treelist

 def listaCarpetas(self, usuarios):
  treestore = Gtk.TreeStore(str)
  numCorreo = 0
  for usuario in usuarios:
   carpetas = listdirs(os.path.join(path,usuario))
   piter = treestore.append(None, ['%s' % usuario])
   for carpeta in carpetas:
    numCorreo = contarCorreo(os.path.join(path,usuario,carpeta))
    if carpeta=="Entrada":
     treestore.prepend(piter, ['%s \t %d' % (carpeta, numCorreo)])
    else:
     treestore.append(piter, ['%s \t %d' % (carpeta, numCorreo)])

  treeview = Gtk.TreeView(treestore)   
  tvcolumn = Gtk.TreeViewColumn('Cuentas de Correos')
  tvcolumn.set_reorderable(False)
  treeview.append_column(tvcolumn)
  treeview.connect("row-activated",self.celdasCarp)
  cell = Gtk.CellRendererText()
  tvcolumn.pack_start(cell, True)
  tvcolumn.add_attribute(cell, 'text', 0)
  treeview.set_search_column(0)
  tvcolumn.set_sort_column_id(0)
  treeview.set_reorderable(False)
  return treeview

 def celdasCorreo(self, treeview, posi, column):
  model = treeview.get_model()
  car = model.get_iter(posi)
  correo = (model.get_value(car, 0),
            model.get_value(car, 1),
            model.get_value(car, 2),
            model.get_value(car, 3))
            
  for key in self.listaCotejoCorreos:
   aux = self.listaCotejoCorreos[key]
   if ((aux[1]==correo[1]) and (aux[2]==correo[2])):
    archivo=key
    break
\end{lstlisting}
\begin{lstlisting}[frame=single]
  ruta=os.path.join(path, 
                    self.selectUsuario, 
                    self.selectCarpeta, 
                    archivo)

  fp=open(ruta,"r")
  ms = email.message_from_file(fp)
  fp.close()
  self.destinatorio.set_text("Para: "+ms['To'])
  self.emisor.set_text("De: "+ms['From'])
  self.asunto.set_text("Asunto: "+ms['Subject'])
  textbody = body(ms)
  print(textbody)
  buffered = Gtk.TextBuffer()
  buffered.set_text(textbody.strip())
  self.cuerpo.set_buffer(buffered)
  #print(correo in self.listaCotejoCorreos)

 def celdasCarp(self, treeview, posi, column):
  model = treeview.get_model()
  car = model.get_iter(posi)
  carpeta = model.get_value(car, 0)
  if carpeta.find('@')>0:
   return
  carpeta = carpeta.split('\t')[0].strip()
  self.selectCarpeta=carpeta
  usu = model.iter_parent(car)
  usuario = model.get_value(usu, 0)
  self.selectUsuario=usuario
  self.listaCotejoCorreos = listaCorreosView(
                            os.path.join(path,usuario,carpeta))
                            
  software_liststore = Gtk.ListStore(str, str, str, str)
  for reg in self.listaCotejoCorreos: 
   software_liststore.append(self.listaCotejoCorreos[reg])
  lista = software_liststore.filter_new()
  self.listaCar.set_model(lista)

 def headerMail(self):
  box = Gtk.HBox(False,0)
  botonNewMail = Gtk.Button(label="Nuevo correo")
  botonNewMail.connect("clicked", self.nuevoCorreo,"newMail")
  botonEnviarRecibir = Gtk.Button(label="Enviar y Recibir")
  botonEnviarRecibir.connect("clicked", self.enviarRecibir)
  botonHerramientas = Gtk.Button(label="Herramientas")
  box.pack_start(botonNewMail, False, False, 0)
\end{lstlisting}
\begin{lstlisting}[frame=single]
  box.pack_start(botonEnviarRecibir, False, False, 0)
  #box.pack_end(botonHerramientas, False, False, 0)
  return box

 def nuevoCorreo(self, button,name):
  print(name)
  self.pageNuevoCorreo = self.visorCorreoNuevo()
  self.pageNuevoCorreo.set_border_width(10)
  self.notebook.insert_page(self.pageNuevoCorreo, 
                            Gtk.Label("Nuevo Correo"),1)
  self.notebook.show_all()

 def enviarRecibir(self, button):
  print(self.config)
  host = self.config["hostPop"]
  port = self.config["portPop"]
  keyfile = self.config["keyfile"]
  certfile = self.config["certfile"]
  user = self.config["user"]
  passwd = self.config["passwd"]
  ssl = self.config["ssl"]
  delete = self.config["delete"]
  ruta = os.path.join(path,user,"Entrada")
  t=conexionPop3(host, 
                 port, 
                 keyfile, 
                 certfile, 
                 user, 
                 passwd, 
                 ssl, 
                 delete, 
                 ruta)
  t.start()
  t2=salida(os.path.join(path,user),self.config)
  t2.start()

 def newPage(self):
  marco = Gtk.Box(orientation=Gtk.Orientation.VERTICAL, spacing=10)
  barra = self.headerMail()
  areaCorreo = Gtk.Box(spacing=10)
  listaCap = Gtk.Box(orientation=Gtk.Orientation.VERTICAL, 
                     spacing=10)
                    
  listaCap.add(self.listaCarpetas(self.user))
  areaViewCorreo = Gtk.Box(orientation=Gtk.Orientation.VERTICAL, 
                           spacing=10)
\end{lstlisting}
\begin{lstlisting}[frame=single]
  listaCorreo = Gtk.Box(spacing=10)
  self.listaM = self.listaMail(self.user[0],'Entrada')
  listaCorreo.add(self.listaM)
  
  viewCorreo = Gtk.Box(spacing=10)
  self.visorCo = self.visorCorreo()
  viewCorreo.add(self.visorCo)

  areaViewCorreo.pack_start(listaCorreo, False, True, 0)
  areaViewCorreo.pack_start(viewCorreo, True, True, 0)
  areaCorreo.add(listaCap)
  areaCorreo.add(areaViewCorreo)

  marco.pack_start(barra, False, False, 0)
  marco.pack_end(areaCorreo, True, True, 0)
  return marco

 def cuerpoDk(self,valores,op):
  print(valores)
  bodyBuffer=self.cuerpo.get_buffer()
  start_iter = bodyBuffer.get_start_iter()
  
  end_iter = bodyBuffer.get_end_iter()
  text = bodyBuffer.get_text(start_iter, end_iter, True) 
  firma=text.find("------SSE Cipher------")
  
  text= text[:firma]
  descifrado=SSE.Ek_din.descifrar(text,valores,op)
  
  try:
   aux=descifrado.decode("utf8")
   buf = Gtk.TextBuffer()
   print("utf-8 encode")
   buf.set_text(aux.encode("utf8"))
   self.cuerpo.set_buffer(buf)
   
  except Exception, e:
  
   print("Error al descifrar")
   dialog = Gtk.MessageDialog(self, 0, Gtk.MessageType.ERROR,
    Gtk.ButtonsType.CANCEL, "Error al descifrar")
    
   dialog.format_secondary_text(
    "El CAPTCHA fue ingresado incorrectamente")
   dialog.run()
   dialog.destroy()
\end{lstlisting}
\begin{lstlisting}[frame=single]
class ventanaCaptcha(Gtk.Window):
 
 def __init__(self,ventana,despliegue):
  self.ventana = ventana
  self.ruta=despliegue[0]
  self.archivos=despliegue[1]
  self.op=despliegue[2]
  Gtk.Window.__init__(self, title="Cliente de Correos")
  self.set_border_width(4)
  self.set_default_size(500,300)
  self.add(self.viewCAPTCHAS())
  self.show_all()

 def descifrado(self, button):
  for c in self.box1.get_children():
   for x in c.get_children():
    if isinstance(x,Gtk.Entry):
     print(x.get_name())
     valor=x.get_text()
     print(valor)
  if valor=="":
   dialog = Gtk.MessageDialog(self, 0, Gtk.MessageType.ERROR,
    Gtk.ButtonsType.CANCEL, "Error en el CAPTCHA")
   dialog.format_secondary_text(
    "Resolver el CAPTCHA")
   dialog.run()
   dialog.destroy()
  else:
   self.ventana.cuerpoDk(valor,self.op)

 def descifrado2(self, button):
  valor=[]
  for c in self.box1.get_children():
   for x in c.get_children():
    if isinstance(x,Gtk.Entry):
     print(x.get_name())
     aux=x.get_text()
     if aux!="":
      valor.append([self.archivos[x.get_name()],x.get_text()])
      print(valor)
  for c in self.box2.get_children():
   for x in c.get_children():
    if isinstance(x,Gtk.Entry):
     print(x.get_name())
     aux=x.get_text()
     if aux!="":
\end{lstlisting}
\begin{lstlisting}[frame=single]
      valor.append([self.archivos[x.get_name()],x.get_text()])
      print(valor)
  for c in self.box3.get_children():
   for x in c.get_children():
    if isinstance(x,Gtk.Entry):
     print(x.get_name())
     aux=x.get_text()
     if aux!="":
      valor.append([self.archivos[x.get_name()],x.get_text()])
      print(valor)
  if len(valor)==0:
   dialog = Gtk.MessageDialog(self, 0, Gtk.MessageType.ERROR,
    Gtk.ButtonsType.CANCEL, "Error en el CAPTCHA")
   dialog.format_secondary_text(
    "Resolver el CAPTCHA")
   dialog.run()
   dialog.destroy()
  else:
   self.ventana.cuerpoDk(valor,self.op)

 def viewCAPTCHAS(self):
  marco = Gtk.Box(orientation=Gtk.Orientation.VERTICAL, spacing=10)
  scrol = Gtk.ScrolledWindow()
  scrol.set_hexpand(True)
  boxGen=Gtk.VBox(False,0)
  boxGen.set_spacing(10)
  boxGen.set_border_width(10)
  separator = Gtk.HSeparator()
  separator.set_size_request(400, 5)
  separator2 = Gtk.HSeparator()
  separator2.set_size_request(400, 5)
  self.box1=Gtk.HBox(False,0)
  boxGen.pack_start(self.box1,False,False,0)
  boxGen.pack_start(separator, False, True, 5)
  self.box2=Gtk.HBox(False,0)
  boxGen.pack_start(self.box2,False,False,0)
  boxGen.pack_start(separator2, False, True, 5)
  self.box3=Gtk.HBox(False,0)
  boxGen.pack_start(self.box3,False,False,0)
  box4=Gtk.HBox(False,0)
  descifrado= Gtk.Button(label="Descifrar")
  if isinstance(self.archivos,dict):
   index=0
   for img in self.archivos.keys():
    self.set_default_size(700,400)
    aux = Gtk.VBox(False,0)
\end{lstlisting}
\begin{lstlisting}[frame=single]
    texto = Gtk.Entry(name=img)
    image = Gtk.Image()
    rutaImg=os.path.join(self.ruta,img)
    print(rutaImg)
    image.set_from_file(rutaImg)
    image.show()
    aux.pack_start(image,False,False,0)
    aux.pack_end(texto,False,False,0)
    if index<2:
     self.box1.pack_start(aux,False,False,0)
    if index<4:
     self.box2.pack_start(aux,False,False,0)
    else:
     self.box3.pack_start(aux,False,False,0)
    index+=1
   descifrado.connect("clicked", self.descifrado2)
   
  else:
   for img in self.archivos:
    aux = Gtk.VBox(False,0)
    texto = Gtk.Entry(name=img)
    image = Gtk.Image()
    
    rutaImg=os.path.join(self.ruta,img)
    print(rutaImg)
    image.set_from_file(rutaImg)
    image.show()
    aux.pack_start(image,False,False,0)
    aux.pack_end(texto,False,False,0)
    self.box1.pack_start(aux,False,False,0)
   descifrado.connect("clicked", self.descifrado)

  box4.pack_end(descifrado,False,False,0)
  marco.pack_start(boxGen,False,False,0)
  marco.pack_start(box4,False,False,0)
  scrol.add(marco)
  return scrol

class configView(Gtk.Window):

 def __init__(self):
  Gtk.Window.__init__(self, title="Configuracion")
  self.set_border_width(4)
  self.set_default_size(500, 600)
  self.add(self.viewConfig())
  self.show_all()
\end{lstlisting}
\begin{lstlisting}[frame=single]
 def viewConfig(self):
 
  marco = Gtk.Box(orientation=Gtk.Orientation.VERTICAL, spacing=10)
  marco.pack_start(Gtk.Label("Servidor Smtp"),False,False,0)
  
  self.servidorSmtp=Gtk.Entry()
  marco.pack_start(self.servidorSmtp,False,False,0)
  marco.pack_start(Gtk.Label("Puerto Smtp"),False,False,0)
  
  self.puertoSmtp=Gtk.Entry()
  marco.pack_start(self.puertoSmtp,False,False,0)
  marco.pack_start(Gtk.Label("Servidor Pop"),False,False,0)
  
  self.servidorPop=Gtk.Entry()
  marco.pack_start(self.servidorPop,False,False,0)
  marco.pack_start(Gtk.Label("Puerto Pop"),False,False,0)
  
  self.puertoPop=Gtk.Entry()
  marco.pack_start(self.puertoPop,False,False,0)
  marco.pack_start(Gtk.Label("Usuario de Correo Electronico")
                             ,False,False,0)
                             
  self.usuCorreoElec=Gtk.Entry()
  marco.pack_start(self.usuCorreoElec,False,False,0)
  marco.pack_start(Gtk.Label("Contrasena de Correo Elecctronico")
                             ,False,False,0)
                             
  self.contraCorreoElec=Gtk.Entry()
  self.contraCorreoElec.set_visibility(False)
  marco.pack_start(self.contraCorreoElec,False,False,0)
  marco.pack_start(Gtk.Label("Conexion POP SSL"),False,False,0)
  
  self.conexSSL=Gtk.Switch()
  self.conexSSL.set_active(False)
  marco.pack_start(self.conexSSL,False,False,0)
  marco.pack_start(Gtk.Label("Usuario del Servidor de CAPTCHAS")
                             ,False,False,0)
                             
  self.usuSerCAPTCHA=Gtk.Entry()
  marco.pack_start(self.usuSerCAPTCHA,False,False,0)
  marco.pack_start(Gtk.Label("Contrasena del Servidor de CAPTCHAS")
                             ,False,False,0)
                             
  self.contraSerCAPTCHA=Gtk.Entry()
  self.contraSerCAPTCHA.set_visibility(False)
  marco.pack_start(self.contraSerCAPTCHA,False,False,0)
\end{lstlisting}
\begin{lstlisting}[frame=single]
  marco.pack_start(Gtk.Label(
                   "Activar Esquema de Secreto Compartido"),
                   False,False,0)
                             
  self.SSE=Gtk.Switch()
  self.SSE.set_active(False)
  marco.pack_start(self.SSE,False,False,0)
  boton = Gtk.Button(label="Activar")
  
  boton.connect("clicked", self.Activar)
  marco.pack_start(boton,False,False,0)
  return marco
  
 def Activar(self,button):
 
  dic={}
  dic["nombre"]=self.usuCorreoElec.get_text()
  dic["contrasena"]=self.contraSerCAPTCHA.get_text()
  dic["correo_electronico"]=self.usuCorreoElec.get_text()
  
  if http.httpAltaUsu(dic):
   disc={}
   disc["host"]=self.servidorPop.get_text()
   disc["port"]=self.puertoPop.get_text()
   disc["keyfile"]="./Seguridad/server2048.key"
   disc["certfile"]="./Seguridad/server2048.pem"
   disc["user"]=self.usuCorreoElec.get_text()
   disc["passwd"]=self.contraCorreoElec.get_text()
   disc["ssl"]=self.conexSSL.get_active()
   
   if validarPop(disc):
    disc["host"]=self.servidorSmtp.get_text()
    disc["port"]=self.puertoSmtp.get_text()
    disc["ssl"]=False
    
    if validarSmtp(disc):
     disc={}
     disc["hostSmtp"]=self.servidorSmtp.get_text()
     disc["portSmtp"]=self.puertoSmtp.get_text()
     disc["hostPop"]=self.servidorPop.get_text()
     disc["portPop"]=self.puertoPop.get_text()
     disc["keyfile"]="./Seguridad/server2048.key"
     disc["certfile"]="./Seguridad/server2048.pem"
     disc["user"]=self.usuCorreoElec.get_text()
     disc["passwd"]=self.contraCorreoElec.get_text()
     disc["ssl"]=self.conexSSL.get_active()
\end{lstlisting}
\begin{lstlisting}[frame=single]
     disc["delete"]=0
     disc["SSE"]=self.SSE.get_active()
     disc["nombre"]=self.usuSerCAPTCHA.get_text()
     disc["passwdSSE"]=self.contraSerCAPTCHA.get_text()
     
     lista=open("config.json","w")   
     lista.write(json.dumps(disc))
     lista.close
     
     win = MyWindow(disc)
     win.connect("delete-event", Gtk.main_quit)
     win.show_all()
     
    else:
     dialog = Gtk.MessageDialog(self, 0, Gtk.MessageType.ERROR,
     Gtk.ButtonsType.CANCEL, "Error en el servidor SMTP")
     dialog.format_secondary_text(
      "No se logro estableser comunicacion con el 
      servidor SMTP, verificar los datos ingresados")
     dialog.run()
     dialog.destroy()
     
   else :
    dialog = Gtk.MessageDialog(self, 0, Gtk.MessageType.ERROR,
    Gtk.ButtonsType.CANCEL, "Error en el servidor POP")
    dialog.format_secondary_text(
     "No se logro estableser comunicacion con el servidor POP, 
     verificar los datos ingresados")
    dialog.run()
    dialog.destroy()
    
  else:
   dialog = Gtk.MessageDialog(self, 0, Gtk.MessageType.ERROR,
   Gtk.ButtonsType.CANCEL, "Error en el servidor de CAPTCHAS")
   dialog.format_secondary_text(
    "Ocurrio un error en el registro como 
    usuario en el servidor de CAPTCHAS")
   dialog.run()
   dialog.destroy()
   
def setup_logger(logger_name, log_file, level=logging.INFO):

 l = logging.getLogger(logger_name)
 formatter = logging.Formatter('%(asctime)s : %(message)s')
 fileHandler = logging.FileHandler(log_file, mode='w')
 fileHandler.setFormatter(formatter)
\end{lstlisting}
\begin{lstlisting}[frame=single]
 streamHandler = logging.StreamHandler()
 streamHandler.setFormatter(formatter)

 l.setLevel(level)
 l.addHandler(fileHandler)
 l.addHandler(streamHandler) 

setup_logger('debug', r'./logs/debug.log')
setup_logger('errorLog', r'./logs/error.log')
debug = logging.getLogger('debug')
errorLog = logging.getLogger('errorLog')

win=None
if not os.path.exists("config.json"):
 debug.info('Inicia Aplicacion')
 win = configView()
 win.connect("delete-event", Gtk.main_quit)
 win.show_all()
 Gtk.main()

else:
 jsonCorreo = open("config.json","r")
 jsonLectura = jsonCorreo.readline()
 jsonCorreo.close()
 configuracion = json.loads(jsonLectura)
 win = MyWindow(configuracion)
 win.connect("delete-event", Gtk.main_quit)  
 win.show_all()
 Gtk.main()

\end{lstlisting}
\begin{itemize}
\item Conexión SMTP (smtp2.py).
\end{itemize}

\begin{lstlisting}[frame=single]
import os
import smtplib
import mimetypes
import hashlib
import time
import email
# For guessing MIME type based on file name extension
from email import encoders
from email.message import Message
from email.mime.audio import MIMEAudio
from email.mime.base import MIMEBase
from email.mime.image import MIMEImage
from email.mime.multipart import MIMEMultipart
from email.mime.text import MIMEText
\end{lstlisting}
\begin{lstlisting}[frame=single]
def validarSmtp(dat):
 print("abriendo conexion")
 try:
  if dat["ssl"]:
   M = smtplib.SMTP_SSL(host=dat["host"], port=dat["port"], 
                        keyfile=dat["keyfile"], 
                        certfile=dat["certfile"])
  else:
   M = smtplib.SMTP(host=dat["host"], port=dat["port"])
  #M.set_debuglevel(True)
 except Exception, e:
  print(e)
  print("Error de conexion")
  return False
 print("Validando usuario")
 try:
  M.ehlo()
  M.starttls()
  M.ehlo()
  M.login(dat["user"], dat["passwd"])
  M.close()
  return True
 except Exception, e:
  print("Invalid credentials")
  return False

def smtpOneMensaje(to, subject, fromUser, text, attach, passwd
                   , server, port, op, ssl):
 outer = MIMEMultipart()
 if to == None:
  return 0
 elif fromUser == None:
  return 1
 elif (subject == None) and (text == None):
  return 2
 elif passwd == None:
  return 3
 elif server == None:
  return 4
 elif port == None:
  return 5

 outer['From'] = fromUser
 outer['To'] = to
 outer['Subject'] = subject
 outer['Date'] = time.asctime(time.localtime(time.time()))
 \end{lstlisting}
\begin{lstlisting}[frame=single]
 outer.attach(MIMEText(text))
 for path in attach:
  if not os.path.isfile(path):
   continue

  ctype, encoding = mimetypes.guess_type(path)
  if ctype is None or encoding is not None:
   ctype = 'application/octet-stream'
  maintype, subtype = ctype.split('/', 1)
  if maintype == 'text':
   fp = open(path)
   msg = MIMEText(fp.read(), _subtype=subtype)
   fp.close()
   
  elif maintype == 'image':
   fp = open(path, 'rb')
   msg = MIMEImage(fp.read(), _subtype=subtype)
   fp.close()
   
  elif maintype == 'audio':
   fp = open(path, 'rb')
   msg = MIMEAudio(fp.read(), _subtype=subtype)
   fp.close()
  else:
   fp = open(path, 'rb')
   msg = MIMEBase(maintype, subtype)
   msg.set_payload(fp.read())
   fp.close()
   encoders.encode_base64(msg)
  msg.add_header('Content-Disposition', 
                 'attachment', 
                 filename=os.path.basename(path))
                 
  outer.attach(msg) 
 composed = outer.as_string()
 if op:
  m = hashlib.md5()
  m.update(time.asctime(time.localtime(time.time())))
  aux = m.hexdigest()+".txt"
  fp = open(aux, 'w')
  fp.write(composed)
  fp.close()
  return 6
 else:
  print("abriendo conexion")
  try:
\end{lstlisting}
\begin{lstlisting}[frame=single]
   if ssl:
    s = smtplib.SMTP_SSL(server,port)
   else:
    s = smtplib.SMTP(server,port)
   #s.set_debuglevel(2)
   s.ehlo()
   print("tls")
   s.starttls()
   s.ehlo()
   print("login")
   print(fromUser,passwd)
   s.login(fromUser, passwd)
   print("enviando correo")
   s.sendmail(fromUser, to, composed)
   print("fin")
   s.close()
   return 6
  except Exception, e:
   m = hashlib.md5()
   m.update(time.asctime(time.localtime(time.time())))
   aux = m.hexdigest()+".txt"
   fp = open(aux, 'w')
   fp.write(composed)
   fp.close()
   return 8

def smtpAllMensaje(correos, user, passwd, server, port, ssl):
 if correos == None:
  return 0
 if user == None:
  return 1
 elif passwd == None:
  return 3
 elif server == None:
  return 4
 elif port == None:
  return 5

 print("abriendo conexion")
 try:
  if ssl:
   s = smtplib.SMTP_SSL(server,port)
  else:
   s = smtplib.SMTP(server,port)
  #s.set_debuglevel(2)
  s.ehlo()
\end{lstlisting}
\begin{lstlisting}[frame=single]
  print("tls")
  s.starttls()
  s.ehlo()
  print("login")
  s.login(user, passwd)
  print("enviando correo")
  for ms in correos:
   try:
    composed = ms.as_string()
    time.sleep(1)
    s.sendmail(ms['From'], ms['To'], composed)
   except Exception, e:
    m = hashlib.md5()
    m.update(time.asctime(time.localtime(time.time())))
    aux = m.hexdigest()+".txt"
    fp = open(aux, 'w')
    fp.write(composed)
    fp.close()
  print("fin")
  s.close()
  return 6
  
 except Exception, e:
  return 8

def smtpEnpaquetar(to, subject, fromUser, text, attach):
 outer = MIMEMultipart()
 if to == None:
  return 0
 elif fromUser == None:
  return 1
 elif (subject == None) and (text == None):
  return 2

 outer['From'] = fromUser
 outer['To'] = to
 outer['Subject'] = subject
 outer['Date'] = time.asctime(time.localtime(time.time()))

 outer.attach(MIMEText(text))
 for path in attach:
  if not os.path.isfile(path):
   continue
  ctype, encoding = mimetypes.guess_type(path)
  if ctype is None or encoding is not None:
   ctype = 'application/octet-stream'
\end{lstlisting}
\begin{lstlisting}[frame=single]
  maintype, subtype = ctype.split('/', 1)
  if maintype == 'text':
   fp = open(path)
   msg = MIMEText(fp.read(), _subtype=subtype)
   fp.close()
   
  elif maintype == 'image':
   fp = open(path, 'rb')
   msg = MIMEImage(fp.read(), _subtype=subtype)
   fp.close()
   
  elif maintype == 'audio':
   fp = open(path, 'rb')
   msg = MIMEAudio(fp.read(), _subtype=subtype)
   fp.close()
   
  else:
   fp = open(path, 'rb')
   msg = MIMEBase(maintype, subtype)
   msg.set_payload(fp.read())
   fp.close()
   encoders.encode_base64(msg)
  msg.add_header('Content-Disposition', 
                 'attachment', 
                 filename=os.path.basename(path))
                 
  outer.attach(msg)
 return outer

def smtpEnvio(ms,server,port,passwd,ssl):
 print("enviando smtp")
 fromUser = ms['From']
 to = ms['To']
 composed = ms.as_string()
 print("Datos del mensaje")
 print("Servidor: "+server)
 print("Port: "+str(port))
 print("Pass: "+str(passwd))
 try:
  s = smtplib.SMTP(server,port)
  s.ehlo()
  print("tls")
  s.starttls()
  s.ehlo()
  print("login")
  s.login(fromUser, passwd)
\end{lstlisting}
\begin{lstlisting}[frame=single]
  print("enviando correo")
  s.sendmail(fromUser, to, composed)
  print("fin")
  s.close()
  return 0
 except Exception, e:
  return 1

def datosPrincipales(arc):
 fp = open(arc, 'r')
 ms = email.message_from_file(fp)
 fp.close()
 if len(ms.get_payload())>1:
  res =(arc,ms['Subject'],ms['From'],ms['Date'],"mail-attachment")
 else:
  res =(ms['Subject'],ms['From'],ms['Date'],"")
 return list(res)

\end{lstlisting}
\begin{itemize}
\item Conexión POP3 (pop3.py).
\end{itemize}

\begin{lstlisting}[frame=single]
import os
import poplib
import string
import StringIO
import email
import hashlib
import listarCorreos
import threading
from listarCorreos import listaCorreos
# For guessing MIME type based on file name extension
from email import encoders
from email.message import Message
from email.mime.audio import MIMEAudio
from email.mime.base import MIMEBase
from email.mime.image import MIMEImage
from email.mime.multipart import MIMEMultipart
from email.mime.text import MIMEText

#path="/home/jonnytest/Documentos/Usuarios/"
class conexionPop3(threading.Thread):

 def __init__(self, host, port, keyfile, certfile, user, passwd, 
              ssl, delete, path):
  threading.Thread.__init__(self)
  self.host = host
  self.port = port
\end{lstlisting}
\begin{lstlisting}[frame=single]
  self.keyfile = keyfile
  self.certfile = certfile
  self.user = user
  self.passwd = passwd
  self.ssl = ssl
  self.delete = delete
  self.path = path
  
 def run(self):
  print("abriendo conexion")
  try:
   if self.ssl:
    M = poplib.POP3_SSL(host=self.host, port=self.port, 
                        keyfile=self.keyfile, 
                        certfile=self.certfile)
   else:
    M = poplib.POP3(host=self.host, port=self.port)
   #M.set_debuglevel(2)
  except poplib.error_proto, e:
   print(e)
   print("Error de conexion")
   return 0;
  success = False
  print("Validando usuario")
  while success == False:
   lista={}
   try:
    M.user(self.user)
    M.pass_(self.passwd)
    numMesanjes = len(M.list()[1])
    for id in range(1,(numMesanjes+1)):
     resp, text, octets = M.retr(id)
     text = string.join(text, "\n")
     ms = email.message_from_string(text)
     print(ms["Date"])
     m = hashlib.md5()
     m.update(ms["Date"])
     aux = m.hexdigest()+".txt"
     file=os.path.join(self.path,aux)
     if os.path.exists(file):
      if self.delete:
       m.dele(id)
     else:
      composed = ms.as_string()
      fp = open(file, 'w')
      fp.write(composed)
\end{lstlisting}
\begin{lstlisting}[frame=single]
      fp.close()
      if self.delete:
       m.dele(id)
    listaCorreos(self.path)
   except poplib.error_proto:
    print("Invalid credentials")
   else:
    print("Successful login")
    success = True
   finally:
    if M: 
     M.quit()

def validarPop(dat):
 print("abriendo conexion")
 try:
  if dat["ssl"]:
   M = poplib.POP3_SSL(host=dat["host"], port=dat["port"], 
                       keyfile=dat["keyfile"], 
                       certfile=dat["certfile"])
  else:
   M = poplib.POP3(host=dat["host"], port=dat["port"])
  #M.set_debuglevel(2)
 except Exception, e:
  print(e)
  print("Error de conexion")
  return False
 print("Validando usuario")
 try:
  M.user(dat["user"])
  M.pass_(dat["passwd"])
  assert M.noop() == '+OK'
 except poplib.error_proto:
  print("Invalid credentials")
  return False
 else:
  print("Successful login")
  success = True
  return True
 finally:
  if M: M.quit()
\end{lstlisting}
\pagebreak
\begin{itemize}
\item Conexión con el servidor de CAPTCHAS (http.py).
\end{itemize}

\begin{lstlisting}[frame=single]
import urllib
import urllib2
import re
from poster.encode import multipart_encode
from poster.streaminghttp import register_openers

def httpEnvio(data):
 register_openers()
 datagen, headers = multipart_encode(data)
 request = urllib2.Request(
           "http://correocifrado.esy.es/AltaMensage.php", 
           datagen, 
           headers)
 found=''
 try:
  for line in urllib2.urlopen(request):
   print(line)
   if line.find("name=\"respuesta\"") >= 0:
    m = re.search('>([0-9]+)<',line)
    if m:
     found = m.group(1)
     print(found)
     if found=="5":
      print("correcto")
      return True
     else:
      return False
    else:
     return False
 except Exception, e:
  return False

def httpDescarga(data,nomArc):
 register_openers()
 datagen, headers = multipart_encode(data)
 print("peticion Server")
 try:
  request = urllib2.Request(
            "http://correocifrado.esy.es/BusquedaArchivo.php", 
            datagen, 
            headers)
  found=''
  for line in urllib2.urlopen(request):
   if line.find("name=\"respuesta\"") >= 0:
    m = re.search('>(http.+zip)<',line)
\end{lstlisting}
\begin{lstlisting}[frame=single]
    if m:
     found = m.group(1)
     print(found)
     break
    else:
     return False
     
 except urllib2.HTTPError, e:
  print e.code
  return False
  
 except urllib2.URLError, e:
  print e.args
  return False
  
 try:
  res=urllib2.urlopen(found)
  arc=open(nomArc,"w")
  arc.write(res.read())
  arc.close()
  return True
  
 except urllib2.HTTPError, e:
  print e.code
  return False
  
 except urllib2.URLError, e:
  print e.args
  return False

def httpAltaUsu(data):
 register_openers()
 datagen, headers = multipart_encode(data)
 request = urllib2.Request(
           "http://correocifrado.esy.es/AltaUsuario.php", 
           datagen, 
           headers)

 found=''
 try:
  for line in urllib2.urlopen(request):
   print(line)
   if line.find("name=\"respuesta\"") >= 0:
    m = re.search('>([0-9]+)<',line)
    if m:
     found = m.group(1)
\end{lstlisting}
\begin{lstlisting}[frame=single]
     print(found)
     if found=="5":
      print("correcto")
      return True
     else:
      return False
    else:
     return False
 except Exception, e:
  return False
\end{lstlisting}
\begin{itemize}
\item Envío de CAPTCHAS (envios.py).
\end{itemize}

\begin{lstlisting}[frame=single]
import threading
import hashlib
import time
import os
import re
import urllib2
import email
import SSE
import http
from subprocess import call
from SSE import empaquetar
#from empaquetar import empaquetar
from subprocess import call

from smtp2 import smtpEnpaquetar, smtpEnvio
from listarCorreos import listaCorreos, listaCorreosView

class envios(threading.Thread):
 
 def __init__(self, correoDes, correoOri, asunto, body, 
              attach, config):
              
  threading.Thread.__init__(self)
  self.correoD = correoDes
  self.correoO = correoOri
  self.asunto = asunto
  self.cuerpo = body
  self.attachment = attach
  self.configuracion=config
 
 def run(self):
  print("envios")
  firma =self.firma()
\end{lstlisting}
\begin{lstlisting}[frame=single]
  body,ruta = empaquetar(self.cuerpo,
                         firma,
                         self.configuracion["SSE"])
  
  body += "\n------SSE Cipher------\n"
  body += firma
  body += "\n------SSE Cipher------\n"
  print(body)
  mv=call("mv "+ruta+" ./CAPTCHAS/"+firma+".zip", shell=True)
  ms=smtpEnpaquetar(self.correoD, 
                    self.asunto, 
                    self.correoO, 
                    body, 
                    self.attachment)
                    
  path=os.path.join("./Usuarios",self.correoO,
                    "Salida")
  print(ms.as_string())
  print(os.path.join(path,firma+".txt"))
  fp = open(os.path.join(path,firma+".txt"), 'w')
  fp.write(ms.as_string())
  fp.close()
  print(path)
  listaCorreos(path)
  self.enviarCorreos(os.path.join("./Usuarios",self.correoO),
                     "Salida")

 def firma(self):
  m = hashlib.md5()
  localtime = time.asctime( time.localtime(time.time()) )
  m.update(self.correoD+localtime+self.correoO)
  return m.hexdigest()

 def enviarCorreos(self,path,carpeta):
  try:
   response=urllib2.urlopen('http://correocifrado.esy.es',
                            timeout=1)
                            
   print("Conexion al servidor")
   salida=os.path.join(path,carpeta)
   dic=listaCorreosView(salida)
   print("Diccionario")
   httpPar={}
   httpPar["nombre"]=self.configuracion["nombre"]
   httpPar["contrasena"]=self.configuracion["passwdSSE"]
   for arc in dic.keys():
\end{lstlisting}
\begin{lstlisting}[frame=single]
    try:
     m = re.search('^(.+)\.txt$',arc)
     firma = m.group(1)
     print("Firma:   "+firma)
     fp = open(os.path.join(salida,arc), 'r')
     ms = email.message_from_file(fp)
     fp.close()
     httpPar["correo_electronico"]=ms["From"]
     httpPar["correo_destino"]=ms["To"]
     httpPar["firma"]=firma
     httpPar["archivo"]=open(
                        os.path.join("./CAPTCHAS",firma+".zip"))
                        
     print(os.path.join("./CAPTCHAS",firma+".zip"))
     print(httpPar)
     if http.httpEnvio(httpPar):
      smtpEnvio(ms,self.configuracion["hostSmtp"], 
                self.configuracion["portSmtp"],
                self.configuracion["passwd"], 
                False)
                
      origen = os.path.join(salida,arc)
      destino =os.path.join(path,"Enviados",arc)
      instruc = "mv "+origen+" "+destino
      call(instruc, shell=True) 
    except Exception, e:
     print("Error al enviar Correo Electronico")
   listaCorreos(os.path.join(path,"Enviados"))
   listaCorreos(salida)
  except urllib2.URLError as err: 
   print("Sin conexion")
#aux = envios("sdfg", "dsfg","asdfd","dfg",[])
#aux.run()
\end{lstlisting}
\begin{itemize}
\item Busqueda de CAPTCHAS (captchas.py).
\end{itemize}

\begin{lstlisting}[frame=single]
import os
from os import path
import json
import http
from subprocess import call

def listFiles(folder):
 return [d for d in os.listdir(folder) 
         if path.isfile(path.join(folder, d))]
\end{lstlisting}
\begin{lstlisting}[frame=single]
def buscarCAPTCHAS(firma, correoDes, correoOri):
 ruta=path.join("./CAPTCHAS",firma)
 if path.exists(ruta):
  return listarCaptchas(ruta)
 elif path.exists(ruta+".zip"):
  unzip="unzip "+ruta+".zip -d ./CAPTCHAS"
  print(unzip)
  unz=call(unzip, shell=True)
  return listarCaptchas(ruta)
 else:
  datos={}
  datos["correo_electronico"]=correoOri
  datos["correo_destino"]=correoDes
  datos["firma"]=firma
  if http.httpDescarga(datos,ruta+".zip"):
   unzip="unzip "+ruta+".zip -d ./CAPTCHAS"
   print(unzip)
   unz=call(unzip, shell=True)
   rm=call("rm "+ruta+".zip", shell=True)
   return listarCaptchas(ruta)
  else:
   return []

#unzip archivo
def listarCaptchas(ruta):
 if path.exists(path.join(ruta,"lista.json")):
  op=1
  jsonCorreo = open(path.join(ruta,"lista.json"),"r")
  jsonLectura = jsonCorreo.readline()
  jsonCorreo.close()
  archivos = json.loads(jsonLectura)
  print(archivos)
 else:
  archivos = listFiles(ruta)
  print("NumArchivos: "+str(len(archivos)))
  print(archivos)
  op=0
 print([ruta,archivos,op])
 return [ruta,archivos,op]
\end{lstlisting}
\pagebreak
\begin{itemize}
\item Listado de mensajes de correo electrónico (listarCorreos.py)
\end{itemize}

\begin{lstlisting}[frame=single]
import json
import os
import email
import quopri

from os import path

def listFiles(folder):
 return [d for d in os.listdir(folder) 
         if path.isfile(path.join(folder, d))]

def listaCorreos(ruta):
 print("listaCorreos")
 archivos = listFiles(ruta)
 disc={}
 for archivo in archivos:
  if archivo.find(".txt")>0:
   fp = open(path.join(ruta,archivo), 'r')
   ms = email.message_from_file(fp)
   fp.close()
   if len(ms.get_payload())>1:
    disc[archivo]=(ms['Subject'],
                   ms['From'],
                   ms['Date'],
                   "mail-attachment")
   else:
    disc[archivo]=(ms['Subject'],ms['From'],ms['Date'],"")
 lista=open(path.join(ruta,"lista.json"),"w") 
 lista.write(json.dumps(disc))
 lista.close
 print("Fin listaCorreos")

def listaCorreosView(ruta):
 print("listaCorreosView")
 disc={}
 jsonCorreo = open(os.path.join(ruta,"lista.json"),"r")
 if jsonCorreo:
  jsonLectura = jsonCorreo.readline()
  jsonCorreo.close()
  dic = json.loads(jsonLectura)
  return dic
 else:
  return None
 print("Fin listaCorreosView")
\end{lstlisting}
\begin{lstlisting}[frame=single]
def contarCorreo(directorio):
 print("contarCorreo")
 if directorio==None:
  return -1
 for files in os.walk(directorio):
  num=0
  for file in files[2]:
   if file.find(".txt")>0:
    num+=1
 print("Fin contarCorreo")
 return num

def body(ms):
 print("body")
 charset = ms.get_content_charset()
 if ms.is_multipart():
  for payload in ms.get_payload():
   ctype=payload.get_content_type()
   cdispo = str(payload.get('Content-Disposition'))
   if payload.is_multipart():
    return body(payload)
    break
   elif ctype=='text/plain' and 'attachment' not in cdispo:
    charset = payload.get_content_charset()
    aux = payload.get_payload(decode=True)
    print("Fin body")
    return aux.decode(charset)
 else:
  aux = ms.get_payload(decode=True)
  print("Fin body")
  return aux.decode(charset)
\end{lstlisting}


\backmatter
\addcontentsline{toc}{chapter}{Bibliograf\'ia}
\bibliographystyle{abbrv} %{plain}
\bibliography{bibliografia}
\end{document}

