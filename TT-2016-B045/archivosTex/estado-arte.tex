%\chapter{Introducci\'on} % (fold)
%\label{cha:introduccion}
%\addcontentsline{toc}{chapter}{Estado de arte}




\begin{tabular}{ |p{8cm}|p{8cm}| }
\hline
{ \textbf{TahoeFS}}  & {\textbf{ABS: The Apportioned Backup System} } \\
\hline
{Firmas Digitales}  & {Firmas Digitales} \\
\hline
{SHA256}  & {SHA256}  \\
\hline
{AES-128}  & {Criptografía asimétrica}  \\
\hline
{Modo de operación CTR }  & {Codificado en c++} \\
\hline
{Cifrado convergente}  & {Cifrado convergente}  \\
\hline
{Árbol Merkle}  & {Huella digital} \\
\hline
{RSA de 2048 bits (256 bytes) }  & { rsync  } \\
\hline
\end{tabular}
\\
\begin{itemize}
%\item DupLESS\\
%Este protocolo usa un servicio de almacenamiento en la nube, además implementa una interfaz sencilla con operaciones como guardar, recuperar o borrar un archivo. Es más adecuado para aplicaciones backup y busca proteger la confidencialidad de datos de los clientes, para ello usa seguridad semántica. Además promete capacidad de resistencia ante fallos, protección contra un servidor malintencionado, evitar duplicación de archivos y compatibilidad con diferentes sistemas operativos~\cite{Bellare}.
\item TahoeFS\\
Tahoe es un sistema para el almacenamiento seguro distribuido. Usa las funciones de control de acceso, criptografía, confidencialidad, integridad y eliminación para tolerancia a fallos. Se ha desplegado en un servicio de copia de seguridad comercial y es actualmente operacional. La aplicación es de código abierto ~\cite{tahoe}.
%\item Flud Backup\\
%El proyecto Flud Backup está actualmente inactivo, sin embargo se crearon diversas versiones para Ubuntu y Fedora donde usan paquetes distribuidos y un sistema de confianza. Prometen que los datos que se copian deben ser indestructibles y copias de seguridad descentralizadas~\cite{flud}.
\item ABS\\
Proporciona un recurso fiable de copia de seguridad de colaboración, aprovechando estos recursos independientes distribuidos. Con ABS, la adquisición y mantenimiento de hardware especializado de copia de seguridad es innecesaria. ABS hace un uso eficiente de los recursos de red y almacenamiento de información mediante el uso de técnicas, como cifrado convergente, almacenamiento, procesos de verificación y control de versiones eficientes de codificación ~\cite{abs}.
%Promete el almacenamiento de datos seguro y eficiente, privacidad y seguridad, esto a través de tablas hash distribuidas, firmas de clave privadas, control de versiones y cifrado convergente. 
%Además es tolerante a fallas catastróficas a nodos y permite unir nodos y restaurar operaciones sin pérdida de datos
\end{itemize}

