A lo largo del desarrollo del protocolo criptográfico, se fue logrando el cumplimiento de los objetivos propuestos en Trabajo Terminal 1. Todos los requerimientos fueron satisfactorios en cuanto a la funcionalidad esperada para el proyecto. \\
Sin embargo, surgieron nuevos requerimientos que pueden mejorar la calidad de la experiencia del usuario en cuanto a la usabilidad de la plataforma. Estos nuevos requerimientos los clasificamos como el trabajo que se realizará en el futuro, es decir, actualizaciones que se realizarán después de haber generado la versión final del protocolo. \\
Estos nuevos requerimientos son: \\
\begin{itemize}
    \item Subir la plataforma a un portal en internet.  Ya que al estar en la red el alcance de accesibilidad de nuestros usuarios sería más amplio, permitiendo que cualquier persona desde un lugar con acceso a internet, podría hacer usos de la plataforma, e inclusive llegar a ser utilizado en un ámbito laboral, escolar o profesional.
    \item Modificaciones para que el protocolo pueda Subir, Descargar y eliminar varios archivos a la vez. Ya que por ahora debido a los alcances establecidos, el usuario que está haciendo uso del protocolo sólo puede subir, descargar o eliminar un archivo a la vez. Esto puede llegar a ser un tanto complicado para usuarios que desean trabajar en más de un archivo.
    \item Tener acceso a Nubes comerciales como Drive, Dropbox, etc. Esto debido a que el protocolo es un proyecto potencialmente comercial y puede llegar a venderse dentro de la industria tecnológica, una compatibiidad con nubes ya reconocidas en el mercado, podría ser más fácil de adaptar a cualquier ámbito o área de trabajo.\\
\end{itemize}

Estos requerimientos son los que durante el desarrollo del protocolo se lograron identificar, y para poder llegar a cumplirlos es necesario poder trabajar en ellos durante un tiempo que el equipo considere necesario para cumplirlo, realizando un análisis de éstos como se realizo desde la propuesta al inicio del protocolo. 