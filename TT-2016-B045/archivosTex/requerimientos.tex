\chapter{Requerimientos del sistema }

\section{Requerimientos Funcionales. }

\begin{table}[htb]
\centering
\begin{tabular}{| p{2.5cm} |  p{13.5cm} |}
\hline
\multicolumn{2}{|c|}{Servidor de Llaves} \\ \hline
ID & Descripción \\
\hline \hline
RF – SLL1 & El sistema permitirá gestionar estados para la administración de llaves de usuario a través de una clave secreta (Ks) propia del servidor de llaves. \\ \hline
RF – SLL2 & El sistema debe gestionar 3 estados en el servidor de llaves: Generación, Cambios y Eliminación de llaves de usuarios para manipular un archivo. \\ \hline
RF – SLL3 & El sistema comenzará la creación de una llave (K) cuando el usuario solicite la carga de un archivo a almacenar. \\ \hline
\end{tabular}
\caption{Requerimientos funcionales del servidor de llaves}
\label{Servidor de Llaves }
\end{table}