\chapter{Preliminares} % (fold)

\section{Criptografía}
Es la ciencia que trata las escrituras ocultas, está comprendida por la Criptografía, el Criptoanálisis y la Esteganografía
La Criptografía es la ciencia que se encarga del estudio de técnicas para transformar la información a una forma que no pueda entenderse a simple vista; sin embargo, el objetivo de la Criptografía no es sólo mantener los datos secretos, sino también protegerlos contra modificación y comprobar la fuente de los mismos.

\subsection{Criptoanálisis}
Es la ciencia que se ocupa del análisis de un texto cifrado para obtener la información original sin conocimiento de la clave secreta, esto es, de forma ilícita rompiendo así los procedimientos de cifrado establecidos por la Criptografía, por lo que se dice que Criptoanálisis y Criptografía son ciencias complementarias pero contrarias.
El criptoanálisis es el arte de descifrar comunicaciones encriptadas sin conocer las llaves correctas

\subsection{Esteganografía}
 Por su parte, estudia la forma de ocultar la existencia de un mensaje. Esta ciencia consiste en esconder en el interior de un mensaje, otro mensaje secreto, el cual sólo podrá ser entendido por el emisor y el receptor y pasará inadvertido para todos los demás.

\subsection{Servicios de Seguridad}
Los servicios de seguridad, son aquellos que garantizan en un sistema de información la adquisición, almacenamiento, procesamiento y transmisión de la información y para lograrlo se valen de uno o más mecanismos de seguridad.

\subsection{CONFIDENCIALIDAD}
Este servicio asegura que sólo las personas o procesos autorizados tengan acceso a la información. Con ello se busca que un agente no autorizado no pueda leer, copiar o modificar la información.
El servicio de confidencialidad se puede diferenciar en dos tipos:
\begin{itemize}
	\item Servicio de confidencialidad de contenido: se busca proteger el contenido de un recurso del sistema, para ello se cifra la información para que en caso ser interceptada por alguien no autorizado, no pueda ser descubierta. Este servicio puede proporcionar protección a todos los datos transmitidos por un usuario durante una conexión o puede proteger sólo parte de ellos por ejemplo sólo a los mensajes con información importante o incluso se pueden proteger sólo algunos campos de un determinado mensaje.
	\item Servicio de confidencialidad del mensaje: busca ocultar el flujo de un mensaje durante una conexión, para ello se cifra y se utiliza una técnica de envoltura con el objetivo de que si un atacante está realizando un análisis de tráfico, no pueda descubrir por ejemplo quien está enviando la información ni quien la recibe ni la frecuencia con la que se envían los mensajes.
\end{itemize}

\subsection{AUTENTICACIÓN}
Este servicio verifica la identidad de un agente que pretende acceder a la información. En una conexión entre dos entidades, el servicio verifica que las entidades sean quienes dicen ser, además de asegurar que un tercer individuo no pueda hacerse pasar por alguna de las entidades autorizadas y realizar una transmisión o recepción de datos.

\subsection{INTEGRIDAD}
Este servicio asegura que el contenido de los datos no ha sido modificado y que la secuencia de los mismos se ha mantenido a lo largo de toda la transmisión, con ello se evita una réplica o un reordenamiento del mensaje por parte de un atacante.
Al igual que el servicio de confidencialidad, la integridad puede aplicarse a todos los datos transmitidos por un usuario durante una conexión, sólo a parte de ellos o sólo a algunos campos dentro del mensaje.
Cuando se tiene un ataque a la integridad de los datos, el sistema puede o no reportar dicha violación, por lo que se puede distinguir entre servicio de integridad con recuperación y servicio de integridad sin recuperación.
El servicio de integridad también se puede diferenciar entre servicio de integridad del contenido y servicio de integridad de la secuencia del mensaje
\begin{itemize}
	\item Servicio de integridad del contenido: proporciona pruebas de que el contenido no ha sido alterado o modificado.
	\item Servicio de integridad de la secuencia del mensaje: proporciona pruebas de que el orden de una secuencia de mensajes ha sido mantenida durante la transmisión.
\end{itemize}

\subsection{NO REPUDIO}
Este servicio evita que las entidades en una conexión nieguen haber transmitido o recibido un mensaje.
Existen varios tipos de este servicio y cada uno de ellos proporciona pruebas de haberse llevado a cabo:
\begin{itemize}
	\item No repudio de origen: con este servicio, el emisor de un mensaje no puede negar haber sido él quien transmitió dicho mensaje.
	\item No repudio de envío: comprueba que los datos fueron enviados.
	\item No repudio de presentación: protege contra cualquier intento falso de negar que los datos fueron presentados para el envío.
	\item No repudio de transporte: protege contra cualquier intento de negar que los datos fueron transportados.
	\item No repudio de recepción: con este servicio, el receptor de un mensaje no puede negar haber recibido un mensaje.
\end{itemize}

\subsection{CONTROL DE ACCESO}
El servicio de control de acceso es utilizado con el fin de restringir el acceso a los medios de almacenamiento de la información. Este servicio está muy relacionado con el de autenticación ya que cualquier agente que quiera tener acceso a algún recurso del sistema primero deberá identificarse para que le sea permitido el acceso a dicha información y de acuerdo a los permisos o privilegios que tenga podrá manipularla.

\subsection{DISPONIBILIDAD}
El servicio de disponibilidad asegura que los agentes autorizados tengan acceso a la información en el momento en que ellos lo requieran y tantas veces como lo soliciten sin importar si ésta es correcta o no.

\subsection{Ataques}
Un ataque es una violación a la seguridad de la información realizada por intrusos que tienen acceso físico al sistema sin ningún tipo de restricción, su objetivo es robar la información o hacer que ésta pierda valor relativo, o que disminuyan las posibilidades de su supervivencia a largo plazo.
Un intruso puede obtener información como:
\begin{itemize}
	\item Bloques de direcciones IP
	\item Localización de sistemas críticos (DNSs, WINS, DHCPs, Servidores de correo, etc.)
	\item Puntos de acceso para números telefónicos y VPNs
	\item Información personal de los trabajadores de la organización
	\item Organizaciones asociadas, subsidiarias, etc.
\end{itemize}
Existen dos tipos de ataques que amenazan las comunicaciones secretas:
\begin{itemize}
	\item Pasivo: es aquel en el cual el intruso sólo busca obtener la información y al hacerlo no la modifica, por lo que es difícil percatarse de que se está siendo atacado.
	\item Activo: el intruso además de obtener la información la modifica de tal modo que sirva a sus intereses y al ser modificada es más fácil percatarse de que se está siendo atacado.
Los ataques activos se dividen en dos tipos: Ataques a los métodos de cifrado y Ataques a los protocolos criptográficos.
\end{itemize}

\subsection{INTEGRIDAD}

\subsection{INTEGRIDAD}










\subsection{Criptografía Simétrica}
La criptografía simétrica utiliza la misma clave para cifrar y descifrar el mensaje de datos, es decir se basa en un secreto compartido~\cite{criptosimetrica}. \\ Características de la Criptografía simétrica: \begin{itemize}
	\item La clave es la misma para cifrar que para descifrar un mensaje, por lo que sólo el emisor y el receptor deben conocerla.
	\item Se basan en operaciones matemáticas sencillas, por ello son fácilmente implementados en hardware.
	\item Debido a su simplicidad matemática son capaces de cifrar grandes cantidades de datos en poco tiempo.
			       \end{itemize} ~\cite{sime} \\
Los algoritmos criptográficos simétricos tienen dos versiones: cifrador en bloque y cifrador en flujo. Una cifra es una palabra para describir un algoritmo de cifrado. El beneficio del uso de un algoritmo simétrico radica en el procesamiento rápido
para encriptar y desencriptar un alto volumen de datos. El cifrado simétrico es una eficaz táctica de almacenamiento de información
sensible en una base de datos, un registro o archivo ~\cite{sime} El cifrado simétrico puede ser representado con el siguiente diagrama ~\ref{fig:1-2-1}.

\begin{figure}[H]
\centering
	\includegraphics[width=10cm, height=5cm]{./images/Cripto_Simetrica.jpg}
	\caption{Diagrama Criptografía Simétrica}
	\label{fig:1-2-1}
\end{figure}
La sintaxis de un esquema de cifrado sim\'etrico, esta dada por la siguiente definic\'on.
\begin{definition} 
Un esquema de cifrado sim\'etrico est\'a conformado por una tripleta de algoritmos 
$\sf \Pi=(Gen, Enc, Dec)$, definidos como se describe a continuaci\'on:
\begin{itemize}
\item  El algoritmo generador de claves $\sf Gen$ selecciona una llave  $K$ al azar del conjunto de llaves $\cal K$, esto se denotar\'a como $K \rand {\cal K}$.
Esta clave $K$  ser\'a usada por los algoritmos  $\sf Enc$ y $\sf Dec$, esta clave la compartir\'an  emisor y receptor. 
\item El algoritmo de cifrado $\sf Enc$, toma como entrada un texto en claro  $M \in {\cal M}$ y una clave $K$ generada por  $\sf Gen$  y regresa un texto cifrado $C \in {\cal C}$.  Usualmente esto se denota como $C \leftarrow {\sf Enc}_K(M)$.
 \item El algoritmo de descifrado $\sf Dec$, toma como entrada un texto cifrado $C$ y una llave $K$ y regresa $M$. Esta operaci\'on se denota por  $M \leftarrow {\sf Dec}_K(C)$.
Para que cualquier algoritmo de cifrado sim\'etrico funcione correctamente, se debe garantizar que para
todas las llaves posibles en  $\cal K$ y todos los posibles mensajes $\cal M$, $$ {\sf Dec}_K({\sf Enc}_K(M)) = M.$$
\end{itemize}
\end{definition}

\subsection{Criptografía asimétrica}
La criptografía asimétrica es por definición aquella que utiliza dos claves diferentes para cada usuario, una para cifrar que se le llama clave pública y otra para descifrar que es la clave privada. Los algoritmos asimétricos son diferentes a los simétricos en un sentido muy importante ~\cite{sime}. Cuando se genera una clave simétrica, simplemente se escoge un número aleatorio de la longitud apropiada. Al generar claves asimétricas el proceso es más complejo. Los algoritmos asimétricos se llaman asimétricos porque en lugar de usar una sola clave para realizar la codificación y la decodificación, se utilizan dos claves diferentes: una para cifrar y otra para descifrar. Estas dos claves se encuentran asociadas matemáticamente, cuya característica fundamental es que una clave no puede descifrar lo que cifra. ~\cite{sime}.

\bigskip Características de la Criptografía simétrica: \begin{itemize}
	\item Se utiliza una clave para cifrar y otra para descifrar. El emisor emplea la clave pública del receptor para cifrar el mensaje, 	éste último lo descifra con su clave privada.
	\item Se basan en operaciones matemáticas complejas.
	\item Se ejecutan de 100 a 1000 veces más lento que los algoritmos simétricos.
\end{itemize} ~\cite{sime} \\ \\ 

 Los beneficios de la criptografía asimétrica son la solución a los problemas de la criptografía simétrica, pues las claves públicas pueden ser distribuidas con toda tranquilidad, no valen de nada sin las claves privadas. El cifrado asimétrico se le emplea muy frecuente para pasar con seguridad una clave privada, que posteriormente, será la que se utilice para cifrar y/o descifrar otra información. El cifrado asimétrico puede ser representado con el siguiente diagrama ~\ref{fig:2-2-1}.

\begin{figure}[H]
\centering
	\includegraphics[width=10cm, height=5cm]{./images/Cripto_Asimetrica.jpg}
	\caption{Diagrama Criptografía Asimétrica}
	\label{fig:1-2-1}
\end{figure}

\subsection{Cifrado por bloques}
Los algoritmos de cifrado por bloques toman bloques de tamaño fijo del texto en claro y producen un bloque de tamaño fijo de texto cifrado, generalmente del mismo tamaño que la entrada. El tamaño del bloque debe ser lo suficientemente grande como para evitar ataques de texto cifrado. La asignación de bloques de entrada a bloques de salida debe ser uno a uno para hacer el proceso reversible y parecer aleatoria.\\ 
Para la asignación de bloques los algoritmos de cifrado simétrico realizan sustituciones y permutaciones en el texto en claro hasta obtener el texto cifrado.\\ 
La sustitución es el reemplazo de un valor de entrada por otro de los posibles valores de salida, en general, si usamos un tamaño de bloque k, el bloque de entrada puede ser sustituido por cualquiera de los 2k bloques posibles.
La permutación es un tipo especial de sustitución en el que los bits de un bloque de entrada son reordenados para producir el bloque cifrado, de este modo se preservan las estadísticas del bloque de entrada (el número de unos y ceros). \\ \\  Los algoritmos de cifrado por bloques iterativos funcionan aplicando en sucesivas rotaciones una transformación
(función de rotación) a un bloque de texto en claro. La misma función es aplicada a los datos usando una subclave
obtenida de la clave secreta proporcionada por el usuario. El número de rotaciones en un algoritmo de cifrado por
bloques iterativo depende del nivel de seguridad deseado.


La sustitución es el reemplazo de un bloque de $n$ bits por otro bloque de $n$ bits en un espacio de 
$2^{k}$~\cite{bloc}. Los cifradores por bloques mas usados son AES (Advanced Encryption Standard, por sus 
siglas en ingl\'es) y DES (Data Encryption Standard, por sus siglas en ingl\'es).



\subsection{Modos de operación}
Los modos de operaci\'on fueron desarrollados para el algoritmo DES, estos fueron estandarizados en Diciembre de 1980. 
Cuando la informaci\'on es cifrada usando la misma clave, surge una serie de problemas de seguridad. En esencia los modos de operaci\'on son una técnica para mejorar el efecto criptogr\'afico de los 
cifradores por bloques~\cite{modes}.\\\\
\textit{ECB}(Electronic codebook): Este modo de operación es probablemente el más simple de todos, el texto plano M está segmentado como $ M=M_1||M_2||...||M_m$ donde cada $M_i$ es un bloque de n bits. A continuación la funci\'on de cifrado $E_k$ se aplica por separado a cada bloque $M_i$. \\
A continuación tenemos el diagrama de este modo de cifrado.\\
\begin{figure}[h]
    \centering
    \begin{subfigure}[t]{0.5\textwidth}
        \centering
        \includegraphics[height=1.7in]{./images/ecb1.png}
        \caption{Diagrama ECB Cifrado}
        \label{fig:1-3-1}
    \end{subfigure}%
    ~ 
    \begin{subfigure}[t]{0.5\textwidth}
        \centering
        \includegraphics[height=1.7in]{./images/ECB2.png}
        \caption{Diagrama ECB Descifrado}
        \label{fig:1-3-1}
    \end{subfigure}
    \label{fig:protocol}
\end{figure}


\textit{CBC}(Cipher-block chaining): Para este modo de operación la salida de un bloque de cifrado se introduce en el siguiente bloque de cifrado junto con el siguiente bloque del mensaje.\\

\begin{figure}[h]
    \centering
    \begin{subfigure}[t]{0.5\textwidth}
        \centering
        \includegraphics[height=1.7in]{./images/cbc1.png}
        \caption{Diagrama CBC Cifrado}
        \label{fig:1-4-1}
    \end{subfigure}%
    ~ 
    \begin{subfigure}[t]{0.5\textwidth}
        \centering
        \includegraphics[height=1.7in]{./images/CBC2.png}
        \caption{Diagrama CBC Descifrado}
        \label{fig:1-4-1}
    \end{subfigure}
    \label{fig:protocol}
\end{figure}

 \begin{figure}[H]
 \centering
	\includegraphics[width=14cm, height=6cm]{./images/pcbc.png}
	
\end{figure}

CBC toma como bloques de mensajes de entrada M y un vector de inicialización (IV). Durante el cifrado, la salida del i-ésimo bloque depende de los i-1 bloques anteriores. Así, el cifrado CBC es intrínsecamente secuencial.\\

\textit{CFB}(Cipher Feedback): En este modo de operación, los bloques de cifrado también están encadenados pero a la salida se produce de una manera muy diferente de la de CBC. Cada bloque de salida se le aplica XOR con el siguiente bloque de entrada.\\

\begin{figure}[h]
    \centering
    \begin{subfigure}[t]{0.5\textwidth}
        \centering
        \includegraphics[height=1.7in]{./images/cfb1.png}
		\caption{Diagrama CFB Cifrado}
		\label{fig:1-5-1}
    \end{subfigure}%
    ~ 
    \begin{subfigure}[t]{0.5\textwidth}
        \centering
        \includegraphics[height=1.7in]{./images/CFB2.png}
		\caption{Diagrama CFB Descifrado}
		\label{fig:1-5-1}
    \end{subfigure}
    \label{fig:protocol}
\end{figure}

\begin{figure}[H]
\centering
	\includegraphics[width=14cm, height=6cm]{./images/pcfb.png}
	
\end{figure}


\textit{OFB}(Output feedback): En este modo de operación el IV se cifra varias veces para obtener un flujo de bytes aleatorios, el resultado de esto se aplica XOR con el bloque de texto plano mientras que el flujo de bytes aleatorios se usa como parámetro del siguiente bloque. A diferencia de los otros modos en OFB ninguna parte del texto claro entra directamente a cifrarse.

\begin{figure}[h]
    \centering
    \begin{subfigure}[t]{0.5\textwidth}
        \centering
        \includegraphics[height=1.7in]{./images/ofb1.png}
		\caption{Diagrama OFB Cifrado}
		\label{fig:1-6-1}
    \end{subfigure}%
    ~ 
    \begin{subfigure}[t]{0.5\textwidth}
        \centering
        \includegraphics[height=1.7in]{./images/OFB2.png}
		\caption{Diagrama OFB Descifrado}
		\label{fig:1-6-1}
    \end{subfigure}
    \label{fig:protocol}
\end{figure}

\begin{figure}[H]
\centering
	\includegraphics[width=14cm, height=6cm]{./images/pofb.png}
	
\end{figure}
\pagebreak
\subsection{Funciones Hash}
A continuaci\'on se describir\'an las caracter\'isticas de las {\it funciones
hash}, tambi\'en conocidas como {\it funciones de resumen}. Las funciones hash basan
su definici\'on en funciones de un solo sentido  ({\it one-way functions}, en ingl\'es).
Una funci\'on de un s\'olo sentido es aquella que para un valor $x$, es 
muy f\'acil calcular $f(x)$, pero es muy dif\'icil hallar $f^{-1}(x)$. Es 
complicado en general, hallar funciones de \'este tipo y probar que lo 
son.
\begin{definition}
Una funci\'on hash, es una funci\'on de un s\'olo sentidocuya entrada $m$
 es un mensaje de longitud arbitraria
y la salida es una cadena binaria de longitud fija. Al resumen o hash de 
un mensaje $m$, se le denotar\'a como $h(m)$. Una funci\'on hash debe
tener las siguientes propiedades:
\begin{itemize}
\item Para cualquier mensaje $m$, debe ser posible calcular $h(m)$ 
eficientemente. 
\item Dado $h(m)$, debe ser computacionalmente dif\'icil, hallar un mensaje
$m'$, tal que $h(m)=h(m')$.
\item Debe ser computacionalmente dif\'icil, hallar dos mensajes $m$ y $m'$ 
tales que $h(m)=h(m')$.
\end{itemize}
\end{definition}
 
Entre las funciones hash que se usan para criptograf\'ia est\'an: MD2, MD4,
MD5, donde MD significa {\it Message Digest}, y el algoritmo est\'andar al momento de escribir \'estas notas es el {\it Secure Hash Algorithm} por sus siglas
en ingl\'es SHA.
  La MD5 fue dise\~nada por Ron Rivest, toma como entrada un mensaje de 
longitud arbitraria y proporciona como salida una cadena binaria de 128 bits.
El mensaje de entrada se procesa por bloques de 512 bits. 
  La SHA fue dise\~nada por en NIST y se estableci\'o como est\'andar
en 1993. Recibe como entrada un mensaje con longitud menor a $2^{64}$ bits y
como salida se obtiene una cadena binaria de 160 bits. Al igual que el
MD5, se procesa en bloques de 512 bits~\cite{modes}.



%Las funciones hash son usadas para construir una pequeña huella digital de la informaci\'on, si la informaci\'on es alterada tambi\'en la huella digital es alterada. Esta caracter\'istica hace que las funciones hash sean ampliamente usadas para verificar la integridad de datos.\\
%De manera formal, un funci\'on Hash es una Cuadrupla$(X,Y,K,H)$ donde:
%\begin{enumerate}
% \item $X$ es un conjunto de posibles mensajes.
% \item $Y$ es un conjunto finito de posibles resúmenes de mensajes o etiquetas de autenticación.
% \item $K$, el espacio de claves, es un conjunto finito de posibles claves.
% \item Para cada $k\quad \epsilon\quad K$, existe una función hash $h_k\quad \epsilon\quad H$. Parac ada $h_k: X \longrightarrow Y$.
% \cite{stinson}
%\end{enumerate}



