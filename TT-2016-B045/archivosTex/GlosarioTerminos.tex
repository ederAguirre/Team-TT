\section{Glosario de Términos}

\begin{description}
\item [Usuario o Entidad:] Persona que utiliza el servicio de almacenamiento para guardar archivos en la nube.
\item [Archivo: ] Conjunto de datos almacenados en la memoria de una computadora que puede manejarse con una instrucción única.
\item [Nube: ] Espacio de almacenamiento y procesamiento de datos y archivos ubicado en internet, al que puede acceder el usuario desde cualquier dispositivo. 
\item [Privacidad: ] Capacidad que una organización o individuo tiene para determinar qué datos en un sistema informático pueden ser compartidos con terceros.
\item [Seguridad: ] Conjunto de medidas preventivas y reactivas de las organizaciones o sistemas tecnológicos que permiten resguardar y proteger la información, buscando mantener la confidencialidad, disponibilidad e integridad de datos de la misma.
\item [Duplicación: ] Acción y efecto de duplicar.
\item [Duplicar: ] Repetir exactamente algo, hacer una copia de ello.
\item [Cifrar: ] Escribir un mensaje en clave mediante un sistema de signos formado por números, letras, símbolos, etc.
\item [Descifrar: ] Declarar lo que está escrito en cifra o en caracteres desconocidos, sirviéndose de clave dispuesta para ella, o sin clave, por conjeturas y reglas críticas.
\item [Conjetura: ] Juicio que se forma de algo por indicios u observaciones.
\item [Mensaje: ] Información transmitida.
\item [Algoritmo Criptográfico: ] Es una función matemática usada en los procesos de cifrado y descifrado.Trabaja en combinación con una llave para cifrar y descifrar datos. 
Modifica los datos de un documento con el objeto de alcanzar algunas características de seguridad (autenticación, integridad y confidencialidad).
\item [Clave o Llave: ] Una clave es un número de gran tamaño, que una persona puede conceptualizar como un mensaje digital, como un archivo binario o como una cadena de bits o bytes.
\item [Aritmética Modular: ] Es un sistema aritmético para clases de equivalencia de números enteros llamadas clases de congruencia. 
\item [Función Computacional: ] Funciones que pueden ser calculadas por una máquina de Turing.
\item [Opacidad: ] Cualidad de opaco.
\item [Opaco: ] Oscuro.
\end{description}
