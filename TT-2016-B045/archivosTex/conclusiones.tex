
El protocolo criptográfico para evitar las duplicaciones en la nube, es una plataforma tecnológica que proporciona seguridad, integridad en archivos de los usuarios que hoy en día tienen acceso al almacenamiento en la nube. De igual manera, el protocolo proporciona un ahorro en el espacio de memoria en los dispositivos de cómputo de los usuarios así como en el almacenamiento en la nube, ya que el protocolo logra eliminar archivos que tengan el mismo contenido almacenando y guarda sólo una copia de estos. \\
Con el desarrollo de este protocolo, se logró obtener resultados satisfactorios en cuanto a los objetivos planteados dentro de la propuesta de solución, principalmente garantizando una optimización de ahorro en espacio de memoria para los servidores que cubren el servicio de almacenamiento en la nube sin descuidar la seguridad de los usuarios. \\
Luego de analizar, desarrollar y probar a detalle todos los componentes que integran el protocolo, como equipo de trabajo terminal podemos concluir que éste proyecto criptográfico se creó bajo los requerimientos y especificaciones establecidos desde su inicio, integrando en éste la tecnología y componentes tecnológicos que satisfacen los requerimientos, ya que a lo largo de su desarrollo, se compararon diversas opciones que podían cubrir los requerimientos y al final se logró integrar sólo las que mejor se adecuaban al proyecto. 


