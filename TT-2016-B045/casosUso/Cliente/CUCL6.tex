% Copie este bloque por cada caso de uso:
%-------------------- COMIENZA descripción del caso de uso.

\begin{UseCase}{CUCL6}{Iniciar Sesión.}{Permitir el acceso al sistema con su usuario y contraseña correspondientes, el cual es autenticado y autorizado para la utilización del sistema.
	}
	\UCitem{Versión}{1.0 - 19/04/17}
	\UCitem{Autor}{Jhonatan Saulés Cortés.}
	\UCitem{Prioridad}{Alta}
	\UCitem{Módulo}{Cliente}
	\UCitem{Actor}{Cliente}
	\UCitem{Propósito}{Dar acceso al usuario al sistema para poder realizar sus actividades.}
	\UCitem{Entradas}{Nombre de usuario, Contraseña.}
	\UCitem{Salidas}{Pagina principal del usuario que inicio sesión}
	\UCitem{Precondiciones}{Estar registrado en el sistema.}
	\UCitem{Postcondiciones}{}
	\UCitem{Reglas del negocio}{  
	\begin{itemize}
			\item \BRref{RN4}{Usuario registrado} 
	\end{itemize}
	}	
	\UCitem{Mensajes}{
	\begin{itemize}
			\item \MSGref{MSG1}{Operación exitosa}. 
			\item \MSGref{MSG5}{Dato incorrecto}. 
			\item \MSGref{MSG6}{Longitud inválida}. 
			\item \MSGref{MSG9}{Dato requerido}. 
			\item \MSGref{MSG10}{No existe información}. 
    \end{itemize}
	}
\end{UseCase}


% ------------ Trayectoria principal, poner referencias a pantallas del sistema, reglas entre otras cosas que sean necesarias. ------------
\begin{UCtrayectoria}{Principal}
	\UCpaso[\UCactor] Da clic en la opción \textit{Iniciar sesión}.
	\UCpaso Despliega los campos para introducir nombre de usuario y contraseña.
	\UCpaso[\UCactor] Ingresa su nombre de usuario y contraseña en los campos mostrados. \label{CUCL6Regresa}  
	\UCpaso[\UCactor] Da clic en el botón \textit{Ingresar}.
	\UCpaso Autentica y autoriza el nombre usuario y contraseña con base en la regla de negocio \BRref{RN4}{Usuario registrado}.  \Trayref{A}  \Trayref{B}  \Trayref{C}  \Trayref{D}  
	\UCpaso Muestra el mensaje \MSGref{MSG1}{Operación exitosa}. 
	\UCpaso Muestra el menú principal del usuario.
	\UCpaso Fin del caso de uso.
\end{UCtrayectoria}

% ---------------- Trayectorias alternativas -------------- Colocar los mensajes  {\bf MSG1-}

\begin{UCtrayectoriaA}{A}{Datos incorrectos}
	\UCpaso Muestra el mensaje \MSGref{MSG5}{Dato incorrecto}. 
	\UCpaso[\UCactor] Da clic en el botón \textit{Cerrar}.
	\UCpaso[\UCactor] Continua en el paso \ref{CUCL6Regresa} del \UCref{CUCL6}

\end{UCtrayectoriaA} 

\begin{UCtrayectoriaA}{B}{Longitud inválida}
	\UCpaso Muestra el mensaje \MSGref{MSG6}{Longitud inválida}. 
	\UCpaso[\UCactor] Da clic en el botón \textit{Cerrar}.
	\UCpaso[\UCactor] Continua en el paso \ref{CUCL6Regresa} del \UCref{CUCL6}

\end{UCtrayectoriaA}

\begin{UCtrayectoriaA}{C}{Datos requeridos}
	\UCpaso Muestra el mensaje \MSGref{MSG9}{Dato requerido}. 
	\UCpaso[\UCactor] Da clic en el botón \textit{Cerrar}.
	\UCpaso[\UCactor] Continua en el paso \ref{CUCL6Regresa} del \UCref{CUCL6}

\end{UCtrayectoriaA}

\begin{UCtrayectoriaA}{D}{No existe información}
	\UCpaso Muestra el mensaje \MSGref{MSG10}{No existe información}. 
	\UCpaso[\UCactor] Da clic en el botón \textit{Cerrar}.
	\UCpaso[\UCactor] Continua en el paso \ref{CUCL6Regresa} del \UCref{CUCL6}

\end{UCtrayectoriaA}

		
%-------------------------------------- TERMINA descripción del caso de uso.