% Copie este bloque por cada caso de uso:
%-------------------- COMIENZA descripción del caso de uso.

\begin{UseCase}{CUCL7}{Registrar usuario.}{Solicitar los datos importantes de un usuario nuevo, generar sus llaves pública y privada para darlo de alta en el sistema.
	}
	\UCitem{Versión}{1.0 - 19/04/17}
	\UCitem{Autor}{Jhonatan Saulés Cortés.}
	\UCitem{Prioridad}{Alta}
	\UCitem{Módulo}{Cliente}
	\UCitem{Actor}{Cliente}
	\UCitem{Propósito}{Habilitar un nuevo usuario generandole sus llaves privada y pública.}
	\UCitem{Entradas}{Datos del usuario.}
	\UCitem{Salidas}{Llaves del usuario}
	\UCitem{Precondiciones}{}
	\UCitem{Postcondiciones}{}
	\UCitem{Reglas del negocio}{  
	\begin{itemize}
			\item \BRref{RN1}{Datos requeridos}  
			\item \BRref{RN2}{Datos correctos} 
			\item \BRref{RN3}{Unicidad de elementos} 
	\end{itemize}
	}	
	\UCitem{Mensajes}{
	\begin{itemize}
			\item \MSGref{MSG1}{Operación exitosa}. 
			\item \MSGref{MSG4}{Registro repetido} 
			\item \MSGref{MSG5}{Dato incorrecto}. 
			\item \MSGref{MSG6}{Longitud inválida}. 
			\item \MSGref{MSG9}{Dato requerido}. 
    \end{itemize}
	}
\end{UseCase}


% ------------ Trayectoria principal, poner referencias a pantallas del sistema, reglas entre otras cosas que sean necesarias. ------------
\begin{UCtrayectoria}{Principal}
	\UCpaso[\UCactor] Da clic en la opción \textit{Registrarse}.
	\UCpaso Despliega los campos para introducir nombre, primer apellido, segundo apellido, nombre de usuario, contraseña, fecha de nacimiento, CURP.
	\UCpaso[\UCactor] Ingresa sus datos que han sido solicitados. \label{CUCL7Regresa}  
	\UCpaso[\UCactor] Da clic en el botón \textit{Registrar}.
	\UCpaso Verifica los datos proporcionados por el usuario con base en las reglas de negocios \BRref{RN1}{Datos requeridos},  \BRref{RN2}{Datos correctos}, \BRref{RN3}{Unicidad de elementos} .  \Trayref{A}  \Trayref{B}  \Trayref{C}  \Trayref{D}
	\UCpaso Genera su llave privada y pública del usuario con RSA. 
	\UCpaso Almacena sus llaves en el servidor y en el dispositivo actual.
	\UCpaso Muestra el mensaje \MSGref{MSG1}{Operación exitosa}. 
	\UCpaso Muestra el menú principal del usuario.
	\UCpaso Fin del caso de uso.
\end{UCtrayectoria}

% ---------------- Trayectorias alternativas -------------- Colocar los mensajes  {\bf MSG1-}

\begin{UCtrayectoriaA}{A}{Datos incorrectos}
	\UCpaso Muestra el mensaje \MSGref{MSG5}{Dato incorrecto}. 
	\UCpaso[\UCactor] Da clic en el botón \textit{Cerrar}.
	\UCpaso[\UCactor] Continúa en el paso \ref{CUCL7Regresa} del \UCref{CUCL7}

\end{UCtrayectoriaA} 

\begin{UCtrayectoriaA}{B}{Longitud inválida}
	\UCpaso Muestra el mensaje \MSGref{MSG6}{Longitud inválida}. 
	\UCpaso[\UCactor] Da clic en el botón \textit{Cerrar}.
	\UCpaso[\UCactor] Continúa en el paso \ref{CUCL7Regresa} del \UCref{CUCL7}

\end{UCtrayectoriaA}

\begin{UCtrayectoriaA}{C}{Datos requeridos}
	\UCpaso Muestra el mensaje \MSGref{MSG9}{Dato requerido}. 
	\UCpaso[\UCactor] Da clic en el botón \textit{Cerrar}.
	\UCpaso[\UCactor] Continúa en el paso \ref{CUCL7Regresa} del \UCref{CUCL7}

\end{UCtrayectoriaA}

\begin{UCtrayectoriaA}{D}{Registro repetido}
	\UCpaso Muestra el mensaje \MSGref{MSG4}{Registro repetido}  
	\UCpaso[\UCactor] Da clic en el botón \textit{Cerrar}.
	\UCpaso[\UCactor] Continúa en el paso \ref{CUCL7Regresa} del \UCref{CUCL7}

\end{UCtrayectoriaA}


		
%-------------------------------------- TERMINA descripción del caso de uso.