% Copie este bloque por cada caso de uso:
%-------------------- COMIENZA descripción del caso de uso.



\begin{UseCase}{CUCL5}{Generar llave (k) para archivo.}{El cliente creara una llave llamada k con la cual posteriormente podrá cifrar el archivo con esa llave con el fin de poder detectar los archivos duplicados en el servicio de almacenamiento.}

	\UCitem{Versión}{1.0 - 16/04/17}
	\UCitem{Autor}{Jhonatan Saulés Cortés}
	\UCitem{Prioridad}{Alta}
	\UCitem{Módulo}{Cliente}
	\UCitem{Actor}{Cliente}
	\UCitem{Propósito}{Que el cleinte genere la llave k para poder cifrar el archivo.}
	\UCitem{Entradas}{Firma a ciegas (y).}
	\UCitem{Salidas}{Llave k}
	\UCitem{Precondiciones}{}
	\UCitem{Postcondiciones}{}
	\UCitem{Reglas del negocio}{
		%\begin{itemize}	
		%	\item \BRref{RN-E1}{Alumno inscrito} 
		%\end{itemize}
		}
	\UCitem{Mensajes}{
		\begin{itemize}	
			\item \MSGref{MSG-CL1}{Llave generada con éxito}
		\end{itemize}
		}
\end{UseCase}


% ------------ Trayectoria principal, poner referencias a pantallas del sistema, reglas entre otras cosas que sean necesarias. ------------
\begin{UCtrayectoria}{Principal}
	\UCpaso Recibe la firma a ciegas (y) del servidor.
	\UCpaso Genera la llave k utilizadno la firma a ciegas.
	\UCpaso Almacena la llave en una base de datos, junto con el nombre del achivo al que le corresponde.
\end{UCtrayectoria}

% ---------------- Trayectorias alternativas -------------- Colocar los mensajes  {\bf MSG1-}


