% Copie este bloque por cada caso de uso:
%-------------------- COMIENZA descripción del caso de uso.

\begin{UseCase}{CUCL3}{Descargar archivos descifrados.}{El cliente podrá descargar su archivo y descifrarlo.}

\UCitem{Versión}{1.0 - 16/04/17}	
\UCitem{Autor}{Diana Leslie González Olivier}
\UCitem{Prioridad}{Alta}
\UCitem{Módulo}{Cliente}
\UCitem{Actor}{Cliente}
\UCitem{Propósito}{Que el cliente pueda obtener su archivo con texto en claro}
\UCitem{Entradas}{C1 y C2}
\UCitem{Salidas}{Archivo descargado}
\UCitem{Precondiciones}{El archivo debe existir en la Nube}
\UCitem{Postcondiciones}{}
\UCitem{Reglas del negocio}{
%\begin{itemize} 
% \item \BRref{RN-E1}{} 
%\end{itemize}
}
\UCitem{Mensajes}{
\begin{itemize} 
\item \MSGref{MSG1}{Operación exitosa}
\item \MSGref{MSG-CL4}{Archivo inexistente}
\end{itemize}
}
\end{UseCase}


% ------------ Trayectoria principal, poner referencias a pantallas del sistema, reglas entre otras cosas que sean necesarias. ------------
\begin{UCtrayectoria}{Principal}
\UCpaso [\UCactor] Envía a la nube una petición con el nombre del archivo que desea descargar.
\UCpaso Recibe los archivos C1 y C2 asociados al nombre que envió
\UCpaso Descifra C2 con la llave “Ka” del cliente y obtiene un archivo con la llave “K”
\UCpaso Descifra C1 con la llave “K” y obtiene su archivo “F” en claro
\UCpaso Muestra el mensaje \MSGref{MSG1}{Operación exitosa}.
\end{UCtrayectoria}

% ---------------- Trayectorias alternativas -------------- Colocar los mensajes {\bf MSG1-}
\begin{UCtrayectoria}{Trayectoria Alternativa}
\UCpaso [\UCactor] Envía a la nube una petición con el nombre del archivo que desea descargar.
\UCpaso Muestra el mensaje \MSGref{MSG-CL4}{Archivo inexistente}.\end{UCtrayectoria}


