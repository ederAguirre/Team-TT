% Copie este bloque por cada caso de uso:
%-------------------- COMIENZA descripción del caso de uso.



\begin{UseCase}{CUCL1}{Subir archivo}{El usuario seleccionará el archivo que desea que sea almacenado en la nube, este archivo será cifrado y enviado de manera transparente para el usuario, y dependiendo si el archivo se detecta duplicado se enviarán distintos archivos.}

	\UCitem{Versión}{1.0 - 19/04/17}
	\UCitem{Autor}{Jhonatan Saulés Cortés}
	\UCitem{Prioridad}{Alta}
	\UCitem{Módulo}{Cliente}
	\UCitem{Actor}{Usuario}
	\UCitem{Propósito}{Almacenar un archivo en la nube, el cual debe estar cifrado para que no lo pueda entender el servicio de almacenamiento.}
	\UCitem{Entradas}{ 
                                           \begin{itemize}
					\item Archivo a almacenar \textit{M}
					\item Llave pública del servidor \textit{d}
				\end{itemize}

                                          }
	\UCitem{Salidas}{Archivo X a firmar por el servidor de llaves}
				
	\UCitem{Precondiciones}{El servidor de llaves debe tener asignada tanto su llave pública como llave privada. }
	\UCitem{Postcondiciones}{El archivo del usuario estará listo para ser firmado por el servidor de llaves. }
	\UCitem{Reglas del negocio}{
		%\begin{itemize}	
		%	\item \BRref{RN-E1}{Alumno inscrito} 
		%	\item \BRref{RN-E4}{Calificación} 
		%	\item \BRref{RN-E6}{Calificación registrada}
		%\end{itemize}
		}
	\UCitem{Mensajes}{
		\begin{itemize}	
			\item \MSGref{MSG-CL1}{Carpeta vacía} 
			\item \MSGref{MSG-CL2}{Archivo incomplatible}
			\item \MSGref{MSG-CL3}{Número incorrecto}
			\item \MSGref{MSG-CL4}{Error al generar la llave}.
			\item \MSGref{MSG-CL5}{Archivo almacenado}.
		\end{itemize}
		}
\end{UseCase}


% ------------ Trayectoria principal, poner referencias a pantallas del sistema, reglas entre otras cosas que sean necesarias. ------------
\begin{UCtrayectoria}{Principal}
	\UCpaso[\UCactor] Da un clic en la opción $Subir Archivo$ en la pantalla .
	\UCpaso Despliega una ventana con la carpeta personal que muestra los archivos del usuario. \Trayref{A}
	\UCpaso[\UCactor] Selecciona el archivo \textit{(M)} que va a subir y da un clic en el botón \IUbuttonAceptar en la pantalla . \Trayref{B}
	\UCpaso Elige un número aleatorio \textit{r} dentro del campo de números primos de igual o menor tañano a las llaves del servidor de llaves. \Trayref{C} \label{CUCL2Regreso}
	\UCpaso Calcula una función hash del archivo seleccionado \textit{H(M)}.
	\UCpaso Eleva el número aleatorio \textit{r} a la potencia llave pública del servidor de llaves,  \textit{$r^{e}$}. 
	\UCpaso Multiplica \textit{H(M)} por \textit{$r^{e}$}, \textbf{\textit{H(M)} $\cdot$ \textit{$r^{e}$}}.
	\UCpaso Obtiene Archivo X a firmar y lo envía al servidor de llaves.
	\UCpaso Recibe la firma a ciegas \textit{Y} del servidor.
	\UCpaso Calcula el inverso multiplikcativo del numero aleatorio \textit{r}.
	\UCpaso Multiplica \textit{Y} por \textit{$r^{-1}$}, \textbf{\textit{Y} $\cdot$ \textit{$r^{-1}$}}.
	\UCpaso Verifica que \textit{$k^{e}$} sea igual a \textit{H(M)}.  \Trayref{D}
	\UCpaso Obtiene llave \textit{k} y la almacena, junto con el nombre del achivo al que le corresponde.
	\UCpaso Cifra el archivo \textit{(M)} con su llave \textit{k}.
	\UCpaso Obtiene el archivo \textit{C1}.
	\UCpaso Cifra el archivo \textit{k} con su llave publica del usuario \textit{ka}.
	\UCpaso Obtiene el archivo \textit{C2}.
	\UCpaso Envia a la nube el hash del archivo \textit{C1}, \textit{H(C1)}.
	\UCpaso Recibe solicitud de archivos. \Trayref{E}
	\UCpaso Envía los archivos \textit{C1} y \textit{C2}.
	\UCpaso Muestra el Mensaje \MSGref{MSG-CL5}{Archivo almacenado}. \label{CUCL2Regreso2}
	

\end{UCtrayectoria}

% ---------------- Trayectorias alternativas -------------- Colocar los mensajes  {\bf MSG1-}

		
\begin{UCtrayectoriaA}{A}{Archivos inexistentes}
	\UCpaso Despliega una ventana con la carpeta personal del usuario sin archivos existentes.
	\UCpaso Muestra el Mensaje \MSGref{MSG-CL1}{Carpeta vacía}.
	\UCpaso Termina el caso de uso.
\end{UCtrayectoriaA}

\begin{UCtrayectoriaA}{B}{Archivo incompatible}
	\UCpaso [\UCactor] Selecciona el archivo \textit{(M)} en un formato incompatible para el protocolo y su almacenamiento
	\UCpaso Muestra el Mensaje \MSGref{MSG-CL2}{Archivo incomplatible}.
	\UCpaso Termina el caso de uso.
\end{UCtrayectoriaA}

\begin{UCtrayectoriaA}{C}{Número aleatorio incorrecto}
	\UCpaso Elegir un número aleatorio no primo o mayor al tamaño de las llaves del servidor de llaves.
	\UCpaso Muestra el Mensaje \MSGref{MSG-CL3}{Número incorrecto}.
	\UCpaso Continúa en el paso \ref{CUCL2Regreso} del \UCref{CUCL2}.
\end{UCtrayectoriaA}

\begin{UCtrayectoriaA}{D}{Comparacion es diferente}
	\UCpaso Detecta que \textit{$k^{e}$} y \textit{H(M)} son diferentes.
	\UCpaso Muestra el Mensaje \MSGref{MSG-CL4}{Error al generar la llave}.
	\UCpaso Continúa en el paso \ref{CUCL2Regreso} del \UCref{CUCL2}.
\end{UCtrayectoriaA}

\begin{UCtrayectoriaA}{E}{Archivo duplicado}
	\UCpaso Detecta que el archivo \textit{H(C1)} ya ha sido almacenado.
	\UCpaso Envía el archivo \textit{C2}.
	\UCpaso Continúa en el paso \ref{CUCL2Regreso2} del \UCref{CUCL2}.
\end{UCtrayectoriaA}
