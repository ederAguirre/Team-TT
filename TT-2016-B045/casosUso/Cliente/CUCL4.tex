% Copie este bloque por cada caso de uso:
%-------------------- COMIENZA descripción del caso de uso.

\begin{UseCase}{CUCL4}{Eliminar archivos cifrado.}{El cliente podrá elegir la opción de eliminar un archivo cifrado del servicio de almacenamiento en la nube.}

	\UCitem{Versión}{1.0 - 16/04/17}	
	\UCitem{Autor}{Diana Leslie González Olivier}
	\UCitem{Prioridad}{Alta}
	\UCitem{Módulo}{Cliente}
	\UCitem{Actor}{Cliente}
	\UCitem{Propósito}{Que el cliente pueda eliminar un archivo}
	\UCitem{Entradas}{}
	\UCitem{Salidas}{Archivo eliminado}
	\UCitem{Precondiciones}{El archivo debe existir en la Nube}
	\UCitem{Postcondiciones}{}
	\UCitem{Reglas del negocio}{
		%\begin{itemize} 
			% \item \BRref{RN-E1}{} 
		%\end{itemize}
	}
	\UCitem{Mensajes}{
\begin{itemize} 
	\item \MSGref{MSG-CL1}{Archivo Eliminado}
\end{itemize}
	}
\end{UseCase}


% ------------ Trayectoria principal, poner referencias a pantallas del sistema, reglas entre otras cosas que sean necesarias. ------------
\begin{UCtrayectoria}{Principal}
	\UCpaso [\UCactor] El cliente da clic en el botón eliminar archivo.
	\UCpaso  El sistema despliega la pantalla para seleccionar el archivo que se desea eliminar.
	\UCpaso [\UCactor] El cliente selecciona el archivo que desea eliminar.
	\UCpaso  El sistema recibe petición para eliminar archivo.
	\UCpaso  El sistema busca el nombre de archivo asociado en su base de datos.\Trayref{A}
	\UCpaso  El sistema elimina C1 y C2.
	\UCpaso  El sistema despliega la lista de usuarios en la base de datos.
	\UCpaso Muestra el mensaje \MSGref{MSG1}{El archivo fue eliminado correctamente}.\end{UCtrayectoria}

% ---------------- Trayectorias alternativas -------------- Colocar los mensajes {\bf MSG1-}
\begin{UCtrayectoriaA}{A}
	{El archivo no existe}
	\UCpaso [\UCactor] El cliente da clic en el botón aceptar.
	\UCpaso El sistema despliega la pantalla principal.
\end{UCtrayectoriaA}
