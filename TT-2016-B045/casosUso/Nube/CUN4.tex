% Copie este bloque por cada caso de uso:
%-------------------- COMIENZA descripción del caso de uso.



\begin{UseCase}{CUN4}{Descargar archivo cifrado}{Descargar un archivo cifrado del servicio de almacenamiento (Nube) junto con la llave secreta del usuario que este tiene almacenada en la nube}

	\UCitem{Versión}{1.0 - 15/04/17}
	\UCitem{Autor}{Eder Jonathan Aguirre Cruz}
	\UCitem{Prioridad}{Media}
	\UCitem{Módulo}{Servidor de Llaves}
	\UCitem{Actor}{Usuario}
	\UCitem{Propósito}{Entregar al usuario archivos que desea obtener para un uso posterior. }
	\UCitem{Entradas}{	\begin{itemize}			
						\item Archivo cifrado \textit{C1}.
						\item  Llave secreta cifrada \textit{C2}. 
						\item  Función hash del archivo cifrado.
					\end{itemize}}
	\UCitem{Salidas}{ Lista de archivos del usuario actualizada. }
	\UCitem{Precondiciones}{
					\begin{itemize}			
						\item El servicio de almacenamiento debe estar disponible.
						\item El archivo a almacenar debe estar cifrado bajo un algoritmo criptográfico
					\end{itemize}}
	\UCitem{Postcondiciones}{\begin{itemize}			
						\item El archivo quedará almacenado en el servicio de almacenamiento.
						\item El usuario tendrá actualizada su lista de archivos en la nube. 
					\end{itemize}}
	\UCitem{Reglas del negocio}{
		%\begin{itemize}	
		%	\item \BRref{RN-E1}{Alumno inscrito} 
		%	\item \BRref{RN-E4}{Calificación} 
		%	\item \BRref{RN-E6}{Calificación registrada}
		%\end{itemize}
		}
	\UCitem{Mensajes}{
		\begin{itemize}	
			\item \MSGref{MSG-SLL1}{Generación de llaves}
		\end{itemize}
		}
\end{UseCase}


% ------------ Trayectoria principal, poner referencias a pantallas del sistema, reglas entre otras cosas que sean necesarias. ------------
\begin{UCtrayectoria}{Principal}
	\UCpaso [\UCactor] Envía la función hash del archivo \textit{H(C1)}.
	\UCpaso Recibe la función hash del archivo \textit{H(C1)} a almacenar y corrobora la inexistencia de esta. \Trayref{A}
	\UCpaso Solicita los archivos cifrados a almacenar \textit{C1} y \textit{C2}.
	\UCpaso [\UCactor] Selecciona de su carpeta personal los archivos \textit{C1} y \textit{C2} que va almacenar en la nube.
	\UCpaso Almacena los archivos \textit{C1} y \textit{C2} en la nube. 
	\UCpaso Asocia la función hash \textit{H(C1)} al usuario con el archivo \textit{C1} y \textit{C2}. 
	\UCpaso Actualiza la lista de usuarios y archivos almacenados en la nube. \label{CUN3Regreso}
	\UCpaso Muestra el mensaje \MSGref{MSG1}{Operación exitosa}.	
\end{UCtrayectoria}

% ---------------- Trayectorias alternativas -------------- Colocar los mensajes  {\bf MSG1-}

		
\begin{UCtrayectoriaA}{A}{Archivo existente}
	\UCpaso Detecta la existencia de la función hash \textit{H(C1)} almacenada en la nube. 
	\UCpaso Solicita el archivo cifrado a almacenar \textit{C2}.
	\UCpaso [\UCactor] Selecciona de su carpeta personal el archivos \textit{C2} que va almacenar en la nube.
	\UCpaso Almacena el archivo \textit{C2} en la nube. 
	\UCpaso Asocia la función hash \textit{H(C1)} al usuario con el archivo \textit{C2}. 
	\UCpaso Continua en el paso \ref{CUN3Regreso} del \UCref{CUN3}.
\end{UCtrayectoriaA}

