% Copie este bloque por cada caso de uso:
%-------------------- COMIENZA descripción del caso de uso.



\begin{UseCase}{CUSLL2}{Generar firma ciega (y). }{El servidor realizará una firma a ciegas de un archivo solicitado, este archivo ha sido oculto para que el servidor no sepa de donde proviene o que contiene, esta firma servirá para la generación de una llave para cifrar el archivo solicitado.}

	\UCitem{Versión}{1.0 - 16/04/17}
	\UCitem{Autor}{Diana Leslie González Olivier}
	\UCitem{Prioridad}{Alta}
	\UCitem{Módulo}{Servidor de Llaves}
	\UCitem{Actor}{Servidor}
	\UCitem{Propósito}{Que el servidor firme el archivo solicitado sin saber a que cliente corresponde.}
	\UCitem{Entradas}{Archivo oculto $x$}
	\UCitem{Salidas}{Firma a ciegas $y$}
	\UCitem{Precondiciones}{}
	\UCitem{Postcondiciones}{}
	\UCitem{Reglas del negocio}{
		%\begin{itemize}	
		%	\item \BRref{RN-E1}{Alumno inscrito} 
		%\end{itemize}
		}
	\UCitem{Mensajes}{
		%\begin{itemize}	
		%	\item \MSGref{MSG-E8}{Calificaciones no registradas}
		%\end{itemize}
		}
\end{UseCase}


% ------------ Trayectoria principal, poner referencias a pantallas del sistema, reglas entre otras cosas que sean necesarias. ------------
\begin{UCtrayectoria}{Principal}
	\UCpaso Recibe el archivo oculto $x$ .
	\UCpaso Firma el archivo generando un nuevo archivo $y$.
	\UCpaso Guarda en la base de datos el archivo $y$.
	\UCpaso Envía al cliente el archivo $y$.
\end{UCtrayectoria}

% ---------------- Trayectorias alternativas -------------- Colocar los mensajes  {\bf MSG1-}

		

